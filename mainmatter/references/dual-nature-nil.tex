\title{The Dual Nature of NIL}
\author{Alexander Burger}
% Use \authorrunning{Short Title} for an abbreviated version of
% your contribution title if the original one is too long
\institute{\texttt{abu@software-lab.de}}
%
% Use the package "url.sty" to avoid
% problems with special characters
% used in your e-mail or web address
%


\maketitle

% 16jul12
% Alexander Burger

% \documentclass[10pt,a4paper]{article}
% \usepackage{graphicx}

% \textwidth 1.4\textwidth
% \textheight 1.125\textheight
% \oddsidemargin 0em
% \evensidemargin 0em
% \headsep 0em
% \parindent 0em
% \parskip 6pt

% \title{The Dual Nature of NIL}
% \author{Alexander Burger\\abu@software-lab.de}
% \date{2012-06-07}

% \begin{document}
% \maketitle

\begin{abstract}
  In this article, the dual nature of \texttt{NIL} as symbol and cons
  pair, a basic design requirement of PicoLisp, is discussed. 
\end{abstract}

\section{The Dual Nature of NIL}
\label{sec:dual-nature-nil}

\texttt{NIL} is a very fundamental piece of data. It has a dual
nature, being both a symbol \textit{and} a cons pair of the form
\texttt{(NIL . NIL)}. From the beginning, this has been a basic design
requirement of PicoLisp.

As shown in ``doc64/structures`` (similar also in ''doc/structures``):

\begin{wideverbatim}
      NIL:  /
            |
            V
      +-----+-----+-----+-----+
      |'LIN'|  /  |  /  |  /  |
      +-----+--+--+-----+-----+
\end{wideverbatim}

Physically, the pointer to \texttt{NIL} (shown with the 'V' in the
above diagram) is a true symbol, as it points at an offset of a
pointer size into the memory heap. Taken as such, \texttt{NIL} is a
normal symbol,

\begin{wideverbatim}
      NIL:  /
            |
            V
      +-----+-----+
      |'LIN'|  /  |
      +-----+--+--+
\end{wideverbatim}

having a tail with the characters 'N', 'I' and 'L' (the name), and a
value which in turn points to \texttt{NIL} (symbolized with the '/').

However, when this pointer is boldly used as a cell pointer (by
ignoring the pointer tags caused by the pointer size offset)

\begin{wideverbatim}
      NIL:  /
            |
            V
            +-----+-----+
            |  /  |  /  |
            +--+--+-----+
\end{wideverbatim}

then it is a normal cons pair, with \texttt{NIL} in its CAR and
\texttt{NIL} in its CDR, which just happens to be misaligned in memory
(i.e. at the place of a symbol).

This structure has great advantages. Any proper list (ending with
\texttt{NIL}) becomes sort of ``infinite'', allowing to take the CDR
as often as possible and still obtain \texttt{NIL} again and again.

Therefore, a function doesn't need to check whether it actually
received an argument or not. It can simply take the next argument with
CDR from the argument list, and doesn't see any difference between
\texttt{(foo NIL)} and \texttt{(foo)}. This makes interpretation both
simpler and faster.


% Local variables:
% mode: latex
% TeX-master: "../../editor.tex"
% End:
