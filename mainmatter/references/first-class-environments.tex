\title{First Class Environments}
\author{Alexander Burger}
% Use \authorrunning{Short Title} for an abbreviated version of
% your contribution title if the original one is too long
\institute{\texttt{abu@software-lab.de}}
%
% Use the package "url.sty" to avoid
% problems with special characters
% used in your e-mail or web address
%


\maketitle

% 16jul12
% Alexander Burger

% \documentclass[10pt,a4paper]{article}
% \usepackage{graphicx}

% \textwidth 1.4\textwidth
% \textheight 1.125\textheight
% \oddsidemargin 0em
% \evensidemargin 0em
% \headsep 0em
% \parindent 0em
% \parskip 6pt

% \title{First Class Environments}
% \author{Alexander Burger\\abu@software-lab.de}
% \date{2012-05-24}

% \begin{document}
% \maketitle

\begin{abstract}
  This article discusses the advantages and technical details of
  \emph{first class environments} in PicoLisp. First, the differences
  between \emph{dynamic binding} and \emph{lexical scoping} are
  highlighted, then \emph{first class data types} and \emph{first
    class environments} are discussed. 
\end{abstract}


\section{Dynamic Binding vs Lexical Scoping}
\label{sec:firstclass-dynamic-binding}

PicoLisp uses dynamic binding for symbolic variables. This means that
the value of a symbol is determined by the current runtime context,
not by the lexical context in the source file.

This has advantages in practical programming. It allows you to write
independent code fragments as \textbf{data} which can be passed to
other parts of the program to be later executed as \textbf{code} in
that context.

For example, the major part of the PicoLisp GUI framework consists of
code snippets, which are installed as event handlers, button actions
or update managers in GUI components.

In a lexically scoped language, you can't separate code from its
environment. A call like \texttt{(foo X Y)} has no meaning by itself
(unless \texttt{X} and \texttt{Y} are treated as ``free'' variables,
i.e. dynamically, not lexically). Instead, you must write a
\texttt{let} or \texttt{lambda} expression, or pass around full
function definitions as closures. All this is much more noisy
(consider the typical case of dozens of such fragments in a single GUI
page).

PicoLisp tries to be as dynamic as possible. This involves also (and
much more important than just variable bindings) things like
I/O-channels,
'\underline{make}\footnote{http://software-lab.de/doc/refM.html\#make}'
environments, the place of the currently executed method in the
inheritance hierarchy (for
'\underline{super}\footnote{http://software-lab.de/doc/refS.html\#super}'
and
'\underline{extra}\footnote{http://software-lab.de/doc/refE.html\#extra}')
or the current
'\underline{This}\footnote{http://software-lab.de/doc/refT.html\#This}'
object. A running code fragment can always assume to have access to
these environments.

This article is only concerned about variable bindings.

\section{First Class Data Type}
\label{sec:firstclass-first-class-data-type}

In programming languages, a first class data type can be created and manipulated
separately within the language, independent from its intended purpose.

For a \textbf{function}, for example, the intended purpose is to \textit{call} it. If it is
a first class function, it can be build, passed around etc., independent from
that. The two tasks (creating and calling the function) can be handled
independently within the language.

For an \textbf{environment}, the intended purpose is to
\begin{enumerate}
   \item activate it
   \item execute one or more expressions in it
   \item deactivate it, possibly restoring the previous environment
\end{enumerate}
If it is a first class environment it, can also be created and manipulated
within the language.

\section{First Class Environments}
\label{sec:firstclass-first-class-environments}

To make such an environment useful, it is important that the three tasks
(creation, activation/deactivation and execution) can be controlled
independently.

Let us consider the expression \texttt{(foo X Y)} again. Assume we want to execute it
in an environment where \texttt{X} is 3 and \texttt{Y} is 4. The direct way is
\begin{wideverbatim}
   (let (X 3 Y 4)
      (foo X Y) )
\end{wideverbatim}

This handles the three tasks all at once. It creates and activates the
environment, immediately executes the code, and then deactivates the
environment.

If the expression \texttt{(foo X Y)} is supplied from the application code,
stand-alone in one part of the application, and is supposed to run in some other
part of the application in an environment active there at that moment, it is
also no problem. It can simply be executed:
\begin{wideverbatim}
   (foo X Y)
\end{wideverbatim}

If, however, that application environment supposed to be manipulated
separately, for example because it is passed over across a HTTP transaction or
appears in an RPC call, then the
\begin{itemize}
   \item creation of that environment is typically done in the application
   \item activation of that environment is done in the GUI framework
   \item execution of the expression(s) is done in the application
   \item deactivation is done in the GUI framework again
\end{itemize}

\subsection{Creation}
\label{sec:firstclass-creation}

The environment is represented by a list of symbol-value pairs. It can be
created with the direct, explicit lisp operations, or more conveniently with the
'\underline{env}\footnote{http://software-lab.de/doc/refE.html\#env}' function.

If the environment is intended as a subset of the current environment, i.e. if
you simply want to create it with the current values of \texttt{X} and \texttt{Y}, you may
write
\begin{wideverbatim}
   : (setq Env (env '(X Y)))
   -> ((Y . 4) (X . 3))
\end{wideverbatim}

Otherwise, you may pass explit values
\begin{wideverbatim}
   : (setq Env (env 'X 3 'Y 4))
   -> ((Y . 4) (X . 3))
\end{wideverbatim}


\subsection{Activation}
\label{sec:firstclass-activation}

The simplest way to activate an environment would be to to iterate over the list
and '\underline{set}\footnote{http://software-lab.de/doc/refS.html\#set}' each symbol to its value.
This is normally not recommended, because it would not restore the previous
environment when done.

In most cases either '\underline{bind}\footnote{http://software-lab.de/doc/refB.html\#bind}' or
'\underline{job}\footnote{http://software-lab.de/doc/refJ.html\#job}' are used. The main difference
between these two functions is that job modifies the environment destructively,
to store modified values before restoring the previous environment.
\begin{wideverbatim}
   : (bind Env (* X Y))
   -> 12

   : (job Env (* X Y))
   -> 12
\end{wideverbatim}

If the environment is to be modified by the expression, use 'job':
\begin{wideverbatim}
   : Env
   -> ((Y . 4) (X . 3))

   : (job Env (* X (inc 'Y)))
   -> 15

   : Env
   -> ((Y . 5) (X . 3))
\end{wideverbatim}

These examples don't show the separation of activation and execution, as this
cannot be simply done in an isolated example. For a real-world use case, take a
look at the top-level GUI function in ``@lib/form.l``. The relevant part can be
reduced to
\begin{wideverbatim}
   (with "*App"                        # Point 'This' to the current form
      (job (: env)                     # Activate environment of that form
         (<post> ...
            (<hidden> ...)
            ...
            (if *PRG
               (let gui
                  ...
                  (htPrin "Prg") )     # Execute the GUI code
               ...
               (let gui
                  ...
                  (htPrin "Prg") )     # Execute the GUI code
\end{wideverbatim}

The GUI code which actually builds the page is in \texttt{"Prg"}. When it runs,
\texttt{This} and the desired environment (from the form's \texttt{env} property) are
properly activated.

A similar case can be found more down in ``@lib/form.l`` in 'postGui' to handle
HTTP POST and XMLHttpRequest events.
