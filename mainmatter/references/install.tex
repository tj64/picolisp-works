\title{INSTALL}
\author{Alexander Burger}
% Use \authorrunning{Short Title} for an abbreviated version of
% your contribution title if the original one is too long
\institute{\texttt{abu@software-lab.de}}
%
% Use the package "url.sty" to avoid
% problems with special characters
% used in your e-mail or web address
%


\maketitle

% \section{INSTALL}
% \label{sec:install-install}


\begin{abstract}
This is the INSTALL file from the PicoLisp distribution.   
\end{abstract}


\section{PicoLisp Installation}
\label{sec:install-picolisp-installation}

There is no 'configure' procedure, but the PicoLisp file structure is
simple enough to get along without it (we hope). It should compile and
run on GNU/Linux, FreeBSD, Mac OS X (Darwin), Cygwin/Win32, and
possibly other systems without problems.

PicoLisp supports two installation strategies: Local and Global.

The default (if you just download, unpack and compile the release) is a local
installation. It will not interfere in any way with the world outside its
directory. There is no need to touch any system locations, and you don't have to
be root to install it. Many different versions - or local modifications - of
PicoLisp can co-exist on a single machine.

For a global installation, allowing system-wide access to the executable and
library/documentation files, you can either install it from a ready-made
distribution, or set some symbolic links to one of the local installation
directories as described below.

Note that you are still free to have local installations along with a global
installation, and invoke them explicitly as desired.


\section{Local Installation}
\label{sec:install-local-installation}

\subsection{Unpack the distribution}
\label{sec:install-unpack-the-distribution}

\begin{wideverbatim}
  $ tar xfz picoLisp-XXX.tgz
\end{wideverbatim}

\subsection{Change the directory}
\label{sec:install-change-the-directory}

\begin{wideverbatim}
  $ cd picoLisp-XXX
\end{wideverbatim}

\subsection{Compile the PicoLisp interpreter}
\label{sec:install-compile-the-picolisp-interpreter}

\begin{wideverbatim}
  $ (cd src; make)
\end{wideverbatim}

Or - if you have an x86-64 system (under Linux or SunOS), or a ppc64
system (under Linux) - build the 64-bit version

\begin{wideverbatim}
  $ (cd src64; make)
\end{wideverbatim}

In both cases the executable bin/picolisp will be created.

To build the 64-bit version the first time (bootstrapping), you have the
following three options:

\begin{enumerate}
\item If a Java runtime system (version 1.6 or higher) is installed, it will
build right out of the box.
\item Otherwise, download one of the pre-generated "*.s" file packages
  \begin{itemize}
  \item http://software-lab.de/x86-64.linux.tgz
  \item http://software-lab.de/x86-64.sunOs.tgz
  \item http://software-lab.de/ppc64.linux.tgz
  \end{itemize}
\item Else, build a 32-bit version first, and use the resulting
  bin/picolisp to generate the "*.s" files:
\begin{wideverbatim}
$ (cd src; make)
$ (cd src64; make x86-64.linux)
\end{wideverbatim}
After that, the 64-bit binary can be used to rebuild itself.

Note that on the BSD family of operating systems, 'gmake' must be used
instead of 'make'.
\end{enumerate}


\section{Global Installation}
\label{sec:install-global-installation}

The recommended way for a global installation is to use a picolisp
package from the OS distribution.

If that is not available, you can (as root) create four symbolic links
from /usr/lib, /usr/share and /usr/bin to a local installation
directory

\begin{wideverbatim}
   # ln -s /<installdir> /usr/lib/picolisp
   # ln -s /<installdir> /usr/share/picolisp
   # ln -s /usr/lib/picolisp/bin/picolisp /usr/bin/picolisp
   # ln -s /usr/lib/picolisp/bin/pil /usr/bin/pil
\end{wideverbatim}


\section{Invocation}
\label{sec:install-invocation}

In a global installation, the 'pil' command should be used. You can
either start in plain or in debug mode. The difference is that for
debug mode the command is followed by single plus ('+') sign. The '+'
must be the very last argument on the command line.

\begin{wideverbatim}
   $ pil       # Plain mode
   :

   $ pil +     # Debug mode
   :
\end{wideverbatim}

In both cases, the colon ':' is PicoLisp's prompt. You may enter some
Lisp expression,

\begin{wideverbatim}
 : (+ 1 2 3)
   -> 6
\end{wideverbatim}

To exit the interpreter, enter

\begin{wideverbatim}
 : (bye)
\end{wideverbatim}

or just type Ctrl-D.

For a local invocation, specify a path name, e.g.

\begin{wideverbatim}
   $ ./pil     # Plain mode
   :

   $ ./pil +   # Debug mode
   :
\end{wideverbatim}

or
\begin{wideverbatim}
$ /home/app/pil  # Invoking a local installation from some other directory
\end{wideverbatim}

A shortcut for debug mode is the 'dbg' script:

\begin{wideverbatim}
   $ ./dbg
   :
\end{wideverbatim}

It is available only for local installaions, and is eqivalent to
\begin{wideverbatim}
  $ ./pil +
\end{wideverbatim}

Note that 'pil' can also serve as a template for your own stand-alone scripts.

If you just want to test the ready-to-run Ersatz PicoLisp (it needs a Java
runtime system), use

\begin{wideverbatim}
   $ ersatz/pil +
   :
\end{wideverbatim}

instead of './dbg' or './pil +'.

\section{Documentation}
\label{sec:install-documentation}

For further information, please look at "doc/index.html". There you find the
PicoLisp Reference Manual ("doc/ref.html"), the PicoLisp tutorials
("doc/tut.html", "doc/app.html", "doc/select.html" and "doc/native.html"), and
the frequently asked questions ("doc/faq.html").

For details about the 64-bit version, refer to "doc64/README", "doc64/asm" and
"doc64/structures".

As always, the most accurate and complete documentation is the source code ;-)
Included in the distribution are many utilities and pet projects, including
tests, demo databases and servers, games (chess, minesweeper), 3D animation
(flight simulator), and more.

Any feedback is welcome!
Hope you enjoy :-)

