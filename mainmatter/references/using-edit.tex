\title{Using 'edit'}
\author{Alexander Burger}
% Use \authorrunning{Short Title} for an abbreviated version of
% your contribution title if the original one is too long
\institute{\texttt{abu@software-lab.de}}
%
% Use the package "url.sty" to avoid
% problems with special characters
% used in your e-mail or web address
%


\maketitle

\begin{abstract}
  This articles is about browsing the database or arbitrary data structures
  and definitions with the powerful \texttt{edit} function. 
\end{abstract}

\section{Introduction}
\label{sec:edit-introduction}

 A quite powerful - but little known - function in PicoLisp is
`\href{http://software-lab.de/doc/refE.html#edit}{edit}'.
 It allows you to edit any Lisp symbol (intern, transient or extern), by
\href{http://software-lab.de/doc/refP.html#pretty}{pretty}-printing its
value and properties to a temporary file, calling an external editor,
and `\href{http://software-lab.de/doc/refR.html#read}{read}'ing back the
changes when done. For the external editor, currently `vim' is
supported.
 The clou: You can ``click'' on any other symbol somewhere embedded in the
nested structures of the value or property list, to have it added to the
editor screen, and thus browse through potentially the whole system.
 This works transparently not only for internal symbols, but also for
transient (which are normally not directly accessible) and external
(database) symbols. In the case of external symbols, it doesn't even
matter whether these are objects in a local database, or whether they
reside on remote machines in a distributed system (except that remote
objects cannot be modified).

\section{PicoLisp Symbols}
\label{sec:edit-picolisp-symbols}


A symbol in PicoLisp consists of three - possibly empty - components: A
value, a property list and a name. Of those, usually only the value and
the properties are modified during programming. The value is nothing
more than a special property, used implicitly for prominent purposes
like variable bindings or function definitions.
 There exist a large number of functions to access, set or modify a
symbol's value or property list. Setting the value and a few properties
of the symbol X:


\begin{wideverbatim}
: (setq X "Hello")
-> "Hello"

: (put 'X 'a 1)
-> 1

: (with 'X
   (=: b 2)
   (push (:: lst) "OK" '(a b c d) 17) )
-> 17
\end{wideverbatim}

These data can be accessed individually


\begin{wideverbatim}
: X
-> "Hello"

: (val 'X)
-> "Hello"

: (get 'X 'lst)
-> (17 (a b c d) "OK")
\end{wideverbatim}

or looked at as a whole, using the function
`\href{http://software-lab.de/doc/refS.html#show}{show}':


\begin{wideverbatim}
: (show 'X)
X "Hello"
   lst (17 (a b c d) "OK")
   b 2
   a 1
-> X
\end{wideverbatim}

show displays the symbol's name (here ``X''), then the value (here
``Hello'') on the same line, followed by all properties, each on its own
line indented by three spaces.


\section{Editing a Symbol}
\label{sec:edit-editing-a-symbol}


Instead of just showing the symbol to the console, you can use edit to
get a similar display in a `vim' session:


\begin{wideverbatim}
: (edit 'X)
\end{wideverbatim}

The editor's window will appear as:


\begin{wideverbatim}
X "Hello"
   a 1
   b 2
   lst (17 (a b c d) "OK")

(********)
\end{wideverbatim}

The difference to a plain show is that you can change the value or
properties.
 The pattern (********) is used by edit internally as a delimiter, and
should not be modified.
 For example, change the e in the value ``Hello'' to a, and the a in the
lst property to x:


\begin{wideverbatim}
X "Hallo"
   a 1
   b 2
   lst (17 (x b c d) "OK")

(********)
\end{wideverbatim}

then exit `vim' in the normal way, e.g. with ``:x''. On your console you
see


\begin{wideverbatim}
: (edit 'X)
# X redefined
# X lst redefined
-> NIL
\end{wideverbatim}

You can use show or other commands to see that X was indeed changed


\begin{wideverbatim}
: X
-> "Hallo"

: (get 'X 'lst)
-> (17 (x b c d) "OK")
\end{wideverbatim}

 It is even possible to start with an empty - or ``nonexistent'' - symbol


\begin{wideverbatim}
: (edit 'Y)
\end{wideverbatim}

It displays, as expected


\begin{wideverbatim}
Y NIL

(********)
\end{wideverbatim}

Now add a value and some properties


\begin{wideverbatim}
Y (This is a value)
   bar (and this is a property)
   foo 17

(********)
\end{wideverbatim}

After exiting the editor:


\begin{wideverbatim}
: (show 'Y)
Y (This is a value)
   foo 17
   bar (and this is a property)
-> Y
\end{wideverbatim}


\section{Browsing}
\label{sec:edit-browsing}


Now let's try the ``browsing'' capability mentioned above. When edit
starts up the `vim' editor, it defines two key mappings for that edit
session:

\begin{itemize}
\item Once you edit a symbol, you can move the cursor to the first
   character of some other symbol appearing in the value or properties,
   and press an upper-case `K'. This will cause that symbol to be added
   to the edit session, separated by another (********)
\item Pressing an upper-case `Q' goes one step back to the previous view
\end{itemize}

We can try this while editing X. Moving the cursor to the x in the lst
property, and hitting `K' gives:


\begin{wideverbatim}
x NIL

(********)

X "Hallo"
   a 1
   b 2
   lst (17 (x b c d) "OK")

(********)
\end{wideverbatim}

Now we see both x and X being edited. Unfortunately, x is not very
interesting here, as it has only the default value of NIL and no
properties.
 The same effect can be achieved by calling


\begin{wideverbatim}
(edit 'x 'X)
\end{wideverbatim}

You can pass any number of symbols to edit.
 A little more happens if we move down to lst again, and hit `K' on the
symbol d:


\begin{wideverbatim}
d (NIL (let *Dbg NIL (dbg ^)))
   *Dbg ((216 . "/usr/lib/picolisp/lib/debug.l"))

(********)

x NIL

(********)

X "Hallo"
   a 1
   b 2
   lst (17 (x b c d) "OK")

(********)
\end{wideverbatim}

Indeed, now we found something! This is not surprising, though, as
`\href{http://software-lab.de/doc/refD.html#d}{d}' has a definition in the
debugger context. The value is the function


\begin{wideverbatim}
(() (let *Dbg NIL (dbg ^)))
\end{wideverbatim}

and the `\href{http://software-lab.de/doc/refD.html#*Dbg}{*Dbg}' property
contains the file and line number of its source.

\section{Transient Symbols}
\label{sec:edit-transient-symbols}


We can use edit to inspect itself.


\begin{wideverbatim}
: (edit 'edit)
\end{wideverbatim}

The result looks meager


\begin{wideverbatim}
edit (@
   (let *Dbg NIL
      (setq "*F" (tmp '"edit.l"))
      (catch NIL ("edit" (rest))) ) )
   *Dbg ((6 . "lib/edit.l"))

(********)
\end{wideverbatim}

because - as can be seen in the fourth line - edit is a short function
wich calls ``edit'' (defined in a transient symbol) to do the actual work.
 The transient symbol ``edit'' is not directly reachable. In the REPL


\begin{wideverbatim}
: (pp '"edit")
(de "edit" . "edit")
-> "edit"
\end{wideverbatim}

we see just the string ``edit''.
 But if we edit edit, place the cursor on the first double quote
character of ``edit'' in line four, and press `K', we get


\begin{wideverbatim}
"edit" (("Lst")
   (let "N" 1
      (loop
         (out "*F"
            (setq
               "*Lst" (make
                  (for "S" "Lst"
                     ("loc" (printsp "S"))
                     ("loc" (val "S"))
...

(********)

edit (@
   (let *Dbg NIL
      (setq "*F" (tmp '"edit.l"))
      (catch NIL ("edit" (rest))) ) )
   *Dbg ((6 . "lib/edit.l"))

(********)
\end{wideverbatim}

BTW, you can see another transient function in line 8: ``loc''. You may
click on that one to see its definition.
 In contrast, if you look at the
`\href{http://software-lab.de/doc/refL.html#locale}{locale}' function


\begin{wideverbatim}
(edit 'locale)
\end{wideverbatim}

you'll find in there another, completely different, ``loc'' function. This
is an example for the locality of transient symbols. The two ``loc'''s
have nothing to do with each other, and don't conflict in their
definitions, yet you can see - and possibly change - them both
(separately) in the editor.


\section{Browsing the Database}
\label{sec:edit-browsing-the-database}


For the following examples we use the
\href{http://software-lab.de/doc/app.html#minApp}{demo application} in the
PicoLisp distribution. Start it as described in the
\href{http://software-lab.de/doc/app.html#getStarted}{Getting Started}
section:


\begin{wideverbatim}
$ ln -s /usr/share/picolisp/app
$ pil app/main.l -main -go +
\end{wideverbatim}

Then connect with a browser to
`\href{http://localhost:8080}{http://localhost:8080}' to get a PicoLisp
REPL prompt in your terminal window. Log in as ``admin'' / ``admin'' in the
browser GUI.
 Now you can navigate through the whole database. Start at an arbitrary
object. For a first overview, the
`\href{http://software-lab.de/doc/refD.html#*DB}{*DB}' root object is just
fine.


\begin{wideverbatim}
(edit *DB)
\end{wideverbatim}

You see the external symbol \{1\}, pointing to the base objects of the
entity classes.


\begin{wideverbatim}
{1} NIL
   +Role {3}
   +User {7}
   +Sal {16}
   +CuSu {31}
   +Item {32}
   +Ord {33}
   +Pos {34}

(********)
\end{wideverbatim}

The first one, \{3\}, is the base of the +Role entity. Move to the opening
brace and press `K'.


\begin{wideverbatim}
{3} NIL
   nm (3 . {D1})

(********)

{1} NIL
   +Role {3}
   +User {7}
   +Sal {16}
   +CuSu {31}
   +Item {32}
   +Ord {33}
   +Pos {34}

(********)
\end{wideverbatim}

We see that \{3\} contains only a single index, the nm (name) property of
roles. The number 3 tells us that this index tree has three nodes, and
its root node is \{D1\}.
 If we inspect that index root node, by clicking on \{D1\}


\begin{wideverbatim}
{D1} (NIL ("Accounting" NIL . {4}) ("Administration" NIL . {2}) ("Assistance" NIL . {5}))
...
\end{wideverbatim}

The first role in that list is ``Accounting'', the object \{4\}.
 \{4\} in turn leads us to


\begin{wideverbatim}
{4} (+Role)
   nm "Accounting"
   usr ({12} {11} {10})
   perm (Customer Item Order Report Delete)
...
\end{wideverbatim}

We see an object of class +Role (as expected), with the name
``Accounting'', the users in the usr list, and a list of permissions.
 Again, we might click on the first user, \{12\}


\begin{wideverbatim}
{12} (+User)
   role {4}  # (+Role)
   nam "Sandra Bullock"
   nm "sandy"
   pw "sandy"
...
\end{wideverbatim}

The role of that user points back to \{4\}, as we have a +Joint - a
bi-directional relation. We might verify this, by exiting edit with ``:q''
and call (vi `+User) to inspect the sources


\begin{wideverbatim}
...
### Role ###
(class +Role +Entity)

(rel nm (+Need +Key +String))          # Role name
(rel perm (+List +Symbol))             # Permission list
(rel usr (+List +Joint) role (+User))  # Associated users


### User ###
(class +User +Entity)

(rel nm (+Need +Key +String))          # User name
(rel pw (+String))                     # Password
(rel role (+Joint) usr (+Role))        # User role
...
\end{wideverbatim}

showing that role of +User points to a +Role object, and the usr
property of +Role has a list of +User objects.
 A similar information can also be obtained directly from the runtime
system. Go back to the user \{4\} again


\begin{wideverbatim}
(edit '{4})
\end{wideverbatim}

then click on the first character of \texttt{+Role} (i.e. the
`\texttt{+}' character) in the classes list of \{4\}.


\begin{wideverbatim}
+Role ((url> (Tab) (and (may RoleAdmin) (list "app/role.l" '*ID This)))
   +Entity )
   nm $53165764545663  # (+Need +Key +String)
   perm $53165764545716  # (+List +Symbol)
   usr $53165764545754  # (+List +Joint)
   Dbf (1 . 512)
   *Dbg ((39 . "lib/adm.l")
      (url> 26 . "app/er.l")
      (usr 43 . "lib/adm.l")
      (perm 42 . "lib/adm.l")
      (nm 41 . "lib/adm.l") )
...
\end{wideverbatim}

Note that now we are no longer in a database object, but in the class
definition. It shows that +Role defines a single method url>, is a
subclass of +Entity, and has relations nm, perm and usr. The property
Dbf used for database maintenance, and *Dbg holds debug information.
 You may experiment more. You can click on the `\$' of a relation
maintenance daemon object, and even on a commented symbol like +List or
+Joint.

\section{Debugging}
\label{sec:edit-debugging}


edit comes in handy also during debugging.
 You can easily do on-the-fly changes to a function, like inserting a
call to print a `\href{http://software-lab.de/doc/refM.html#msg}{msg}', or
setting some explicit breakpoint with
`\href{http://software-lab.de/doc/ref_.html#!}{!}', without actually
touching the source code.
 To edit a certain object in a large database, it is often easier to
find it by going to that object in the GUI. In the demo app, click on
the ``Orders'' menu item to the left, then on the `@' link in the leftmost
column of the first order. You should get a form with that order.
 Now, in the REPL, you can access the form that is currently shown in
the browser via the *Top global variable. You may look at it with (show
*Top), or edit it with (edit *Top).
 You get an awful lot of data, mostly for the GUI components in that
form. As before, you can click on any of them to see what they contain.
 Scrolling down a bit, there is an obj property. This is the database
object held by that form.


\begin{wideverbatim}
...
evt 0
obj {B7}  # (+Ord)
gui ($53165764713535
   $53165764713635
   $53165764713747
...
\end{wideverbatim}

Here, it is \{B7\}. Again, you can click on that,


\begin{wideverbatim}
{B7} (+Ord)
   nr 1
   dat 733027  # 2007-02-14
   cus {C3}  # (+CuSu)
   pos ({A1} {A2} {A3})
...
\end{wideverbatim}

and again you are ``in'' the database. You can follow the links to the
customer (the +CuSu object \{C3\}), or the three positions in that order
pos.
 Let's pick the first position \{A1\}


\begin{wideverbatim}
{A1} (+Pos)
   cnt 1
   pr 29900
   itm {B1}  # (+Item)
   ord {B7}  # (+Ord)
...
\end{wideverbatim}

and see a link to the item \{B1\}, and back to the order \{B7\}.
 The item \{B1\} leads us to


\begin{wideverbatim}
{B1} (+Item)
   nr 1
   inv 100
   pr 29900
   sup {C1}  # (+CuSu)
   nm "Main Part"
   ...
\end{wideverbatim}

in turn pointing to sup, the item's supplier \{C1\}, and so on.
 The database objects can be modified here in any conceivable way, but
you should be very sure about what you do, if you don't want an
inconsistent database. Relations involving index trees or ``joint''ed
objects need corresponding changes in other objects, and are better
avoided. In any case, a change to a DB object will only be manifest if
you enter (commit) after exiting from the editor.

\section{Distributed Database}
\label{sec:edit-distributed-database}


Though the demo app doesn't really make use of remote objects, it
contains a hook to experiment with them. If the demo application was
started as above, it automatically also listens on port 4040 for remote
requests.
 A distributed database requires some setup and administration. We don't
go into the details here, but a simple setup can be made by starting (in
addition to the app server above) a stand-alone PicoLisp interpreter in
another terminal window


\begin{wideverbatim}
$ pil +
\end{wideverbatim}

and initialize the *Ext variable as described in
`\href{http://software-lab.de/doc/refR.html#remote/2}{remote/2}'


\begin{wideverbatim}
(setq *Ext
   (mapcar
      '((@Host @Ext)
         (cons @Ext
            (curry (@Host @Ext (Sock)) (Obj)
               (when (or Sock (setq Sock (connect @Host 4040)))
                  (ext @Ext
                     (out Sock (pr (cons 'qsym Obj)))
                     (prog1 (in Sock (rd))
                        (unless @
                           (close Sock)
                           (off Sock) ) ) ) ) ) ) )
      '("localhost")
      '(20) ) )
\end{wideverbatim}

to let the system know where where to fetch remote objects from.
 If you started the remote server on another machine (you didn't forget
to open port 4040 in the firewall, did you?), supply its name or IP
address instead of ``localhost''.
 Then request the order with the number 1, and edit it:


\begin{wideverbatim}
: (let Sock (connect "localhost" 4040)
   (ext 20
      (out Sock (pr '(pr (db 'nr '+Ord 1))))
      (prog1 (in Sock (rd)) (close Sock)) ) )
-> {AF7}

: (edit @)
\end{wideverbatim}

From here on, continue as with the local database. Just that the (now
remote) order object \{B7\} appears locally as \{AF7\}.


\begin{wideverbatim}
{AF7} (+Ord)
   nr 1
   dat 733027  # 2007-02-14
   cus {AG3}  # (+CuSu)
   pos ({AE1} {AE2} {AE3})
\end{wideverbatim}

The same holds for the customer and the positions. Clicking on the first
position \{AE1\} gives


\begin{wideverbatim}
{AE1} (+Pos)
   cnt 1
   pr 29900
   itm {AF1}  # (+Item)
   ord {AF7}  # (+Ord)
...
\end{wideverbatim}

Except for the fact that the names of all external symbols appear with
an offset, everything else behaves like in the local case.


\begin{center}
\begin{tabular}{ll}
 27oct11  &  abu  \\
\end{tabular}
\end{center}


% Local variables:
% mode: latex
% TeX-master: "../../editor.tex"
% End:
