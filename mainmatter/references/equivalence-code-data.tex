\title{The Equivalence of Code and Data}
\author{Alexander Burger}
% Use \authorrunning{Short Title} for an abbreviated version of
% your contribution title if the original one is too long
\institute{\texttt{abu@software-lab.de}}
%
% Use the package "url.sty" to avoid
% problems with special characters
% used in your e-mail or web address
%


\maketitle

% % 16jul12
% % Alexander Burger

% \documentclass[10pt,a4paper]{article}
% \usepackage{graphicx}

% \textwidth 1.4\textwidth
% \textheight 1.125\textheight
% \oddsidemargin 0em
% \evensidemargin 0em
% \headsep 0em
% \parindent 0em
% \parskip 6pt

% \title{The Equivalence of Code and Data}
% \author{Alexander Burger\\abu@software-lab.de}
% \date{2011-07-29}

% \begin{document}
% \maketitle

\begin{abstract}
  This article demonstrates how the \emph{equivalence of code and
    data}, as a powerful abstraction instrument, supports the
  ``locality principle of structure, content and behavior'' in PicoLisp.
\end{abstract}

\section{The Equivalence of Code and Data}
\label{sec:equiv-equivalence-code-data}

Why does PicoLisp make so much fuss about that equivalence of code and data?
Answer: Because it is such a powerful abstraction instrument! It supports the
``locality principle of structure, content and behavior''.

Let me try to explain this using a GUI example, though the same principle also
applies to most other libraries and applications. Take a simple GUI page:
\begin{wideverbatim}
   (action
      (html 0 "Increment" "lib.css" NIL
         (form NIL
            (gui '(+Var +NumField) '*Num 9)
            (gui '(+JS +Button) "+" '(inc '*Num)) ) ) )
\end{wideverbatim}

This page shows two components, a numeric input field and a button labeled ``+``.
You can either enter a value into that field, or increment its value with the
button.

The page has a \textit{structure}, namely a HTML body containing a form, which in turn
contains the two components. It has \textit{content}, the numeric value shown in the
field. And it has \textit{behavior}, as the numeric field handles string conversion
and error checking, and the button increments the number.

Note that a fourth attribute of the page, the \textit{layout and style}, is not
discussed here. It is maintained separately in the CSS file.

The three attributes structure, content and behavior, however, cannot be
separated. This may be contrary to some schools of thought, which advocate a
separation of structure and content, and behavior anyway. But obviously the HTML
structure also holds content like the ``Increment'' page title or the ``+`` button
label. And this content might also be subject to behavior, for example the
button label may change dynamically:
\begin{wideverbatim}
   (gui '(+JS +Button)
      (if *VerboseDisplay "Increment" "+")
      '(inc '*Num) )
\end{wideverbatim}

Or the button may prompt a confirmation dialog to the user:
\begin{wideverbatim}
   (gui '(+JS +Button) "+"
      '(ask "Increment value?"
         (inc '*Num) ) )
\end{wideverbatim}

The equivalence of code and data makes the development of functions like
\texttt{html}, \texttt{form}, \texttt{gui} or \texttt{ask} particularly simple. The list
\begin{wideverbatim}
   (ask "Increment value?" (inc '*Num))
\end{wideverbatim}

is \textit{data} for the \texttt{gui} function. \texttt{gui} constructs a component of the class
\texttt{+Button}, and simply stows away that list into the new component. Later, when
the button is pressed, it is executed. It is just that simple.

\texttt{gui} itself doesn't understand anything about buttons and what they do with
these arguments. It is all data. From the view of the button, however, that list
is a piece of code, to be executed when the button is pressed.

Later, when \texttt{ask} is executed, the situation is similar. A dialog is shown
which displays the question ``Increment value?'', and a ``Yes'' and a ``No'' button.
The ``Yes'' button receives the expression \texttt{(inc '*Num)}, to increment the number
when pressed.

The example can be modified, to operate on database entities instead. Instead of
connecting the numeric field to a variable (with the \texttt{+Var} prefix class), we
connect it to the \texttt{count} property of the database object, and instead of
incrementing the variable, the button increments that object's count attribute:
\begin{wideverbatim}
   (gui '(+E/R +NumField) '(count : home obj) 9)
   (gui '(+JS +Button) "+" '(inc!> (: home obj) 'count))
\end{wideverbatim}

No need to write some separate database ``select'' or ``update'' functionality! It
is all in a single place.

This shows how easily any kind of application logic can be implemented using
this style. In a typical application hundreds or thousands of such components
are put together, so it is essential that this can be done with as little noise
and redundancy as possible.

This becomes especially evident if you compare this style with the GUI and
application strategies of other languages, for example in the RosettaCode
tasks
\underline{GUI component interaction}\footnote{http://rosettacode.org/wiki/GUI\_component\_interaction\#PicoLisp}
and
\underline{GUI enabling/disabling of controls}\footnote{http://rosettacode.org/wiki/GUI\_enabling/disabling\_of\_controls\#PicoLisp}.

In other environments, you usually end up defining constants, components and
functions in separate locations. This is the opposite of what PicoLisp
advocates, the ``locality principle of structure, content and behavior''.


