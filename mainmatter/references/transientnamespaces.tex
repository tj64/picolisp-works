\title{Transient Namespaces}
\author{Alexander Burger}
% Use \authorrunning{Short Title} for an abbreviated version of
% your contribution title if the original one is too long
\institute{\texttt{abu@software-lab.de}}
%
% Use the package "url.sty" to avoid
% problems with special characters
% used in your e-mail or web address
%


\maketitle

\begin{abstract}
  This articles is about using use a transient symbol (instead of an
  internal symbol) for a \textit{temporary} namespace.
\end{abstract}


\section{Introduction}
\label{sec:transient-namespaces-introduction}

A few days ago I hit upon an interesting concept: Transient Namespaces.

With Transient Namespaces I do not mean namespaces for transient symbols (there
is always only one single namespace for transient symbols at any moment), but to
use a transient symbol (instead of an internal symbol) for a \textit{temporary}
namespace.

It turns out that this makes quite some sense.

\section{Using transient symbols}
\label{sec:transient-namespaces-using-transient-symbols}

Last week I added an example for the Common-Lisp-style \texttt{DO*} function to
\underline{macros}\footnote{http://software-lab.de/doc/faq.html\#macros}
\begin{verbatim}
   (de do* "Args"
      (bind (mapcar car (car "Args"))
         (for "A" (car "Args")
            (set (car "A") (eval (cadr "A"))) )
         (until (eval (caadr "Args"))
            (run (cddr "Args"))
            (for "A" (car "Args")
               (and (cddr "A") (set (car "A") (run @))) ) )
         (run (cdadr "Args")) ) )
\end{verbatim}

It can be used as
\begin{verbatim}
   : (do* ((A 123) (B 'abc) (C 1 (inc C)))
      ((> C 7) (pack A B))
      (println C) )
   1
   2
   3
   4
   5
   6
   7
   -> "123abc"
\end{verbatim}

As you see, the above implementation of \texttt{do*} uses \textit{transient} symbols for
\texttt{"A"} and \texttt{"Args"}. This is the recommended way, to avoid symbol conflicts
(``captures'' in CL-terminology).

\section{Using internal symbols}
\label{sec:transient-namespaces-using-internal-symbols}

An implementation with \textit{internal} symbols
\begin{verbatim}
   (de do* Args
      (bind (mapcar car (car Args))
         (for A (car Args)
            (set (car A) (eval (cadr A))) )
         (until (eval (caadr Args))
            (run (cddr Args))
            (for A (car Args)
               (and (cddr A) (set (car A) (run @))) ) )
         (run (cdadr Args)) ) )
\end{verbatim}

\textit{looks} better, but gives a wrong result: Because the symbol \texttt{A} is captured,
the function returns \texttt{"abc"} instead of \texttt{"123abc"}.

So it is wise to use transient symbols, as in the first version.

\section{Using transient namespaces}
\label{sec:transient-namespaces-using-transient-namespaces}


\subsection{Implementation}
\label{sec:transient-namespaces-implementation}

However, as PicoLisp has namespaces for internal symbols (in the 64-bit version
since 3.0.7.9, and in Ersatz PicoLisp since 3.1.0.10), there is a third way:

Define \texttt{do*} in a separate namespace, and keep \texttt{A} and \texttt{Args} locally:
\begin{verbatim}
   (symbols 'flow 'pico)

   (local Args A)

   (de pico~do* Args
      (bind (mapcar car (car Args))
         (for A (car Args)
            (set (car A) (eval (cadr A))) )
         (until (eval (caadr Args))
            (run (cddr Args))
            (for A (car Args)
               (and (cddr A) (set (car A) (run @))) ) )
         (run (cdadr Args)) ) )
\end{verbatim}

The '\underline{symbols}\footnote{http://software-lab.de/doc/refS.html\#symbols}' call in \texttt{(symbols
'flow 'pico)} creates a new namespace \texttt{flow} as a copy of the
\underline{pico}\footnote{http://software-lab.de/doc/refP.html\#pico} namespace, and activates it.

\texttt{(local Args A)} ensures that \texttt{Args} and \texttt{A} are
\underline{local}\footnote{http://software-lab.de/doc/refL.html\#local} to the new namespace.

\texttt{pico~do*} refers to the existing or newly created symbol \texttt{do*} in the \texttt{pico}
namespace.

Now, back in the \texttt{pico} namespace (e.g. after hitting EOF), \texttt{do*} is available
\begin{verbatim}
   : (pp 'do*)
   (de do* "Args"
      (bind (mapcar car (car "Args"))
         (for "A" (car "Args")
            (set (car "A") (eval (cadr "A"))) )
         (until (eval (caadr "Args"))
            (run (cddr "Args"))
            (for "A" (car "Args")
               (and (cddr "A") (set (car "A") (run @))) ) )
         (run (cdadr "Args")) ) )
   -> do*
\end{verbatim}

As you see, \texttt{"Args"} and \texttt{"A"} are transient symbols relative to the \texttt{pico}
namespace, as the symbols \texttt{Args} and \texttt{A} are internal to the \texttt{flow}
namespace.

\subsection{Drawback}
\label{sec:transient-namespaces-drawback}


There is just one serious drawback to this approach. The call
\begin{verbatim}
   (symbols 'flow 'pico)
\end{verbatim}

creates the new namespace \texttt{flow} as a copy of \texttt{pico}. For a typical namespace,
this eats up about 25 kB of memory. If you get into the habit of creating a lot
of such namespaces, it could become quite wasteful.

Here \textbf{Transient Namespaces} come into play: Just create the namespace as
\begin{verbatim}
   (symbols "flow" 'pico)
\end{verbatim}

i.e. use the transient symbol \texttt{"flow"} instead of the internal symbol \texttt{flow}.
While it still creates a full copy of \texttt{pico}, it will become garbage as soon as
\texttt{"flow"} goes out of scope, and no space is wasted in the long term!

Normally, you would create a library for such flow functions in a separate
source file, containing more than a single function definition. To be sure to
avoid symbol conflicts also between those function, you may use more than one
call to \underline{local}\footnote{http://software-lab.de/doc/refL.html\#local} (analog to
'\underline{====}\footnote{http://software-lab.de/doc/ref\_.html\#====}' calls for separating
transient symbols):
\begin{verbatim}
   (symbols "flow" 'pico)

   (local Args A)

   (de pico~do* Args
      (bind (mapcar car (car Args))
         (for A (car Args)
            (set (car A) (eval (cadr A))) )
         (until (eval (caadr Args))
            (run (cddr Args))
            (for A (car Args)
               (and (cddr A) (set (car A) (run @))) ) )
         (run (cdadr Args)) ) )

   (local X Y Prg)

   (do pico~mumble (X Y . Prg)
      ... )
   ...
\end{verbatim}
