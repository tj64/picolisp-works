\title{Bash Completion}
\author{Alexander Burger}
% Use \authorrunning{Short Title} for an abbreviated version of
% your contribution title if the original one is too long
\institute{\texttt{abu@software-lab.de}}
%
% Use the package "url.sty" to avoid
% problems with special characters
% used in your e-mail or web address
%


\maketitle

% % 18jul12
% % Alexander Burger

% \documentclass[10pt,a4paper]{article}
% \usepackage{graphicx}

% \textwidth 1.4\textwidth
% \textheight 1.125\textheight
% \oddsidemargin 0em
% \evensidemargin 0em
% \headsep 0em
% \parindent 0em
% \parskip 6pt

% \title{Bash Completion}
% \author{Alexander Burger\\abu@software-lab.de}
% \date{2012-01-05}

% \begin{document}
% \maketitle

\begin{abstract}
This article describes how Bash completion works in PicoLisp.   
\end{abstract}

\section{Bash Completion}
\label{sec:bash-bash-completion}

Since picoLisp-3.0.9 there is support for Bash completion.

While it is not precisely the absolute killer-feature, bash completion is quite
handy when developing PicoLisp applications from the shell command line.

If you installed PicoLisp locally (i.e. not from a distribution package), you
need to copy two files
\begin{wideverbatim}
   $ cp lib/complete.l /usr/lib/picolisp/lib/
   $ cp lib/bash_completion /etc/bash_completion.d/pil
\end{wideverbatim}

As ever, source \texttt{. /etc/bash\_completion} in your .bashrc


Per default, if you hit the TAB key during command line input, Bash completes
things it knows about, like commands and path names.

PicoLisp -- in addition to normal path/file name arguments -- accepts two
particular types of arguments:
\begin{enumerate}
   \item If the argument's first character is '-', then the rest of that argument is
   taken as a Lisp function call (without the surrounding parentheses).
   \item If the argument's first character is '@', then it is interpreted as a path
   into the interpreter's installation directory.
\end{enumerate}

For (1), the expansion actually searches all built-in function names of the
given invocation. For example, entering
\begin{wideverbatim}
   $ pil -ver
\end{wideverbatim}

and then hitting TAB will expand to ``-version``.

The expansion also honors single or double quotes, to allow for function
arguments:
\begin{wideverbatim}
   $ pil -'pri
\end{wideverbatim}

or
\begin{wideverbatim}
   $ pil -"pri
\end{wideverbatim}

This expands to the printing functions.

For (2), the intended path name is properly expanded. This works regardless of
whether it is a global or a local installation, as it always searches the
invoked interpreter's environment.
\begin{wideverbatim}
   $ pil @lib/xh
\end{wideverbatim}

and
\begin{wideverbatim}
   $ <somePath>/pil @lib/xh
\end{wideverbatim}

both will expand to ``@lib/xhtml.l``.

As an extra goody, an empty argument expands to '+' (the trailing debug flag --
perhaps the most often needed command line argument).


