\title{README}
\author{Alexander Burger}
% Use \authorrunning{Short Title} for an abbreviated version of
% your contribution title if the original one is too long
\institute{\texttt{abu@software-lab.de}}
%
% Use the package "url.sty" to avoid
% problems with special characters
% used in your e-mail or web address
%


\maketitle

% \section{README}
% \label{sec:readme-readme}


\begin{abstract}
  This is the README file of the PicoLisp distribution. 
\end{abstract}

\section{The PicoLisp System}
\label{sec:readme-the-picolisp-system}


\begin{wideverbatim}
  _PI_co Lisp is not _CO_mmon Lisp
\end{wideverbatim}

PicoLisp can be viewed from two different aspects: As a general purpose
programming language, and a dedicated application server framework.


\subsection{Programming Language}
\label{sec:readme-programming-language}

As a programming language, PicoLisp provides a 1-to-1 mapping of a
clean and powerful Lisp derivate, to a simple and efficient virtual
machine. It supports persistent objects as a first class data type,
resulting in a database system of Entity/Relation classes and a
Prolog-like query language tightly integrated into the system.

The virtual machine was designed to be
\begin{description}
\item[Simple] The internal data structure should be as simple as
  possible. Only one single data structure is used to build all higher
  level constructs.
\item[Unlimited] There are no limits imposed upon the language due to
  limitations of the virtual machine architecture. That is, there is
  no upper bound in symbol name length, number digit counts, or data
  structure and buffer sizes, except for the total memory size of the
  host machine.
\item[Dynamic] Behavior should be as dynamic as possible ("run"-time
  vs. "compile"-time). All decisions are delayed till runtime where
  possible. This involves matters like memory management, dynamic
  symbol binding, and late method binding.
\item[Practical] PicoLisp is not just a toy of theoretical value.
  PicoLisp is used since 1988 in actual application development,
  research and production.
\end{description}
   
The language inherits the major advantages of classical Lisp systems like
\begin{itemize}
\item Dynamic data types and structures
\item Formal equivalence of code and data
\item Functional programming style
\item An interactive environment
\end{itemize}

PicoLisp is very different from any other Lisp dialect. This is partly
due to the above design principles, and partly due to its long
development history since 1984.

You can download the latest release version at
http://software-lab.de/down.html

\subsection{Application Server Framework}
\label{sec:readme-application-server-framework}

As an application server framework, PicoLisp provides for

\paragraph{NoSQL Database Management}
\label{par:readme-nosql-database-management}

  \begin{itemize}
    \item Index trees
    \item Object local indexes
    \item Entity/Relation classes
    \item Pilog (PicoLisp Prolog) queries
    \item Multi-user synchronization
    \item DB Garbage collection
    \item Journaling, Replication
  \end{itemize}

\paragraph{User Interface}
\label{par:readme-user-interface}

  \begin{itemize}
    \item Browser GUI
     \item X)HTML/CSS
     \item XMLHttpRequest/JavaScript
  \end{itemize}

\paragraph{Application Server} 
\label{par:readme-application-server}

  \begin{itemize}
    \item Process management
    \item Process family communication
    \item XML I/O
    \item Import/export
    \item User administration
    \item Internationalization
    \item Security
    \item Object linkage
    \item Postscript/Printing
  \end{itemize}


PicoLisp is not an IDE. All program development in Software Lab. is
done using the console, bash, vim and the Lisp interpreter.

The only type of GUI supported for applications is through a browser
via HTML. This makes the client side completely platform independent.
The GUI is created dynamically. Though it uses JavaScript and
XMLHttpRequest for speed improvements, it is fully functional also
without JavaScript or CSS.

The GUI is deeply integrated with - and generated dynamically from -
the application's data model. Because the application logic runs on
the server, multiple users can view and modify the same database
object without conflicts, everyone seeing changes done by other users
on her screen immediately due to the internal process and database
synchronization.

PicoLisp is free software, and you are welcome to use and redistribute
it under the conditions of the MIT/X11 License (see "COPYING").

It compiles and runs on current 32-bit GNU/Linux, FreeBSD, Mac OS X
(Darwin), Cygwin/Win32 (and possibly other) systems. A native 64-bit
version is available for x86-64/Linux, x86-64/SunOS and ppc64/Linux.


% Local variables:
% mode: latex
% TeX-master: "../../editor.tex"
% End:
