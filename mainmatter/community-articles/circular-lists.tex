\title{Wacky Stuff with circular Lists}
\author{Jos\'{e} Ignacio Romero}
% Use \authorrunning{Short Title} for an abbreviated version of
% your contribution title if the original one is too long
\institute{\texttt{jir@2.71828.com.ar}}
%
% Use the package "url.sty" to avoid
% problems with special characters
% used in your e-mail or web address
%

\maketitle


% 16jul12
% Thorsten Jolitz

% \documentclass[10pt,a4paper]{article}
% \usepackage{graphicx}

% \textwidth 1.4\textwidth
% \textheight 1.125\textheight
% \oddsidemargin 0em
% \evensidemargin 0em
% \headsep 0em
% \parindent 0em
% \parskip 6pt

% \title{Wacky stuff with circular lists in PicoLisp}
% \author{Thorsten Jolitz\\tj@data-driven.de}
% \date{2011-11-10}

% \begin{document}
% \maketitle

% \section*{Wacky stuff with circular lists in PicoLisp}


\begin{abstract}
  This article demonstrates and explains \emph{cirular lists} in
  PicoLisp.
\end{abstract}

\section{Example 1 with walk-through}
\label{sec:circ-lists-example-1-with-walk-through}

\texttt{ (and and (@ (min @ 5) (prinl @) (gt0 (dec @)) .)) }

I'm \texttt{eval}. I see a list, look inside, I see a symbol called
\texttt{'and}. I look at it's \texttt{val}. It's a number, thus a
function pointer, I call it with the rest of the list unevaluated.

I'm \texttt{doAnd}, I look at the list I was passed, I see a symbol
called \texttt{'and}, I evaluate it, a number came out, it's not
\texttt{NIL}, so I shove it in \texttt{@} and look at the next
\texttt{cadr}. It's a list, I evaluate it, has a symbol called
\texttt{@} in it's \texttt{car}, that symbol resolves to a number, a
pointer to \texttt{doAnd}, i call it with the rest of the list
unchanged.

I'm \texttt{doAnd}, I look at the list I was passed, the first
argument, evaluating it results in a function call that returns
\texttt{5}, it's not \texttt{NIL}, so I shove it to \texttt{@} and go
on. The next element is another list, a call to \texttt{prinl}
happens, it returned \texttt{5}, it's not \texttt{NIL}, so I shove it
to \texttt{@} and go on. Look at the next element, a call to
\texttt{(gt0 (dec @))}, returns \texttt{4}, that is not \texttt{NIL},
so I shove the \texttt{4} in \texttt{@} and go on. Looking at the next
\texttt{cadr} I see \texttt{@} again (but I don't realize, because I
don't know wether a list is circular or not), it evaluates to
\texttt{4}, so I shove it to \texttt{@} keep going.

\begin{wideverbatim}
I see (min @ 5) again.........
 ... 3
 ... 2
 ... 1
 ... (gt0 (dec @)) returns NIL here
\end{wideverbatim}


so I stop evaluating and return \texttt{NIL}

Back in the first \texttt{doAnd}, I see the previous call returned
\texttt{NIL}, so I stop evaluating right there and return
\texttt{NIL}.

\section{Example 2 with graphical depiction}
\label{sec:circ-lists-example-2-with-graphical-depiction}

Example 1 could be written with only one \texttt{'and} using the right
syntax (and losing part of it's \emph{rube-goldberg} appeal):

\texttt{ (and 5 . ((prinl @) (gt0 (dec @)) .)) }

This is how that \emph{S-expression} should look in memory, I omitted
the technically correct representation of numbers and symbols to keep
it simple:

\begin{wideverbatim}
  
                      ,-----------------------,
 +---+---+  +---+---+ '->+---+---+  +---+---+ |
 |   | ---->|     ------>|   | ---->|   | ----'
 + | +---+  + | +---+    + | +---+  + | +---+
   v          v            | ,--------'
  and         5            | |  +---+---+  +---+---+
                           | '->|   | ---->|   | / |
               ,-----------'    + | +---+  + | +---+
               |                  v          |
               |                 gt0         |  +---+---+  +---+---+
               |                             '->|   | ---->|   | / |
               |   +---+---+  +---+---+         + | +---+  + | +---+
               '-->|   | ---->|   | / |           v          v
                   + | +---+  + | +---+          dec         @
                     v          v
                   prinl        @

\end{wideverbatim}

