\title{Asynchronous Programming in PicoLisp}
\author{Henrik Sarvell}
% Use \authorrunning{Short Title} for an abbreviated version of
% your contribution title if the original one is too long
\institute{\texttt{hsarvell@gmail.com}}
%
% Use the package "url.sty" to avoid
% problems with special characters
% used in your e-mail or web address
%


\maketitle

% 16jul12
% Alexander Burger

% \documentclass[10pt,a4paper]{article}
% \usepackage{graphicx}

% \textwidth 1.4\textwidth
% \textheight 1.125\textheight
% \oddsidemargin 0em
% \evensidemargin 0em
% \headsep 0em
% \parindent 0em
% \parskip 6pt

% \title{Multiple asynchronous HTTP POSTs and 'call' versus 'in' - a word of warning}
% \author{Henrik Sarvell\\hsarvell@gmail.com}
% \date{2011-07-29}

% \begin{document}
% \maketitle


\section{Introduction}
\label{sec:bla}

If you've been using PicoLisp for some time doing nitty gritty stuff
like interfacing with such horrible things as PHP and other
abominations you can probably infer from the headline that the pitfall
here is quite trivial. However, it's the journey that is the important
part, not the somewhat anticlimactic end to this whole adventure.
Having said that, let's get to it.

% \section{Multiple asynchronous HTTP POSTs and 'call' versus 'in' - a
%   word of warning}
% \label{sec:async-program}

\section{Asynchronous Evaluation in PicoLisp}
\label{sec:bal}

In PicoLisp there is a clever way of asynchronous evaluation, it can
be useful if you:

\begin{enumerate}
\item Have a very computationally heavy problem, you fork the
  execution to utilize multiple cores and all that cheap RAM.
\item Need to query X services and don't care to wait for each call to
  finish before making the next one.
\end{enumerate}

This ``strategy'' involves

\underline{later}\footnote{http://software-lab.de/doc/refL.html\#later}
and
\underline{wait}\footnote{http://software-lab.de/doc/refW.html\#wait},
\underline{see documents for
  later}\footnote{http://software-lab.de/doc/refL.html\#later} in
particular.

My interest is web development so \#2 is standard for me and up until
now I've used
\underline{sockets}\footnote{http://software-lab.de/doc/refC.html\#connect}
in combination with
\underline{pr}\footnote{http://software-lab.de/doc/refP.html\#pr} and
\underline{rd}\footnote{http://software-lab.de/doc/refR.html\#rd}. The
``standard'' PicoLisp way to do these things if you will.

\section{HTTP only}
\label{sec:bla}

However, at the moment I'm working on something new and I want to work
solely with HTTP, not with PLIO, despite it being faster/more
efficient than HTTP. Let's just say that HTTP is the only allowed way
of communication in my ``spec''.

\subsection{Using \texttt{call}}
\label{sec:bla}

So I went happily on my way coding up all this communication with the
help of \texttt{CURL} and the
\underline{call}\footnote{http://software-lab.de/doc/refC.html\#call}
function.

Let's have a code listing:

\begin{wideverbatim}
(de callAll (Func Data)
   (let Result
      (make
         (for S (collect 'id '+Server)
            (later (chain (cons "void"))
               (list (; S id) (callOne Func Data S)))))
      (wait 30000 (not (memq "void" Result)))
      Result))
\end{wideverbatim}

Here you can see the core of the strategy, we loop through a list of
nodes (\texttt{+Server}) to query each and every for some arbitrary
data. What you don't see in the above listing is the contents of
\texttt{callOne} but it doesn't matter, we'll get to that soon enough.
In essence what we're doing here is getting all the nodes in my
massive distributed database (there are at the time of this writing
two of them running on my laptop but whatever).

These nodes might be more or less busy doing other stuff so response
times might vary and can you imagine if the first one takes 10 seconds
to reply and the second one takes 5 seconds? In a non-asynch world
that would add up to a grand total of 15 seconds for the query.

Luckily we use later and wait to avoid that and query them in parallel
for a grand total of 10 seconds. What's happening here is that we add
the results to a list, each entry in the list will start with
``void'', then we
\underline{fork}\footnote{http://software-lab.de/doc/refP.html\#pipe}.
Each fork will return a result (or not but we'll get to that very very
soon).

The second (list ... ) argument there is supposed to return a list
with the \texttt{id} of the node in the car and the result of the
query in the cdr. If a node times out one or more positions in the
list will still be ``void'' instead of something useful.

Finally the wait call will wait for 30 seconds or until all nodes have
returned something (i.e not timed out). After that wait we return the
list which I expected would look something like this for my specific
case (counting the objects in each node): ((1 ``1``) (2 ``0'')).

The problem was, the result didn't look like that, it looked more like
\texttt{(NIL NIL)}. I was using call like this: 

\begin{wideverbatim}
(call 'curl '-m 30 '-d
   ``key1=value1\&key2=value2`` ''http://localhost:8081/@exec`` )
\end{wideverbatim}

The reason for that is ye old copy paste problem, not even that in
this case actually. Just checking code from \underline{an old
  project}\footnote{http://vizreader.com} and seeing call in action
without realizing that in that case it was actually used correctly (I
wasn't interested in the returned content).

\subsection{Using \texttt{in}}
\label{sec:bla}

Yep, \texttt{call} returns \texttt{T}, not the actual result which
makes it useless in this case. I now use \texttt{in}, like this: 

\begin{wideverbatim}
(in (list ``curl'' ``-m`` Timeout ''-d`` ''key1=value1\&key2=value2``
       ''http://localhost:8081/@exec`` ) (till NIL T) ) 
\end{wideverbatim}

and all is well.

So now you don't have make the same mistake and spend two hours of
your life before you actually RTFM. However, by now you should've
realized that asynchronous logic can be indispensible when you don't
want to wait for too long for something good.


