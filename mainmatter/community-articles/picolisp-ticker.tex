\title{PicoLisp Ticker}
\author{Alexander Burger}
% Use \authorrunning{Short Title} for an abbreviated version of
% your contribution title if the original one is too long
\institute{\texttt{abu@software-lab.de}}
%
% Use the package "url.sty" to avoid
% problems with special characters
% used in your e-mail or web address
%


\maketitle

% 16jul12
% Alexander Burger

% \documentclass[10pt,a4paper]{article}
% \usepackage{graphicx}

% \textwidth 1.4\textwidth
% \textheight 1.125\textheight
% \oddsidemargin 0em
% \evensidemargin 0em
% \headsep 0em
% \parindent 0em
% \parskip 6pt

% \title{The PicoLisp Ticker}
% \author{Alexander Burger\\abu@software-lab.de}
% \date{2011-07-29}

% \begin{document}
% \maketitle

\begin{abstract}
This article describes how a \emph{PicoLisp Ticker} page is set up to produce
an endless stream of pseudo-text, and how the \emph{Googlebot} reacts
to such ``non-sensical'' data.       
\end{abstract}

\section{Producing an endless stream of pseudo-text}
\label{sec:pl-ticker-producing-an-endless-stream-of-pseudo-text}

Around end of May, I was playing with an algorithm I had received from \emph{Bengt
Grahn}, many years ago. A small program - it was even part of the PicoLisp
distribution (``misc/crap.l``) for many years - which when given an arbitrary
sample text in some language, produces an endless stream of pseudo-text which
strongly resembles that language.

It was fun, so I decided to set up a PicoLisp ``Ticker'' page, producing a stream
of ``news'': \underline{http://ticker.picolisp.com}\footnote{http://ticker.picolisp.com}

\section{Implementing a ticker page}
\label{sec:pl-ticker-implementing-a-ticker-page}

The source for the server is simple:

\begin{wideverbatim}
   (allowed ()
      *Page "!start" "@lib.css" "ticker.zip" )

   (load "@lib/http.l" "@lib/xhtml.l")
   (load "misc/crap.l")

   (one *Page)

\end{wideverbatim}

\begin{wideverbatim}

   (de start ()
      (seed (time))
      (html 0 "PicoLisp Ticker" "@lib.css" NIL
         (<h2> NIL "Page " *Page)
         (<div> 'em50
            (do 3 (<p> NIL (crap 4)))
            (<spread>
               (<href> "Sources" "ticker.zip")
               (<this> '*Page (inc *Page) "Next page") ) ) ) )

   (de main ()
      (learn "misc/ticker.txt") )

   (de go ()
      (server 21000 "!start") )
\end{wideverbatim}

The sample text for the learning phase, ``misc/ticker.txt``, is a
plain text version of the PicoLisp
\underline{FAQ}\footnote{http://software-lab.de/doc/faq.html}. The
complete source, including the text generator, can be downloaded via
the ``Sources'' link as ``ticker.zip''.

Now look at the ``Next page'' link, appearing on the bottom right of
the page. It always points to a page with a number one greater than
the current page, providing an unlimited supply of ticker pages.

I went ahead, and installed and started the server. To get some
logging, I inserted the line

\begin{wideverbatim}
   (out 2 (prinl (stamp) " {" *Url "} Page " *Page "  [" *Adr "] " *Agent))
\end{wideverbatim}

at the beginning of the \texttt{start} function.

\section{Googlebot in action}
\label{sec:pl-ticker-googlebot-in-action}

On June 18th I announced it on Twitter, and watched the log files. Immediately,
within one or two seconds (!), Googlebot accessed it:

\begin{wideverbatim}
  2011-06-18 11:22:04  Page 1  [66.249.71.139] Mozilla/5.0
  (compatible; Googlebot/2.1; +http://www.google.com/bot.html)
\end{wideverbatim}

Wow, I thought, that was fast! Don't know if this was just by chance, or if
Google always has such a close watch on Twitter.

Anyway, I was curious about what the search engine would do with such nonsense
text, and how it would handle the infinite number of pages. During the next
seconds and minutes, other bots and possibly human users accessed the ticker:

\begin{wideverbatim}

  2011-06-18 11:22:08 Page 1 [65.52.23.76] Mozilla/4.0 (compatible;
  MSIE 7.0; Windows NT 6.0)

  2011-06-18 11:22:10 Page 1 [65.52.4.133] Mozilla/4.0 (compatible;
  MSIE 7.0; Windows NT 6.0)

  2011-06-18 11:22:20 Page 1 [50.16.239.111] Mozilla/5.0 (compatible;
  Birubot/1.0) Gecko/2009032608 Firefox/3.0.8

  2011-06-18 11:29:52 Page 1 [174.129.42.87] Python-urllib/2.6

  2011-06-18 11:30:34 Page 1 [174.129.42.87] Python-urllib/2.6

  2011-06-18 11:33:54 Page 1 [89.151.99.92] Mozilla/5.0 (compatible;
  MSIE 6.0b; Windows NT 5.0) Gecko/2009011913 Firefox/3.0.6
  TweetmemeBot

  2011-06-18 11:33:5n4 Page 1 [89.151.99.92] Mozilla/5.0 (compatible;
  MSIE 6.0b; Windows NT 5.0) Gecko/2009011913 Firefox/3.0.6
  TweetmemeBot

  2011-06-18 13:47:21 Page 1 [190.175.174.220] Mozilla/5.0 (X11; U;
  Linux i686; en-US; rv:1.9.2.17) Gecko/20110428 Fedora/3.6.17-1.fc14
  Firefox/3.6.17

  2011-06-18 13:49:13 Page 2 [190.175.174.220] Mozilla/5.0 (X11; U;
  Linux i686; en-US; rv:1.9.2.17) Gecko/20110428 Fedora/3.6.17-1.fc14
  Firefox/3.6.17

  2011-06-18 13:49:21 Page 3 [190.175.174.220] Mozilla/5.0 (X11; U;
  Linux i686; en-US; rv:1.9.2.17) Gecko/20110428 Fedora/3.6.17-1.fc14
  Firefox/3.6.17

  2011-06-18 19:43:36 Page 1 [24.167.162.218] Mozilla/5.0 (X11; Linux
  x86_64) AppleWebKit/534.30 (KHTML, like Gecko) Chrome/12.0.n742.91
  Safari/534.30

\end{wideverbatim}

\begin{wideverbatim}

  2011-06-18 19:43:54 Page 2 [24.167.162.218] Mozilla/5.0 (X11; Linux
  x86_64) AppleWebKit/534.30 (KHTML, like Gecko) Chrome/12.0.742.91
  Safari/534.30

  2011-06-18 19:44:11 Page 3 [24.167.162.218] Mozilla/5.0 (X11; Linux
  x86_64) AppleWebKit/534.30 (KHTML, like Gecko) Chrome/12.0.742.91
  Safari/534.30

  2011-06-18 19:44:13 Page 4 [24.167.162.218] Mozillan/5.0 (X11; Linux
  x86_64) AppleWebKit/534.30 (KHTML, like Gecko) Chrome/12.0.742.91
  Safari/534.30

  2011-06-18 19:44:16 Page 5 [24.167.162.218] Mozilla/5.0 (X11; Linux
  x86_64) AppleWebKit/534.30 (KHTML, like Gecko) Chrome/12.0.742.91
  Safari/534.30

  2011-06-18 19:44:18 Page 6 [24.167.162.218] Mozilla/5.0 (X11; Linux
  x86_64) AppleWebKit/534.30 (KHTML, like Gecko) Chrome/12.0.742.91
  Safari/534.30

  2011-06-18 19:44:20 Page 7 [24.167.162.218] Mozilla/5.0 (X11; Linux
  x86_64) AppleWebKit/534.30 (KHTML, like Gecko) Chrome/12.0.742.91
  Safari/534.30

\end{wideverbatim}

Mr. Google came back the following day:

\begin{wideverbatim}

  2011-06-19 00:25:57 Page 2 [66.249.67.197] Mozilla/5.0 (compatible;
  Googlebot/2.1; +http://www.google.com/bot.html)

  2011-06-19 01:03:13 Page 3 [66.249.67.197] Mozilla/5.0 (compatible;
  Googlebot/2.1; +http://www.google.com/bot.html)

  2011-06-19 01:35:57 Page 4 [66.249.67.197] Mozilla/5.0 (compatible;
  Googlebot/2.1; +http://www.google.com/bot.html)

  2011-06-19 02:39:19 Page 5 [66.249.67.197] Mozilla/5.0 (compatible;
  Googlebot/2.1; +http://www.google.com/bot.html)

  2011-06-19 03:n43:39 Page 6 [66.249.67.197] Mozilla/5.0 (compatible;
  Googlebot/2.1; +http://www.google.com/bot.html)

  2011-06-19 04:17:02 Page 7 [66.249.67.197] Mozilla/5.0 (compatible;
  Googlebot/2.1; +http://www.google.com/bot.html)

\end{wideverbatim}

In between (not shown here) were also some accesses, probably by non-bots, who
usually gave up after a few pages.

Mr. Google, however, assiduously went through ``all'' pages. The page numbers
increased sequentially, but he also re-visited page 1, going up again. Now there
were several indexing threads, and by June 23rd the first one exceeded page 150.

I felt sorry for poor Googlebot, and installed a ``robots.txt'' the same day,
disallowing the ticker page for robots. I could see that several other bots
fetched ``robots.txt''. But not Google. Instead, it kept following the pages of
the ticker.

Then, finally, on July 5th, Googlebot looked at ``robots.txt'':

\begin{wideverbatim}

  "robots.txt" 2011-07-05 07:03:05 Mozilla/5.0 (compatible;
  Googlebot/2.1; +http://www.google.com/bot.html) ticker.picolisp.com
  "robots.txt: disallowed all"

\end{wideverbatim}

The indexing, however, went on. Excerpt:

\begin{wideverbatim}

  2011-07-05 04:27:46 {!start} Page 500 [66.249.71.203] Mozilla/5.0
  (compatible; Googlebot/2.1; +http://www.google.com/bot.html)

  2011-07-05 04:58:50 {!start} Page 501 [66.249.71.203] Mozilla/5.0
  (compatible; Googlebot/2.1; +http://www.google.com/bot.html)

  2011-07-05 05:30:24 {!start} Page 502 [66.249.71.203] Mozilla/5.0
  (compatible; Googlebot/2.1; +http://www.google.com/bot.html)

  2011-07-05 06:02:10 {!start} Page 503 [66.249.71.203] Mozilla/5.0
  (compatible; Googlebot/2.1; +http://www.google.com/bot.html)

  2011-07-05 06:32:14 {!start} Page 504 [66.249.71.203] Mozilla/5.0
  (compatible; Googlebot/2.1; +http://www.google.com/bot.html)

  2011-07-05 07:02:41 {!start} Page 505 [66.249.71.203] Mozilla/5.0
  (compatible; Googlebot/2.1; +http://www.google.com/bot.html)

  2011-07-05 08:02:31 {!start} Page 506 [66.249.71.203] Mozilla/5.0
  (compatible; Googlebot/2.1; +http://www.google.com/bot.html)

  2011-07-05 08:45:52 {!start} Page 507 [66.249.71.203] Mozilla/5.0
  (compatible; Googlebot/2.1; +http://www.google.com/bot.html)

\end{wideverbatim}

\begin{wideverbatim}

  2011-07-05 09:20:06 {!start} Page 508 [66.249.71.203] Mozilla/5.0
  (compatible; Googlebot/2.1; +http://www.google.com/bot.html)

  2011-07-05 09:51:49 {!start} Page 509 [66.249.71.203] Mozilla/5.0
  (compatible; Googlebot/2.1; +http://www.google.com/bot.html)

\end{wideverbatim}

Strange. I would have expected the indexing to stop after Page 505.

In fact, all other robots seem to obey ``robots.txt''. Mr. Google, however, even
started a new thread five days later again:

\begin{wideverbatim}

  2011-07-10 02:22:52 {!start} Page 1 [66.249.71.203] Mozilla/5.0
  (compatible; Googlebot/2.1; +http://www.google.com/bot.html)

\end{wideverbatim}

I should feel flattered, if the PicoLisp news ticker is so interesting!

How will that go on? As of today, we have reached

\begin{wideverbatim}

  2011-07-15 09:42:36 {!start} Page 879 [66.249.71.203] Mozilla/5.0
  (compatible; Googlebot/2.1; +http://www.google.com/bot.html)

\end{wideverbatim}

I'll stay tuned ...

Now ... 8 hours later (17:10):

As I wrote in the mailing list, I've extended ``robots.txt/default`` to exclude
also ``/21000/`` from ''picolisp.com``. This was 14:58, two hours ago. Meanwhile,

\begin{wideverbatim}

  2011-07-15 15:29:59 {!start} Page 800 [74.125.94.84] Mozilla/5.0
  (X11; Linux x86_64) AppleWebKit/534.24 (KHTML, like Gecko; Google
  Web Preview) Chrome/11.0.696 Safari/534.24

  2011-07-15 15:43:58 {!start} Page 889 [66.249.71.203] Mozilla/5.0
  (compatible; Googlebot/2.1; +http://www.google.com/bot.html)

  2011-07-15 16:13:54 {!start} Page 890 [66.249.71.203] Mozilla/5.0
  (compatible; Googlebot/2.1; +http://www.google.com/bot.html)

  2011-07-15 16:27:24 {!start} Page 1 [66.249.71.203] Mozilla/5.0
  (compatible; Googlebot/2.1; +http://www.google.com/bot.html)

\end{wideverbatim}

we still have accesses, and even a restart from page one (which I wouldn't have
expected now).

As I wrote, ``robots.txt/default`` now looks like this:
\begin{wideverbatim}
   (prinl "User-Agent: *")

   (prinl "Disallow:"
      (cond
         ((= *Host '`(chop "ticker.picolisp.com")) " /")
         ((= *Host '`(chop "picolisp.com")) " /21000/") ) )
\end{wideverbatim}

Looking at the returned contents:
\begin{wideverbatim}
   : (client "ticker.picolisp.com" 80 "robots.txt" (out NIL (echo)))
   User-Agent: *
   Disallow: /
\end{wideverbatim}

and
\begin{wideverbatim}
   : (client "picolisp.com" 80 "robots.txt" (out NIL (echo)))
   User-Agent: *
   Disallow: /21000/
\end{wideverbatim}


... next morning

Good! This helped. Googlebot seems to have stopped all traversals.

The log entry at 16:27 yesterday is the last one.

In summary, I can't blame Google. It was actually my fault not to explicitly
disallow /21000/, because for a bot the links looks like pointing to a different
site. Just disabling the ``root'' of the traversal is not enough; there is no
garbage collector involved.

