\title{The many uses of @ in PicoLisp}
\author{Thorsten Jolitz}
% Use \authorrunning{Short Title} for an abbreviated version of
% your contribution title if the original one is too long
\institute{\texttt{tjolitz@gmail.com}}
%
% Use the package "url.sty" to avoid
% problems with special characters
% used in your e-mail or web address
%


\maketitle

% 16jul12
% Alexander Burger

% \documentclass[10pt,a4paper]{article}
% \usepackage{graphicx}

% \textwidth 1.4\textwidth
% \textheight 1.125\textheight
% \oddsidemargin 0em
% \evensidemargin 0em
% \headsep 0em
% \parindent 0em
% \parskip 6pt

% \title{The many uses of @ in PicoLisp}
% \author{Thorsten Jolitz\\tj@data-driven.de}
% \date{2011-11-09}

% \begin{document}
% \maketitle


\begin{abstract}
  This article gives an overview over the many uses of \texttt{@} in
  PicoLisp.
\end{abstract}

\section{The \texttt{@} mark in PicoLisp}
\label{sec:at-mark-the-at-mark-in-picolisp}

The \emph{AT-mark} \texttt{@} is everywhere in PicoLisp source code, and
sometimes it is not obvious, at least for beginners, what the meaning
of \texttt{@} in the context at hand is.

Here is a table that summarizes all uses of \texttt{@} in PicoLisp,
giving examples and explanations, as well as links to related docs
with more information. It is probably necessary to read the docs first
to understand the compact information in the table. This summary
serves only as a quick overview, helping to find out the context and
meaning of an otherwise mysterious \texttt{@} mark in some PicoLisp
code.


\begin{wideverbatim}
                    All (?) possible uses of @ with examples and explanations
+------------------------------------------------------------------------------------------------+
|     context     |        use         |                meaning                |    reference    |
|-----------------+--------------------+---------------------------------------+-----------------|
| CAR of a lambda |                    | all arguments are evaluated and kept  | http://         |
| expression      | (de foo @ ...)     | internally in a list                  | software-lab.de |
|                 |                    |                                       | /doc/ref.html   |
|-----------------+--------------------+---------------------------------------+-----------------|
|                 |                    | the result of the last (3) evaluation | http://         |
| read-eval-loops | (- @ @@ @@@)       | (s) stored in the VAL of symbol       | software-lab.de |
|                 |                    |                                       | /doc/ref.html   |
|-----------------+--------------------+---------------------------------------+-----------------|
| flow- and logic | (while (read)      |                                       |                 |
| functions with  | (println @)), (and | store result of (the last)            | http://         |
| conditional     | and (@ (min @ 5)   | conditional expression                | software-lab.de |
| expressions     | (prinl @) (gt0     |                                       | /doc/ref.html   |
|                 | (dec @)) .))       |                                       |                 |
|-----------------+--------------------+---------------------------------------+-----------------|
| flow- and logic | (case @ ("^M" NIL) |                                       | http://         |
| functions with  | ("^J" "^M") (T @)  | store result of controlling           | software-lab.de |
| controlling     | )                  | expression                            | /doc/ref.html   |
| expressions     |                    |                                       |                 |
|-----------------+--------------------+---------------------------------------+-----------------|
| 'match' and     | (match '(@A Zeit)  |                                       | http://         |
| 'fill'          | '(Keine))          | Pattern Wildcard                      | software-lab.de |
|                 |                    |                                       | /doc/ref.html   |
|-----------------+--------------------+---------------------------------------+-----------------|
|                 |                    | replacing all occurrences of an       | http://         |
|                 | (text "abc @1 def  | at-mark "@", followed by one of the   | software-lab.de |
| 'text'          | @2" 'XYZ 123)      | letters "1" through "9", and "A"      | /doc/refT.html# |
|                 |                    | through "Z", with the corresponding   | text            |
|                 |                    | any argument.                         |                 |
|-----------------+--------------------+---------------------------------------+-----------------|
|                 | (load "@lib/       |                                       | http://         |
| path names      | misc.l")           | home directory substitution           | software-lab.de |
|                 |                    |                                       | /doc/tut.html   |
|-----------------+--------------------+---------------------------------------+-----------------|
|                 | (be likes (John    |                                       | http://         |
| Pilog           | @X))               | Pilog variable                        | software-lab.de |
|                 |                    |                                       | /doc/ref.html   |
|-----------------+--------------------+---------------------------------------+-----------------|
|                 | (be likes (John    |                                       | http://         |
| Pilog           | @))                | Anonymous Pilog variable              | software-lab.de |
|                 |                    |                                       | /doc/ref.html   |
|-----------------+--------------------+---------------------------------------+-----------------|
|                 | (native "@"        | (64-bit version only) Calls a native  | http://         |
| shared object   | "getenv" 'S        | C function. The first argument should | software-lab.de |
| libraries       | "TERM") # Same as  | specify a shared object library, e.g. | /doc/refN.html# |
|                 | (sys "TERM")       | "@" (here @ as transient symbol       | native          |
|                 |                    | stands for the current main program). |                 |
+------------------------------------------------------------------------------------------------+


\end{wideverbatim}

% \end{document}

% Local variables:
% mode: latex
% TeX-master: "../../editor.tex"
% End:
