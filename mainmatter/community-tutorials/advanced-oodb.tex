\title{Advanced OODB in PicoLisp}
\author{Henrik Sarvell}
% Use \authorrunning{Short Title} for an abbreviated version of
% your contribution title if the original one is too long
\institute{\texttt{hsarvell@gmail.com}}
%
% Use the package "url.sty" to avoid
% problems with special characters
% used in your e-mail or web address
%


\maketitle

\section{Assumptions}
\label{sec:advanced-oodb}

In this tutorial I will assume you've already glanced at the documents I
linked to in the prior article. I hope you still have cms.db intact.


\begin{wideverbatim}
# Copy paste relations and classes from the prior 
# article here, nothing has changed

(pool "cms.db")

(? 
 (select (@A) 
         ((tag +Tag "pc" (tags +Article)))
         (tolr "computer" @A body) 
 (show @A)))
\end{wideverbatim}

\section{Using select}
\label{sec:advanced-oodb}

A lot easier than the approach we employed in the prior tutorial,
\textbf{select} will first take generators, in our case only
\texttt{(tag +Tag ``pc'' (tags +Article))}. It will go through the
tags and when we find one with tag ``pc'' we will continue and
retrieve all articles connected through the tags reference. Next comes
an arbitrary amount of filter clausfes, in our case only one: \texttt{(tolr
``computer'' @A body)}. \textbf{tolr} is a shortcut for tolerant which
means we do partials too. Try replacing ``computer'' with ``comp'' and
the result will be the same.

The result is of course a list with all articles who are tagged with
``pc'' and contain the substring ``computer'' in their article bodies.

\begin{wideverbatim}
(? 
 (select (@A) 
         ((tag +Tag "pc" (tags +Article)))
         (tolr "computer" @A body)
         (same "sam" @A author username)
         (same "tech" @A folder slug)
 (show @A)))
\end{wideverbatim}

The only addition here is more filtering through the author and folder
references. We now get a list of all articles tagged with ``pc'', written
by Sam, containing ``computer'' in their bodies and located in the ``tech''
folder, feel free to contemplate the equivalent SQL \dots{}

\section{Pilog example}
\label{sec:advanced-oodb}

\subsection{Select and insert}
\label{sec:advanced-oodb}

As you might know already this all works through
\href{http://www.software-lab.de/ref.html#pilog}{Pilog} which is a
PicoLisp implementation of
\href{http://en.wikipedia.org/wiki/Prolog}{Prolog}. To understand how
it works let's play around a little with \textbf{be} and a Pilog
version of the
\href{http://www.swi-prolog.org/documentation.html}{SWI-Prolog
  tutorial}:


\begin{wideverbatim}
(be Indian (vindaloo))
(be Indian (dahl))
(be Indian (tandoori))
(be Indian (kurma))
(be mild (dahl))
(be mild (tandoori))
(be mild (kurma))
(be Chinese (chow-mein))
(be Chinese (chop-suey))
(be Chinese (sweet-and-sour))
(be Italian (pizza))
(be Italian (spaghetti))

(be likes (Sam @F) (Indian @F) (mild @F))
(be likes (Sam @F) (Chinese @F))
(be likes (Sam @F) (Italian @F))
(be likes (Sam chips))

(? (likes Sam @F))
\end{wideverbatim}

Yep, Sam likes Indian food but only the mild curries, vindaloo doesn't
fall into that category, that's why it's missing in the output. This
is the mechanism behind our OODB queries. The \textbf{same} and
\textbf{tolr} keywords we use above are in fact set with \textbf{be}
in pilog.l.

Let's continue with some simple pagination:

\begin{wideverbatim}
(new! '(+Article) 'slug "new-pcs-in-2008" 'headline "New PC's in 2008" 
'body "An article about all the new PC's in 2008." 'author (db 'username '+User "sam"))

(setq *Query (goal '(@Headline "2008" (db headline +Article @Headline @A))))

(do 2 (bind (prove *Query) (println (get @A 'headline))))
\end{wideverbatim}

There are two new things here \textbf{goal} and \textbf{prove}. Until
now we have used the shortcut \textbf{?} to do both at the same time.
Goal will prepare a Lisp statement by turning it into a valid query
that the Pilog engine can prove or disprove like we are doing above
with Sam and his food. Try printing \texttt{*Query} to see what it
looks like. In this case repeated calls to the last line will retrieve
the results two by two because prove will return the next result which
makes it ideal to call repeatedly to get the next two and then the
next two and so on. Try this instead:


\begin{wideverbatim}
(setq *Query (goal '(@Headline "2008" (db headline +Article @Headline @A))))
(do 1 (bind (prove *Query) (println (get @A 'headline))))
(do 1 (bind (prove *Query) (println (get @A 'headline))))
\end{wideverbatim}

In a ``sharp'' situation we could have called that last line to fetch
the next result when our user presses a next button for instance.
Notice also the necessary ``preparation'' of 2008 with
\texttt{@Headline} at the beginning of the quoted list we pass to
goal.

\subsection{Updating and Deleting}
\label{sec:advanced-oodb}

Until now we have only selected and inserted things, let's look at ways
to change and delete our data. As you know most of our articles are
tagged with ``fun'', this is how we could remove that tag from our tech
folder/article:


\begin{wideverbatim}
(del!> (db 'slug '+Article "tech") 'tags (db 'tag '+Tag "fun"))
(mapcar show (collect 'slug '+Article))
\end{wideverbatim}

Note how \textbf{del!\textgreater} automatically deletes the fun tag from the reference
list in the tech article. Updating a pure value is just a matter of
putting again:


\begin{wideverbatim}
(put!> (db 'slug '+Article "tech") 'headline "The technology folder")
(mapcar show (collect 'slug '+Article))
\end{wideverbatim}

Let's get rid of the fun tag altogether:


\begin{wideverbatim}
(lose!> (db 'tag '+Tag "fun"))
(mapcar show (collect 'slug '+Article))
\end{wideverbatim}

The tag is gone but the references are still there, in my case the fun
tag was \texttt{\{P\}} and the \texttt{\{P\}} still shows in the tag
list of each article. So we have a case of orphaning, sometimes it's a
wanted behavior, not now though so let's get rid of the reference:

\begin{wideverbatim}
(for Article (collect 'tags '+Article '{P})
     (put!> Article 'tags (delete '{P} (get Article 'tags))))

(mapcar show (collect 'slug '+Article))
\end{wideverbatim}


The \textbf{for} loop is the PicoLisp version of the old ``for in'' or
``for each''. We collect all articles that are referring to the fun
tag (\texttt{\{P\}}). After that we get the tag list in question, delete the
fun reference and finally put it back. With that in mind we could
create a custom \textbf{lose} method:

\begin{wideverbatim}
(extend +Entity)
(dm loseref!> ()
    (for Child (var: Cascade) 
         (let (ChildClass (car Child) ChildRef (cdr Child)) 
           (for Element (collect ChildRef ChildClass This)
                (put!> Element ChildRef (delete This (get Element ChildRef))))))
    (lose!> This))

(class +User +Entity)
(rel username (+Need +Key +String))
(rel password (+Need +String))

(class +Article +Entity)
(rel slug     (+Need +Key +String))
(rel headline (+Need +Idx +String))
(rel body     (+Need +String))
(rel author   (+Ref +Link) NIL (+User))
(rel folder   (+Ref +Link) NIL (+Article))
(rel tags     (+List +Ref +Link) NIL (+Tag))

\end{wideverbatim}

\begin{wideverbatim}

(class +Tag +Entity)
(rel tag (+Need +Key +String))
(var Cascade . ((+Article . tags)))

(pool "cms.db")

(loseref!> (db 'tag '+Tag "pc"))

(mapcar show (collect 'slug '+Article))
\end{wideverbatim}

This is just repetition of the above with the addition of a Cascade
list that we loop through to find which classes are affected (in our
case only \texttt{+Article}) and the name of the reference to use
\texttt{(tags)}. Note the use of \emph{class variables} (which I
forgot to mention in the simple OO tutorial). We initiate a class
variable with \textbf{var} and retrieve it with \textbf{var:}.

That was one way of doing it, another is to inspect the relations and
use that information to do the cleanup. The problem with this is that it
will delete all references in all tagged objects. Pretend you had
something else in the system that you are tagging, \texttt{+Novel}(s) for
instance. If you only wanted to remove the specific tag for articles,
not novels you would have to specifically state that somewhere and you
are back to something like the above. However, if this is not a problem
you could do like this instead:

\begin{wideverbatim}
(extend +Entity)
(dm loseref!> ()
    (for Child (getRefs> This) 
         (let (ChildClass (car Child) ChildRef (cdr Child)) 
           (for Element (collect ChildRef ChildClass This)
                (put!> Element ChildRef (delete This (get Element ChildRef))))))
    (lose!> This))

(dm getRefs> ()
    (make 
     (for Class (all)
          (when (isa '+Entity Class) 
            (for El (getl Class) 
                 (and 
                  (isa '(+Ref +Link) (car El))
                  (= (list *Class) (get El 1 'type))
                  (link (cons Class (cdr El)))))))))

\end{wideverbatim}

\begin{wideverbatim}


# Relations here without the (var Cascade . ((+Article . tags))) line.

(pool "cms.db")

(loseref!> (db 'tag '+Tag "scuba"))

(mapcar show (collect 'slug '+Article))
\end{wideverbatim}

\texttt{GetRefs>} will loop through all symbols currently loaded, when
the symbol is an \texttt{+Entity} we fetch the whole property list
from the symbol.

We loop through all properties and check if they have \texttt{+Ref
  +Link}, if yes we check if the current class accessed through the
\texttt{*Class} global is equal to the type we fetch from the
\texttt{car} of \texttt{El}, yes \texttt{(get (car El) ‘type)} would
have worked too. If they are equal we move on and \textbf{link} a
\textbf{cons} pair to the list.

We get the name of the relation with \texttt{(cdr El)}, the result is
identical to the explicitly set \texttt{((+Article . tags))} in the
prior example.

