\title{More OO in PicoLisp}
\author{Henrik Sarvell}
% Use \authorrunning{Short Title} for an abbreviated version of
% your contribution title if the original one is too long
\institute{\texttt{hsarvell@gmail.com}}
%
% Use the package "url.sty" to avoid
% problems with special characters
% used in your e-mail or web address
%


\maketitle

\section{Simple single inheritance}
\label{sec:more-oo}

Let's first look at a simple single inheritance example:

\begin{wideverbatim}
(class +Species)

(dm T (Species)
    (=: species Species))

(class +Person +Species)

(dm T (Age Name)
    (=: age Age)
    (=: name Name)
    (super "H. sapiens"))

(setq *John (new '(+Person) 65 'John))

(prin (get *John 'species))
\end{wideverbatim}

Nothing really surprising here, the hierarchy is set from left to
right in the class definition, that's why \texttt{+Person} comes
before \texttt{+Species}: \texttt{(class +Person +Species)}.

\begin{wideverbatim}
(class +Animal +Species)

(dm T (Age Name . @)
    (=: age Age)
    (=: name Name)
    (super (car (rest))))

(setq *John (new '(+Animal) 25 'John "G. gorilla"))

(prin (get *John 'species))
\end{wideverbatim}

So John is now a gorilla instead. It's starting to get interesting,
the above way of doing the argument list will enable us to pass a
variable amount of arguments, the ones ending up in the \texttt{@} can
be retrieved with \textbf{rest}, in this case it's the third one, ``G.
gorilla''. Note that rest will retrieve a list, even if there is only
one argument left in it, hence the use of car in this case to get only
the first element. In this case ``G. gorilla'' will be passed to the
constructor of \texttt{+Species} through \textbf{super}.

There is however at better way of doing the argument handling than
\texttt{(car (rest))}:

\begin{wideverbatim}
(class +Species)

(dm T (Species . @)
    (=: species Species)
    (=: description (next)))

(class +Animal +Species)

(dm T (Age Name . @)
    (=: age Age)
    (=: name Name)
    (super (next) (car (rest))))

(setq *John (new '(+Animal) 25 'John "G. gorilla" 
      "Big leaf eating primate"))

(prin (get *John 'description))
\end{wideverbatim}

That's right, \textbf{next} will retrieve the next value from
\textbf{rest} and in the process remove the value by reference as
demonstrated by the fact that \texttt{(car (rest))} gives the proper
result in this case. You would however want to use just another
\texttt{(next)} instead in a situation like the one above.


\section{Multiple inheritance}
\label{sec:more-oo}

Let's take a look at a more complex ``horizontal'' example, for
another example check out the
\href{http://www.software-lab.de/tut.html#oop}{OOP section} in Alex's
\href{http://www.software-lab.de/tut.html}{Pico Lisp tutorial}. I
found it helpful, even in the very beginning.

\begin{wideverbatim}
(class +Species)
(dm T (@)
    (=: species (next)))

(dm show> ()
    (pack " Species:" (: species)))

(class +Body)
(dm T (Age Weight Height . @)
    (=: age Age)
    (=: weight Weight)
    (=: height Height)
    (pass extra))

(dm show> ()
    (pack " Age:" (: age) " Weight(kg):
     " (: weight) " Height(cm):" (: height) (extra)))

(class +Person)
(dm T (Name Occupation . @)
    (=: occupation Occupation)
    (=: name Name)
    (pass extra))

(dm show> ()
    (pack " Name:" (: name) " Occupation:" (: occupation) (extra)))

(setq *John (new '(+Person +Body +Species) 
      "John" "Teacher" 65 85 180 "H. Sapiens"))

(prin (show> *John))
\end{wideverbatim}

In this case John is the combination of three different classes at once
and the way to call the next function in the horizontal hierarchy (from
left to right) is to use \textbf{extra}. In this case \textbf{pass} is a shortcut for
sending \textbf{rest} to the next constructor. Let's introduce a new class:


\begin{wideverbatim}
(class +Location)
(dm T (Location . @)
    (=: location Location)
    (pass extra))

(dm show> ()
    (pack " Location:" (: location) (extra)))
    
(setq *John (new '(+Person +Body +Location +Species)
 "John" "Teacher" 65 85 180 "New York" "H. Sapiens"))

(prin (show> *John))
\end{wideverbatim}

In this case the two middle classes \texttt{+Body} and
\texttt{+Location} are interchangeable:


\begin{wideverbatim}
(setq *John (new '(+Person +Location +Body +Species)
 "John" "Teacher" "New York" 65 85 180 "H. Sapiens"))
\end{wideverbatim}

This is basically the same thing since it's not a hierarchy in the
traditional sense, the two middle classes do not have to know what is
behind and after in the chain.

This way of using chained relations is important, it is used for
instance in the GUI framework to validate forms by simply having a
\texttt{chk>} function that uses \texttt{(pass extra)} to walk the
chain, each check is of course unique for each input type,
\texttt{+TextField} and \texttt{+NumField} are two examples.


\section{Class extension on demand}
\label{sec:more-oo}

Classes can be extended on demand:

\begin{wideverbatim}
(setq *John (new '(+Person +Body +Location +Species) 
"John" "Teacher" 65 85 180 "New York" "H. Sapiens"))

(extend +Body)
(dm bmi> ()
    (*/ (: weight) 10000 (** (: height) 2) ))

(prin (bmi> *John))
\end{wideverbatim}

The \texttt{**/*} function is necessary to handle cases like this in
order to get the proper result by first multiplying the weight with
10000 and then dividing that result with 180*180. PicoLisp doesn't
handle intermediate floating point numbers automatically. If you
wanted an output with one decimal for instance you could do:


\begin{wideverbatim}
(dm bmi> ()
    (format (*/ (: weight) 100000 (** (: height) 2)) 1))
\end{wideverbatim}

In this case \textbf{format} will take the number 262 and turn it into 26.1. A
more on the fly method of accomplishing the above would be:

\begin{wideverbatim}
(setq *John (new '(+Person +Body +Location +Species) 
"John" "Teacher" 65 85 180 "New York" "H. Sapiens"))

(push *John 
      '(bmi> () (format (*/ (: weight) 100000 (** (: height) 2)) 1)))

(prin (bmi> *John))
\end{wideverbatim}

Or maybe the \texttt{bmi>} method is already part of some old class in
some library and now our program discovers that John needs that class
too:

\begin{wideverbatim}
(class +WeightHandler)
(dm bmi> ()
    (format (*/ (: weight) 100000 (** (: height) 2)) 1))

(setq *John (new '(+Person +Body +Location +Species) 
"John" "Teacher" 65 85 180 "New York" "H. Sapiens"))

(unless (method 'bmi> *John) (push *John '+WeightHandler))

(prin (bmi> *John))
\end{wideverbatim}

In this case \textbf{method} will return \texttt{NIL} if John doesn't
already have the ability to calculate his BMI, in that case we simply
push the \texttt{WeightHandler} class in front of his other classes.

I don't think I've ever experienced a more flexible object system.
