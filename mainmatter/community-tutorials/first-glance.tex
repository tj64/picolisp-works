\title{PicoLisp at first glance}
\author{Henrik Sarvell}
% Use \authorrunning{Short Title} for an abbreviated version of
% your contribution title if the original one is too long
\institute{\texttt{hsarvell@gmail.com}}
%
% Use the package "url.sty" to avoid
% problems with special characters
% used in your e-mail or web address
%


\maketitle


\begin{abstract}

  This is the first in a series of articles for absolute PicoLisp
  beginners. The series will contrast PicoLisp against PHP in the
  examples to make it easier for C-people to understand.
  
\end{abstract}

\section{PicoLisp at first glance}
\label{sec:first-glance}

As I announced earlier, the plan was to create some small proof of
concept web-thing in c-lisp but it didn't work out. Instead I ended up
doing just that in \href{http://www.software-lab.de/down.html}{PicoLisp}
instead, I guess me and c-lisp was not meant to be. Anyway, PicoLisp is
created by Alex Burger without whose help and patience I wouldn't have
gone very far.

Actually at first PicoLisp seemed too good to be true, just a few things:
\begin{itemize}
\item Good documentation (rare in the Lisp world).
\item Object persistence/database.
\item Totally dynamically interpreted (no need for macros).
\item UTF--8 support out of the box.
\item Built in webserver.
\item GUI framework to render HTML.
\item Mailing function with attachments (SMTP).
\item File uploads.
\end{itemize}

OK so maybe it was a little too good to be true, the documentation is
good but covers far from everything, and the reference is what it is, a
reference for people who already know the language to some extent. The
GUI framework is tailored to Alex's work which is administrative
programs for big corporations which differs from what we do here. Most
of the stuff you might want to change is made in PicoLisp though so it won't
be very hard to change, no need to touch C. As far as the documentation
goes; I will try and remedy the situation somewhat in the near future,
the more I learn the more I can teach.

For some time now I've struggled with PicoLisp and it gets easier every day,
my \href{http://www.prodevtips.com/2007/10/15/the-c-dominion/}{C-mind} is
slowly expanding. It has been painful, and still is, but it is worth it
\href{http://www.prodevtips.com/wp-includes/images/smilies/icon_smile.gif}{http://www.prodevtips.com/wp-includes/images/smilies/icon\_smile.gif}

So having said the above I could go directly to fast-explaining what
I've done so far which is ye old registration form. Just like I do with
my PHP stuff. That would, understandably, be totally useless since we're
talking about a language with an extremely small adoption, even counting
all the people who are fluent in other Lisps.

This will instead be a new series for absolute PicoLisp beginners,
just like I was. I will contrast PicoLisp against PHP in the examples
to make it easier for c-people to understand. At the end of this
series will be the explanation of how the registration form works,
hopefully by then it will be easily understood.

\emph{Disclaimer}: This series will only be about PicoLisp, the
content in the tutorials might or might not be applicable to other
Lisps.


