% 18aug12
% joebo

\documentclass[10pt,a4paper]{article}
\usepackage{graphicx}

\textwidth 1.4\textwidth
\textheight 1.125\textheight
\oddsidemargin 0em
\evensidemargin 0em
\headsep 0em
\parindent 0em
\parskip 6pt

\title{Summary}
\author{joebo\\joebogner@gmail.com}
\date{2012-02-23}

\begin{document}
\maketitle

\section*{Summary}
PicoLisp can be compiled for android and a simple web server can be created.

To make it easier to work on the device, I strongly suggest \underline{Terminal IDE}\footnote{http://www.spartacusrex.com/terminalide.htm}

Telnetting into the device is optional - much easier than using a keyboard I found.

\begin{verbatim}
ISSUES:
1. Initially the externally defined functions would not work, e.g. ht:Pack.
   I was able to work around it for ht:Pack by compiling in ht.c and then manually defining it.
   I found other issues though with other externals and stopped due to time
2. Because of this, I can't run (html) or (http) by default
\end{verbatim}


\begin{verbatim}
Bootstrapped Steps:
1. Install Terminal IDE from the android market
2. Telnet into your device

For this, I've had the most success with running utelneted manually after starting terminal IDE)
  2a. Launch Terminal IDE
  2b. Run the following:

  utelnetd -p 2300 -l $HOME/system/bin/bash

  2c. Terminal IDE outputs your IP address on the shell
      terminal++@192.168.2.5:~/

3. Once connected through telnet, fetch arm-picoLisp.tar.gz from my dropbox account

   wget http://107.20.234.233/u/20783971/arm-picoLisp.tar.gz

   *Note: IP Address is used because Terminal IDE can't resolve host names without a rooted device and /resolve.conf

4. mv arm-picoLisp.tar.gz to ~/bin

5. tar -zxvf arm-picoLisp.tar.gz

6. ./pil

\end{verbatim}


To fire up a web server:

\begin{verbatim}
./pil lib/http.l

(redef http (S)
  (in S (line))
  (out S (prinl "Hello World!")))

(server 8001)

\end{verbatim}


Alternate server:

As you can see, we need to define ht:Pack. I could have added it to a file but wanted it here to be transparent

\begin{verbatim}
 ./pil @lib/http.l @lib/xhtml.l -'(de ht:Pack (U) (pack U))' -'server 8001 "test.l"' -wait
:\end{verbatim}


test.l
\begin{verbatim}
  (prinl "hello!")
\end{verbatim}



\begin{verbatim}
FULL STEPS:

1. On ubuntu, install gcc-4.4-arm-linux-gnueabi

   sudo apt-get install gcc-4.4-arm-linux-gnueabi

2. Replace Makefile

abbreviated form here. You can see the full version in the tar.gz above]
Main changes:
 a. replace gcc with $(CC)
 b. make it statically link
 c. remove flags that aren't compatible with  arm-linux-gnueabi-gcc-4.4]


3. make

\end{verbatim}


Makefile
\begin{verbatim}

bin = ../bin
lib = ../lib

picoFiles = main.c gc.c apply.c flow.c sym.c subr.c big.c io.c net.c tab.c ht.c

CC := arm-linux-gnueabi-gcc-4.4
CFLAGS := -c -O2 -pipe \
	-falign-functions -fomit-frame-pointer -fno-strict-aliasing \
	-W -Wimplicit -Wreturn-type -Wunused -Wformat \
	-Wuninitialized -Wstrict-prototypes \
	-D_GNU_SOURCE  -D_FILE_OFFSET_BITS=64


	OS = Linux
	PICOLISP-FLAGS =
	LIB-FLAGS = -static -lc -lm -ldl
	DYNAMIC-LIB-FLAGS = -m32 -static -export-dynamic
	STRIP =


picolisp: $(bin)/picolisp

all: picolisp

.c.o:
	$(CC) $(CFLAGS) -D_OS='"$(OS)"' $*.c


$(picoFiles:.c=.o) ext.o ht.o z3d.o: pico.h
main.o: vers.h


$(bin)/picolisp: $(picoFiles:.c=.o)
	mkdir -p $(bin) $(lib)
	$(CC) -o $(bin)/picolisp$(exe) $(PICOLISP-FLAGS) $(picoFiles:.c=.o) $(LIB-FLAGS)


# Clean up
clean:
	rm -f *.o

\end{verbatim}


\section*{Open Issues}
I haven't yet figured out how to port ht:Out. This is my current progress:

\begin{verbatim}

: (de ht:Out (flg . prg) (out flg prg))

: -> ht:Out
:(html 0 "bye" NIL NIL "foo")

HTTP/1.0 200 OK
Server: PicoLisp
Date: Thu, 23 Feb 2012 15:35:04 GMT
Cache-Control: max-age=0
Cache-Control: private, no-store, no-cache
Content-Type: text/html; charset=utf-8

-> ((prinl "<!DOCTYPE html>")
(prinl "<html lang=\"" (or *Lang "en") "\">") (prinl "<head>")
(and Ttl (<tag> 'title NIL Ttl) (prinl)) (and *Host *Port
(prinl "<base href=\"" (baseHRef) "\"/>")) (when Css (if (atom Css)
("css" Css) (mapc "css" Css))) (mapc javascript *JS) (prinl "</head>")
(tag 'body Attr 2 Prg) (prinl "</html>"))
: :
\end{verbatim}


As you can see I think it should be evaulating what comes after the header.\end{document}
