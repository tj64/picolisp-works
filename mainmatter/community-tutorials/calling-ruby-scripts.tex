\title*{Calling Ruby Scripts from PicoLisp}
\author{Henrik Sarvell}
% Use \authorrunning{Short Title} for an abbreviated version of
% your contribution title if the original one is too long
\institute{\texttt{hsarvell@gmail.com}}
%
% Use the package "url.sty" to avoid
% problems with special characters
% used in your e-mail or web address
%


\maketitle

\section{Calling Ruby scripts from Pico Lisp}
\label{sec:calling-ruby-scripts}

I give up. There you have it, the confession. Too much basic stuff is
lacking from Pico, as you might know we've covered both a way of
\href{http://www.prodevtips.com/2008/07/01/regular-expressions-in-pico-lisp/}{matching} a la PHP's preg\_match and \href{http://www.prodevtips.com/2008/09/11/pico-lisp-and-json/}{JSON.} I've realized however that
I'm never gonna be finished creating a proper application doing it in
100\% Pico.

I'm throwing in the towel and will create pure Pico Lisp libs on a per
needed basis, when calling Ruby scripts is not fast enough for example.

Why not go 100\% Ruby then? Well we've found the Holy Grail here, the
Pico Lisp OODB system + GUI framework. I'm not going to give up on that
easily.

Here's an example of how to do it anyway, the Pico call:


\begin{wideverbatim}
(println 
  (pipe 
     (call 'ruby "projects/rss-reader/htmldecode.rb" "It&#8217;s a lovely world") (line)))
\end{wideverbatim}

\textbf{Update:} The above interfered with the server, preventing further
output, bummer, the below version using (\textbf{in}) works OK though:


\begin{wideverbatim}
(println
     (in (list 'ruby "projects/rss-reader/htmldecode.rb" Str) (line)))
\end{wideverbatim}

The Ruby part:


\begin{wideverbatim}
require 'rubygems'
require 'htmlentities'
coder = HTMLEntities.new
puts coder.decode(ARGV[0])
\end{wideverbatim}

The output:


\begin{wideverbatim}
("I" "t" "’" "s" " " "a" " " "l" "o" "v" "e" "l" "y" " " "w" "o" "r" "l" "d")
\end{wideverbatim}

Great, we can move on and actually get something done here.

This entry was posted on Saturday, October 4th, 2008 at 4:46 am 


% Local variables:
% mode: latex
% TeX-master: "../../editor.tex"
% End:
