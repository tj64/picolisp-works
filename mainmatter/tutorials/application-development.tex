\title{PicoLisp Application Development}
\author{Alexander Burger}
% Use \authorrunning{Short Title} for an abbreviated version of
% your contribution title if the original one is too long
\institute{\texttt{abu@software-lab.de}}
%
% Use the package "url.sty" to avoid
% problems with special characters
% used in your e-mail or web address
%


\maketitle


\begin{abstract}
  This article presents an introduction to writing browser-based
  applications in PicoLisp.

  It concentrates on the XHTML/CSS GUI-Framework (as opposed to the
  previous Java-AWT, Java-Swing and Plain-HTML frameworks), which is
  easier to use, more flexible in layout design, and does not depend
  on plug-ins, JavaScript or cookies.

\end{abstract}

\section{Introduction}
\label{sec:appl-devel-introduction}

A plain HTTP/HTML GUI has various advantages: It runs on any browser,
and can be fully driven by scripts (``@lib/scrape.l'').

To be precise: CSS \emph{can} be used to enhance the layout. And browsers
\emph{with} JavaScript will respond faster and smoother. But this framework
works just fine in browsers which do not know anything about CSS or
JavaScript. All examples were also tested using the w3m text browser.

For basic informations about the PicoLisp system please look at the
\emph{PicoLisp Reference} and the \emph{PicoLisp Tutorial}. Knowledge of HTML, and a bit of CSS and HTTP is assumed.

The examples assume that PicoLisp was started from a global installation
(see \emph{Installation}).

 
\section{Static Pages}
\label{sec:appl-devel-static-pages}


You can use PicoLisp to generate static HTML pages. This does not make
much sense in itself, because you could directly write HTML code as
well, but it forms the base for interactive applications, and allows us
to introduce the application server and other fundamental concepts.

 
\subsection{Hello World}
\label{sec:appl-devel-hello-world}


To begin with a minimal application, please enter the following two
lines into a generic source file named ``project.l'' in the PicoLisp
installation directory.


\begin{wideverbatim}
########################################################################
(html 0 "Hello" "@lib.css" NIL
   "Hello World!" )
########################################################################
\end{wideverbatim}

(We will modify and use this file in all following examples and
experiments. Whenever you find such a program snippet between hash (`\#')
lines, just copy and paste it into your ``project.l'' file, and press the
``reload'' button of your browser to view the effects)

 
\subsubsection{ Start the application server}
\label{sec:appl-devel-start-the-application-server}%
Open a second terminal window, and start a PicoLisp application server


\begin{wideverbatim}
$ pil @lib/http.l @lib/xhtml.l @lib/form.l -'server 8080 "project.l"' +
\end{wideverbatim}

No prompt appears. The server just sits, and waits for connections. You
can stop it later by hitting \texttt{Ctrl-C} in that terminal, or by executing
 \texttt{killall pil}  in some other window.

(In the following, we assume that this HTTP server is up and running)

Now open the URL `\texttt{http://localhost:8080}' with your browser. You should
see an empty page with a single line of text.


\subsubsection{ How does it work?}
\label{sec:appl-devel-how-does-it-work?}%
The above line loads the debugger (via the `+' switch), the HTTP server
code (``@lib/http.l''), the XHTML functions (``@lib/xhtml.l'') and the input
form framework (``@lib/form.l'', it will be needed later for
\emph{interactive forms}).

Then the \texttt{server} function is called with a port number and a
default URL. It will listen on that port for incoming HTTP requests in
an endless loop. Whenever a GET request arrives on port 8080, the file
``project.l'' will be \texttt{(load)}ed, causing the evaluation
(\texttt{=} execution) of all its Lisp expressions.

During that execution, all data written to the current output channel is
sent directly to to the browser. The code in ``project.l'' is responsible
to produce HTML (or anything else the browser can understand).

\subsection{URL Syntax}
\label{sec:appl-devel-url-syntax}

The PicoLisp application server uses a slightly specialized syntax when
communicating URLs to and from a client. The ``path'' part of an URL -
which remains when

\begin{itemize}
\item the preceding protocol, host and port specifications,
\item and the trailing question mark plus arguments
\end{itemize}

are stripped off - is interpreted according so some rules. The most
prominent ones are:

\begin{itemize}
\item If a path starts with an exclamation-mark (`!'), the rest (without
   the `!') is taken as the name of a Lisp function to be called. All
   arguments following the question mark are passed to that function.
\item If a path ends with ``.l'' (a dot and a lower case `L'), it is taken as
   a Lisp source file name to be \texttt{(load)}ed. This is the most common
   case, and we use it in our example ``project.l''.
\item If the extension of a file name matches an entry in the global mime
   type table \texttt{*Mimes}, the file is sent to the client with mime-type
   and max-age values taken from that table.
\item Otherwise, the file is sent to the client with a mime-type of
   ``application/octet-stream'' and a max-age of 1 second.
\end{itemize}

An application is free to extend or modify the \texttt{*Mimes} table with the
\texttt{mime} function. For example


\begin{wideverbatim}
(mime "doc" "application/msword" 60)
\end{wideverbatim}

defines a new mime type with a max-age of one minute.

Argument values in URLs, following the path and the question mark, are
encoded in such a way that Lisp data types are preserved:

\begin{itemize}
\item An internal symbol starts with a dollar sign (`\$')
\item A number starts with a plus sign (`+')
\item An external (database) symbol starts with dash (`-')
\item A list (one level only) is encoded with underscores (`\_')
\item Otherwise, it is a transient symbol (a plain string)
\end{itemize}

In that way, high-level data types can be directly passed to functions
encoded in the URL, or assigned to global variables before a file is
loaded.

 
\subsection{Security}
\label{sec:appl-devel-security}

It is, of course, a huge security hole that - directly from the URL -
any Lisp source file can be loaded, and any Lisp function can be called.
For that reason, applications must take care to declare exactly which
files and functions are to be allowed in URLs. The server checks a
global variable \texttt{*Allow}, and - when its value is non \texttt{NIL} - denies
access to anything that does not match its contents.

Normally, \texttt{*Allow} is not manipulated directly, but set with the
\texttt{allowed} and \texttt{allow} functions


\begin{wideverbatim}
(allowed ("app/")
   "!start" "@lib.css" "customer.l" "article.l" )
\end{wideverbatim}

This is usually called in the beginning of an application, and allows
access to the directory ``@img/'', to the function `start', and to the
files ``@lib.css'', ``customer.l'' and ``article.l''.

Later in the program, \texttt{*Allow} may be dynamically extended with \texttt{allow}


\begin{wideverbatim}
(allow "!foo")
(allow "newdir/" T)
\end{wideverbatim}

This adds the function `foo', and the directory ``newdir/'', to the set of
allowed items.


\subsubsection{ The ``.pw'' File}
\label{sec:appl-devel-the-pw-file}%

For a variety of security checks (most notably for using the \texttt{psh}
function, as in some later examples) it is necessary to create a file
named ``.pw'' in the PicoLisp installation directory. This file should
contain a single line of arbitrary data, to be used as a password for
identifying local resources.

The recommeded way to create this file is to call the \texttt{pw} function,
defined in ``@lib/http.l''


\begin{wideverbatim}
$ pil @lib/http.l -'pw 12' -bye
\end{wideverbatim}

Please execute this command.

\subsection{The \texttt{html} Function}
\label{sec:appl-devel-the-html}


Now back to our ``Hello World'' example. In principle, you could write
``project.l'' as a sequence of print statements


\begin{wideverbatim}
########################################################################
(prinl "HTTP/1.0 200 OK^M")
(prinl "Content-Type: text/html; charset=utf-8")
(prinl "^M")
(prinl "<html>")
(prinl "Hello World!")
(prinl "</html>")
########################################################################
\end{wideverbatim}

but using the \texttt{html} function is much more convenient.

Moreover, \texttt{html} \textbf{is} nothing more than a printing function. You can see
this easily if you connect a PicoLisp Shell (\texttt{psh}) to the server
process (you must have generated a \emph{``.pw'' file} for this), and
enter the \texttt{html} statement


\begin{wideverbatim}
$ /usr/lib/picolisp/bin/psh 8080
: (html 0 "Hello" "@lib.css" NIL "Hello World!")
HTTP/1.0 200 OK
Server: PicoLisp
Date: Fri, 29 Dec 2006 07:28:58 GMT
Cache-Control: max-age=0
Cache-Control: no-cache
Content-Type: text/html; charset=utf-8

\end{wideverbatim}

\begin{wideverbatim}

<!DOCTYPE html PUBLIC "-//W3C//DTD XHTML 1.0 Strict//EN"
"http://www.w3.org/TR/xhtml1/DTD/xhtml1-strict.dtd">
<html xmlns="http://www.w3.org/1999/xhtml" xml:lan="en" lang="en">
<head>
<title>Hello</title>
<base href="http://localhost:8080/"/>
<link rel="stylesheet" type="text/css" href="http://localhost:8080/@lib.css"/>
</head>
<body>Hello World!</body>
</html>
-> </html>
:  # (type Ctrl-D here to terminate PicoLisp)
\end{wideverbatim}

These are the arguments to \texttt{html}:

\begin{enumerate}
\item \texttt{0}: A max-age value for cache-control (in seconds, zero means
   ``no-cache''). You might pass a higher value for pages that change
   seldom, or \texttt{NIL} for no cache-control at all.
\item \texttt{''Hello''}: The page title. 
\item \texttt{"@lib.css"}: A CSS-File name. Pass \texttt{NIL} if you
  do not want to use any CSS-File, or a list of file names if you want
  to give more than one CSS-File.
\item \texttt{NIL}: A CSS style attribute specification (see the description of
   \emph{CSS Attributes} below). It will be passed to the \texttt{body}
   tag.
\end{enumerate}

After these four arguments, an arbitrary number of expressions may
follow. They form the body of the resulting page, and are evaluated
according to a special rule. This rule is slightly different from the
evaluation of normal Lisp expressions:

\begin{itemize}
\item If an argument is an atom (a number or a symbol (string)), its value
   is printed immediately.
\item Otherwise (a list), it is evaluated as a Lisp function (typically
   some form of print statement).
\end{itemize}

Therefore, our source file might as well be written as:


\begin{wideverbatim}
########################################################################
(html 0 "Hello" "@lib.css" NIL
   (prinl "Hello World!") )
########################################################################
\end{wideverbatim}

The most typical print statements will be some HTML-tags:


\begin{wideverbatim}
########################################################################
(html 0 "Hello" "@lib.css" NIL
   (<h1> NIL "Hello World!")
   (<br> "This is some text.")
   (ht:Prin "And this is a number: " (+ 1 2 3)) )
########################################################################
\end{wideverbatim}

\texttt{<h1>} and \texttt{<br>} are tag functions. \texttt{<h1>} takes a CSS attribute as its
first argument.

Note the use of \texttt{ht:Prin} instead of \texttt{prin}. \texttt{ht:Prin} should be used
for all direct printing in HTML pages, because it takes care to escape
special characters.

 
\subsection{CSS Attributes}
\label{sec:appl-devel-css-attributes}

The \texttt{html} function above, and many of the HTML
\emph{tag functions}, accept a CSS attribute specification. This may
be either an atom, a cons pair, or a list of cons pairs. We demonstrate
the effects with the \texttt{<h1>} tag function.

An atom (usually a symbol or a string) is taken as a CSS class name


\begin{wideverbatim}
: (<h1> 'foo "Title")
<h1 class="foo">Title</h1>
\end{wideverbatim}

For a cons pair, the CAR is taken as an attribute name, and the CDR as
the attribute's value


\begin{wideverbatim}
: (<h1> '(id . bar) "Title")
<h1 id="bar">Title</h1>
\end{wideverbatim}

Consequently, a list of cons pairs gives a set of attribute-value pairs


\begin{wideverbatim}
: (<h1> '((id . "abc") (lang . "de")) "Title")
<h1 id="abc" lang="de">Title</h1>
\end{wideverbatim}

 
\subsection{Tag Functions}
\label{sec:appl-devel-tag-functions}


All pre-defined XHTML tag functions can be found in ``@lib/xhtml.l''. We
recommend to look at their sources, and to experiment a bit, by
executing them at a PicoLisp prompt, or by pressing the browser's
``Reload'' button after editing the ``project.l'' file.

For a suitable PicoLisp prompt, either execute (in a separate terminal
window) the PicoLisp Shell (\texttt{psh}) command (works only if the
application server is running, and you did generate a \emph{``.pw'' file})


\begin{wideverbatim}
$ /usr/lib/picolisp/bin/psh 8080
:
\end{wideverbatim}

or start the interpreter stand-alone, with ``@lib/xhtml.l'' loaded


\begin{wideverbatim}
$ pil @lib/http.l @lib/xhtml.l +
:
\end{wideverbatim}

Note that for all these tag functions the above \emph{tag body evaluation rule} applies.

\subsubsection{ Simple Tags}
\label{sec:appl-devel-simple-tags}%
Most tag functions are simple and straightforward. Some of them just
print their arguments


\begin{wideverbatim}
: (<br> "Hello world")
Hello world<br/>

: (<em> "Hello world")
<em>Hello world</em>
\end{wideverbatim}

while most of them take a \emph{CSS attribute specification} as
their first argument (like the \texttt{<h1>} tag above)


\begin{wideverbatim}
: (<div> 'main "Hello world")
<div class="main">Hello world</div>

: (<p> NIL "Hello world")
<p>Hello world</p>

: (<p> 'info "Hello world")
<p class="info">Hello world</p>
\end{wideverbatim}

All of these functions take an arbitrary number of arguments, and may
nest to an arbitrary depth (as long as the resulting HTML is legal)


\begin{wideverbatim}
: (<div> 'main
   (<h1> NIL "Head")
   (<p> NIL
      (<br> "Line 1")
      "Line"
      (<nbsp>)
      (+ 1 1) ) )
<div class="main"><h1>Head</h1>
<p>Line 1<br/>
Line 2</p>
</div>
\end{wideverbatim}

\subsubsection{ (Un)ordered Lists}
\label{sec:appl-devel-(un)ordered-lists}%
HTML-lists, implemented by the \texttt{<ol>} and \texttt{<ul>} tags, let you define
hierarchical structures. You might want to paste the following code into
your copy of ``project.l'':


\begin{wideverbatim}
########################################################################
(html 0 "Unordered List" "@lib.css" NIL
   (<ul> NIL
      (<li> NIL "Item 1")
      (<li> NIL
         "Sublist 1"
         (<ul> NIL
            (<li> NIL "Item 1-1")
            (<li> NIL "Item 1-2") ) )
      (<li> NIL "Item 2")
      (<li> NIL
         "Sublist 2"
         (<ul> NIL
            (<li> NIL "Item 2-1")
            (<li> NIL "Item 2-2") ) )
      (<li> NIL "Item 3") ) )
########################################################################
\end{wideverbatim}

Here, too, you can put arbitrary code into each node of that tree,
including other tag functions.


\subsubsection{ Tables}
\label{sec:appl-devel-tables}%

Like the hierarchical structures with the list functions, you can
generate two-dimensional tables with the \texttt{<table>} and \texttt{<row>}
functions.

The following example prints a table of numbers and their squares:


\begin{wideverbatim}
########################################################################
(html 0 "Table" "@lib.css" NIL
   (<table> NIL NIL NIL
      (for N 10                                    # A table with 10 rows
         (<row> NIL N (prin (* N N))) ) ) )     # and 2 columns
########################################################################
\end{wideverbatim}

The first argument to \texttt{<table>} is the usual CSS attribute, the second
an optional title (``caption''), and the third an optional list specifying
the column headers. In that list, you may supply a list for a each
column, with a CSS attribute in its CAR, and a tag body in its CDR for
the contents of the column header.

The body of \texttt{<table>} contains calls to the \texttt{<row>} function. This
function is special in that each expression in its body will go to a
separate column of the table. If both for the column header and the row
function an CSS attribute is given, they will be combined by a space and
passed to the HTML \texttt{<td>} tag. This permits distinct CSS specifications
for each column and row.

As an extension of the above table example, let's pass some attributes
for the table itself (not recommended - better define such styles in a
CSS file and then just pass the class name to \texttt{<table>}), right-align
both columns, and print each row in an alternating red and blue color


\begin{wideverbatim}
########################################################################
(html 0 "Table" "@lib.css" NIL
   (<table>
      '((width . "200px") (style . "border: dotted 1px;"))    # table style
      "Square Numbers"                                        # caption
      '((align "Number") (align "Square"))                    # 2 headers
      (for N 10                                                  # 10 rows
         (<row> (xchg '(red) '(blue))                         # red or blue
            N                                                 # 2 columns
            (prin (* N N) ) ) ) ) )
########################################################################
\end{wideverbatim}

If you wish to concatenate two or more cells in a table, so that a
single cell spans several columns, you can pass the symbol  \texttt{-}  for the
additional cell data to \texttt{<row>}. This will cause the data given to the
left of the  \texttt{-}  symbols to expand to the right.

You can also directly specify table structures with the simple \texttt{<th>},
\texttt{<tr>} and \texttt{<td>} tag functions.

If you just need a two-dimensional arrangement of components, the even
simpler \texttt{<grid>} function might be convenient:


\begin{wideverbatim}
########################################################################
(html 0 "Grid" "@lib.css" NIL
   (<grid> 3
      "A" "B" "C"
      123 456 789 ) )
########################################################################
\end{wideverbatim}

It just takes a specification for the number of columns (here: 3) as its
first argument, and then a single expression for each cell. Instead of a
number, you can also pass a list of CSS attributes. Then the length of
that list will determine the number of columns. You can change the
second line in the above example to


\begin{wideverbatim}
(<grid> '(NIL NIL right)
\end{wideverbatim}

Then the third column will be right aligned.

\subsubsection{ Menus and Tabs}
\label{sec:appl-devel-menus-and-tabs}%

The two most powerful tag functions are \texttt{<menu>} and \texttt{<tab>}. Used
separately or in combination, they form a navigation framework with

\begin{itemize}
\item menu items which open and close submenus
\item submenu items which switch to different pages
\item tabs which switch to different subpages
\end{itemize}

The following example is not very useful, because the URLs of all items
link to the same ``project.l'' page, but it should suffice to demonstrate
the functionality:


\begin{wideverbatim}
########################################################################
(html 0 "Menu+Tab" "@lib.css" NIL
   (<div> '(id . menu)
      (<menu>
         ("Item" "project.l")                      # Top level item
         (NIL (<hr>))                              # Plain HTML
         (T "Submenu 1"                            # Submenu
            ("Subitem 1.1" "project.l")
            (T "Submenu 1.2"
               ("Subitem 1.2.1" "project.l")
               ("Subitem 1.2.2" "project.l")
               ("Subitem 1.2.3" "project.l") )
            ("Subitem 1.3" "project.l") )
         (T "Submenu 2"
            ("Subitem 2.1" "project.l")
            ("Subitem 2.2" "project.l") ) ) )
   (<div> '(id . main)
      (<h1> NIL "Menu+Tab")
      (<tab>
         ("Tab1"
            (<h3> NIL "This is Tab 1") )
         ("Tab2"
            (<h3> NIL "This is Tab 2") )
         ("Tab3"
            (<h3> NIL "This is Tab 3") ) ) ) )
########################################################################
\end{wideverbatim}

\texttt{<menu>} takes a sequence of menu items. Each menu item is a list, with
its CAR either

\begin{itemize}
\item \texttt{NIL}: The entry is not an active menu item, and the rest of the list
   may consist of arbitrary code (usually HTML tags).
\item \texttt{T}: The second element is taken as a submenu name, and a click on
   that name will open or close the corresponding submenu. The rest of
   the list recursively specifies the submenu items (may nest to
   arbitrary depth).
\item Otherwise: The menu item specifies a direct action (instead of
   opening a submenu), where the first list element gives the item's
   name, and the second element the corresponding URL.
\end{itemize}

\texttt{<tab>} takes a list of subpages. Each page is simply a tab name,
followed by arbitrary code (typically HTML tags).

Note that only a single menu and a single tab may be active at the same
time.


\section{Interactive Forms}
\label{sec:appl-devel-interactive-forms}


In HTML, the only possibility for user input is via \texttt{<form>} and
\texttt{<input>} elements, using the HTTP POST method to communicate with the
server.

``@lib/xhtml.l'' defines a function called \texttt{<post>}, and a collection of
input tag functions, which allow direct programming of HTML forms. We
will supply only one simple example:


\begin{wideverbatim}
########################################################################
(html 0 "Simple Form" "@lib.css" NIL
   (<post> NIL "project.l"
      (<field> 10 '*Text)
      (<submit> "Save") ) )
########################################################################
\end{wideverbatim}

This associates a text input field with a global variable \texttt{*Text}. The
field displays the current value of \texttt{*Text}, and pressing the submit
button causes a reload of ``project.l'' with \texttt{*Text} set to any string
entered by the user.

An application program could then use that variable to do something
useful, for example store its value in a database.

The problem with such a straightforward use of forms is that

\begin{enumerate}
\item they require the application programmer to take care of maintaining
   lots of global variables. Each input field on the page needs an
   associated variable for the round trip between server and client.
\item they do not preserve an application's internal state. Each POST
   request spawns an individual process on the server, which sets the
   global variables to their new values, generates the HTML page, and
   terminates thereafter. The application state has to be passed along
   explicitly, e.g. using \texttt{<hidden>} tags.
\item they are not very interactive. There is typically only a single
   submit button. The user fills out a possibly large number of input
   fields, but changes will take effect only when the submit button is
   pressed.
\end{enumerate}

Though we wrote a few applications in that style, we recommend the GUI
framework provided by ``@lib/form.l''. It does not need any variables for
the client/server communication, but implements a class hierarchy of GUI
components for the abstraction of application logic, button actions and
data linkage.

 
\subsection{Sessions}
\label{sec:appl-devel-sessions}


First of all, we need to establish a persistent environment on the
server, to handle each individual session (for each connected client).

Technically, this is just a child process of the server we started
\emph{above}, which does not terminate immediately after it sent
its page to the browser. It is achieved by calling the \texttt{app} function
somewhere in the application's startup code.


\begin{wideverbatim}
########################################################################
(app)  # Start a session

(html 0 "Simple Session" "@lib.css" NIL
   (<post> NIL "project.l"
      (<field> 10 '*Text)
      (<submit> "Save") ) )
########################################################################
\end{wideverbatim}

Nothing else changed from the previous example. However, when you
connect your browser and then look at the terminal window where you
started the application server, you'll notice a colon, the PicoLisp
prompt


\begin{wideverbatim}
$ pil @lib/http.l @lib/xhtml.l @lib/form.l -'server 8080 "project.l"' +
:
\end{wideverbatim}

Tools like the Unix \texttt{ps} utility will tell you that now two \texttt{picolisp}
processes are running, the first being the parent of the second.

If you enter some text, say ``abcdef'', into the text field in the browser
window, press the submit button, and inspect the Lisp \texttt{*Text} variable,


\begin{wideverbatim}
: *Text
-> "abcdef"
\end{wideverbatim}

you see that we now have a dedicated PicoLisp process, ``connected'' to
the client.

You can terminate this process (like any interactive PicoLisp) by
hitting \texttt{Ctrl-D} on an empty line. Otherwise, it will terminate by
itself if no other browser requests arrive within a default timeout
period of 5 minutes.

To start a (non-debug) production version, the server is commonly
started without the `+' flag, and with \texttt{-wait}


\begin{wideverbatim}
$ pil @lib/http.l @lib/xhtml.l @lib/form.l -'server 8080 "project.l"' -wait
\end{wideverbatim}

In that way, no command line prompt appears when a client connects.

 
\subsection{Action Forms}
\label{sec:appl-devel-action-forms}


Now that we have a persistent session for each client, we can set up an
active GUI framework.

This is done by wrapping the call to the \texttt{html} function with \texttt{action}.
Inside the body of \texttt{html} can be - in addition to all other kinds of tag
functions - one or more calls to \texttt{form}


\begin{wideverbatim}
########################################################################
(app)                                              # Start session

(action                                            # Action handler
   (html 0 "Form" "@lib.css" NIL                   # HTTP/HTML protocol
      (form NIL                                    # Form
         (gui 'a '(+TextField) 10)                 # Text Field
         (gui '(+Button) "Print"                   # Button
            '(msg (val> (: home a))) ) ) ) )
########################################################################
\end{wideverbatim}

Note that there is no longer a global variable like \texttt{*Text} to hold the
contents of the input field. Instead, we gave a local, symbolic name
 \texttt{a}  to a \texttt{+TextField} component


\begin{wideverbatim}
(gui 'a '(+TextField) 10)                 # Text Field
\end{wideverbatim}

Other components can refer to it


\begin{wideverbatim}
'(msg (val> (: home a)))
\end{wideverbatim}

\texttt{(: home)} is always the form which contains this GUI component. So
\texttt{(: home a)} evaluates to the component  \texttt{a}  in the current form. As
\texttt{msg} prints its argument to standard error, and the \texttt{val>} method
retrieves the current contents of a component, we will see on the
console the text typed into the text field when we press the button.

An \texttt{action} without embedded  \texttt{form}s - or a \texttt{form} without a surrounding
\texttt{action} - does not make much sense by itself. Inside \texttt{html} and \texttt{form},
however, calls to HTML functions (and any other Lisp functions, for that
matter) can be freely mixed.

In general, a typical page may have the form


\begin{wideverbatim}
(action                                            # Action handler
   (html ..                                        # HTTP/HTML protocol
      (<h1> ..)                                    # HTML tags
      (form NIL                                    # Form
         (<h3> ..)
         (gui ..)                                  # GUI component(s)
         (gui ..)
         .. )
      (<h2> ..)
      (form NIL                                    # Another form
         (<h3> ..)
         (gui ..)                                  # GUI component(s)
         .. )
      (<br> ..)
      .. ) )
\end{wideverbatim}


\subsubsection{ The \texttt{gui} Function}
\label{sec:appl-devel-the-gui}%

The most prominent function in a \texttt{form} body is \texttt{gui}. It is the
workhorse of GUI construction.

Outside of a \texttt{form} body, \texttt{gui} is undefined. Otherwise, it takes an
optional alias name, a list of classes, and additional arguments as
needed by the constructors of these classes. We saw this example before


\begin{wideverbatim}
(gui 'a '(+TextField) 10)                 # Text Field
\end{wideverbatim}

Here,  \texttt{a}  is an alias name for a component of type \texttt{(+TextField)}. The
numeric argument \texttt{10} is passed to the text field, specifying its width.
See the chapter on \emph{GUI Classes} for more examples.

During a GET request, \texttt{gui} is basically a front-end to \texttt{new}. It builds
a component, stores it in the internal structures of the current form,
and initializes it by sending the \texttt{init>} message to the component.
Finally, it sends it the \texttt{show>} message, to produce HTML code and
transmit it to the browser.

During a POST request, \texttt{gui} does not build any new components. Instead,
the existing components are re-used. So \texttt{gui} does not have much more to
do than sending the \texttt{show>} message to a component.

\subsubsection{ Control Flow}
\label{sec:appl-devel-control-flow}%

HTTP has only two methods to change a browser window: GET and POST. We
employ these two methods in a certain defined, specialized way:

\begin{itemize}
\item GET means, a \textbf{new page} is being constructed. It is used when a page
   is visited for the first time, usually by entering an URL into the
   browser's address field, or by clicking on a link (which is often a
   \emph{submenu item or tab}).
\item POST is always directed to the \textbf{same page}. It is triggered by a
   button press, updates the corresponding form's data structures, and
   executes that button's action code.
\end{itemize}

A button's action code can do almost anything: Read and modify the
contents of input fields, communicate with the database, display alerts
and dialogs, or even fake the POST request to a GET, with the effect of
showing a completely different document (See \emph{Switching URLs}).

GET builds up all GUI components on the server. These components are
objects which encapsulate state and behavior of the HTML page in the
browser. Whenever a button is pressed, the page is reloaded via a POST
request. Then - before any output is sent to the browser - the \texttt{action}
function takes control. It performs error checks on all components,
processes possible user input on the HTML page, and stores the values in
correct format (text, number, date, object etc.) in each component.

The state of a form is preserved over time. When the user returns to a
previous page with the browser's BACK button, that state is reactivated,
and may be POSTed again.

The following silly example displays two text fields. If you enter some
text into the ``Source'' field, you can copy it in upper or lower case to
the ``Destination'' field by pressing one of the buttons


\begin{wideverbatim}
########################################################################
(app)

(action
   (html 0 "Case Conversion" "@lib.css" NIL
      (form NIL
         (<grid> 2
            "Source" (gui 'src '(+TextField) 30)
            "Destination" (gui 'dst '(+Lock +TextField) 30) )
         (gui '(+JS +Button) "Upper Case"
            '(set> (: home dst)
               (uppc (val> (: home src))) ) )
         (gui '(+JS +Button) "Lower Case"
            '(set> (: home dst)
               (lowc (val> (: home src))) ) ) ) ) )
########################################################################
\end{wideverbatim}

The \texttt{+Lock} prefix class in the ``Destination'' field makes that field
read-only. The only way to get some text into that field is by using one
of the buttons.

\subsubsection{ Switching URLs}
\label{sec:appl-devel-switching-urls}%

Because an action code runs before \texttt{html} has a chance to output an HTTP
header, it can abort the current page and present something different to
the user. This might, of course, be another HTML page, but would not be
very interesting as a normal link would suffice. Instead, it can cause
the download of dynamically generated data.

The next example shows a text area and two buttons. Any text entered
into the text area is exported either as a text file via the first
button, or a PDF document via the second button


\begin{wideverbatim}
########################################################################
(load "@lib/ps.l")

(app)

(action
   (html 0 "Export" "@lib.css" NIL
      (form NIL
         (gui '(+TextField) 30 8)
         (gui '(+Button) "Text"
            '(let Txt (tmp "export.txt")
               (out Txt (prinl (val> (: home gui 1))))
               (url Txt) ) )
         (gui '(+Button) "PDF"
            '(psOut NIL "foo"
               (a4)
               (indent 40 40)
               (down 60)
               (hline 3)
               (font (14 . "Times-Roman")
                  (ps (val> (: home gui 1))) )
               (hline 3)
               (page) ) ) ) ) )
########################################################################
\end{wideverbatim}

(a text area is built when you supply two numeric arguments (columns and
rows) to a \texttt{+TextField} class)

The action code of the first button creates a temporary file (i.e. a
file named ``export.txt'' in the current process's temporary space),
prints the value of the text area (this time we did not bother to give
it a name, we simply refer to it as the form's first gui list element)
into that file, and then calls the \texttt{url} function with the file name.

The second button uses the PostScript library ``@lib/ps.l'' to create a
temporary file ``foo.pdf''. Here, the temporary file creation and the call
to the \texttt{url} function is hidden in the internal mechanisms of \texttt{psOut}.
The effect is that the browser receives a PDF document and displays it.


\subsubsection{ Alerts and Dialogs}
\label{sec:appl-devel-alerts-and-dialogs}%

Alerts and dialogs are not really what they used to be ;-)

They do not ``pop up''. In this framework, they are just a kind of
simple-to-use, pre-fabricated form. They can be invoked by a button's
action code, and appear always on the current page, immediately
preceding the form which created them.

Let's look at an example which uses two alerts and a dialog. In the
beginning, it displays a simple form, with a locked text field, and two
buttons


\begin{wideverbatim}
########################################################################
(app)

(action
   (html 0 "Alerts and Dialogs" "@lib.css" NIL
      (form NIL
         (gui '(+Init +Lock +TextField) "Initial Text" 20 "My Text")
         (gui '(+Button) "Alert"
            '(alert NIL "This is an alert " (okButton)) )
         (gui '(+Button) "Dialog"
            '(dialog NIL
               (<br> "This is a dialog.")
               (<br>
                  "You can change the text here "
                  (gui '(+Init +TextField) (val> (: top 1 gui 1)) 20) )
               (<br> "and then re-submit it to the form.")
               (gui '(+Button) "Re-Submit"
                  '(alert NIL "Are you sure? "
                     (yesButton
                        '(set> (: home top 2 gui 1)
                           (val> (: home top 1 gui 1)) ) )
                     (noButton) ) )
               (cancelButton) ) ) ) ) )
########################################################################
\end{wideverbatim}

The \texttt{+Init} prefix class initializes the ``My Text'' field with the string
``Initial Text''. As the field is locked, you cannot modify this value
directly.

The first button brings up an alert saying ``This is an alert.''. You can
dispose it by pressing ``OK''.

The second button brings up a dialog with an editable text field,
containing a copy of the value from the form's locked text field. You
can modify this value, and send it back to the form, if you press
``Re-Submit'' and answer ``Yes'' to the ``Are you sure?'' alert.


\subsubsection{ A Calculator Example}
\label{sec:appl-devel-a-calculator-example}%

Now let's forget our ``project.l'' test file for a moment, and move on to
a more substantial and practical, stand-alone, example. Using what we
have learned so far, we want to build a simple bignum calculator.
(``bignum'' because PicoLisp can do \emph{only} bignums)

It uses a single form, a single numeric input field, and lots of
buttons. It can be found in the PicoLisp distribution (e.g. under
``/usr/share/picolisp/'') in ``misc/calc.l'', together with a directly
executable wrapper script ``misc/calc''.

To use it, change to the PicoLisp installation directory, and start it
as


\begin{wideverbatim}
$ misc/calc
\end{wideverbatim}

or call it with an absolute path, e.g.


\begin{wideverbatim}
$ /usr/share/picolisp/misc/calc
\end{wideverbatim}

If you like to get a PicoLisp prompt for inspection, start it instead as


\begin{wideverbatim}
$ pil misc/calc.l -main -go +
\end{wideverbatim}

Then - as before - point your browser to `\texttt{http://localhost:8080}'.

The code for the calculator logic and the GUI is rather straightforward.
The entry point is the single function \texttt{calculator}. It is called
directly (as described in \emph{URL Syntax}) as the server's
default URL, and implicitly in all POST requests. No further file access
is needed once the calculator is running.

Note that for a production application, we inserted an allow-statement
(as recommended by the \emph{Security} chapter)


\begin{wideverbatim}
(allowed NIL "!calculator" "@lib.css")
\end{wideverbatim}

at the beginning of ``misc/calc.l''. This will restrict external access to
that single function.

The calculator uses three global variables, \texttt{*Init}, \texttt{*Accu} and
\texttt{*Stack}. \texttt{*Init} is a boolean flag set by the operator buttons to
indicate that the next digit should initialize the accumulator to zero.
\texttt{*Accu} is the accumulator. It is always displayed in the numeric input
field, accepts user input, and it holds the results of calculations.
\texttt{*Stack} is a push-down stack, holding postponed calculations
(operators, priorities and intermediate results) with lower-priority
operators, while calculations with higher-priority operators are
performed.

The function \texttt{digit} is called by the digit buttons, and adds another
digit to the accumulator.

The function \texttt{calc} does an actual calculation step. It pops the stack,
checks for division by zero, and displays an error alert if necessary.

\texttt{operand} processes an operand button, accepting a function and a
priority as arguments. It compares the priority with that in the
top-of-stack element, and delays the calculation if it is less.

\texttt{finish} is used to calculate the final result.

The \texttt{calculator} function has one numeric input field, with a width of
60 characters


\begin{wideverbatim}
(gui '(+Var +NumField) '*Accu 60)
\end{wideverbatim}

The \texttt{+Var} prefix class associates this field with the global variable
\texttt{*Accu}. All changes to the field will show up in that variable, and
modification of that variable's value will appear in the field.

The square root operator button has an \texttt{+Able} prefix class


\begin{wideverbatim}
(gui '(+Able +JS +Button) '(ge0 *Accu) (char 8730)
   '(setq *Accu (sqrt *Accu)) )
\end{wideverbatim}

with an argument expression which checks that the current value in the
accumulator is positive, and disables the button if otherwise.

The rest of the form is just an array (grid) of buttons, encapsulating
all functionality of the calculator. The user can enter numbers into the
input field, either by using the digit buttons, or by directly typing
them in, and perform calculations with the operator buttons. Supported
operations are addition, subtraction, multiplication, division, sign
inversion, square root and power (all in bignum integer arithmetic). The
 \texttt{C}  button just clears the accumulator, while the  \texttt{A}  button also
clears all pending calculations.

All that in 53 lines of code!

\subsection{Charts}
\label{sec:appl-devel-charts}

Charts are virtual components, maintaining the internal representation
of two-dimensional data.

Typically, these data are nested lists, database selections, or some
kind of dynamically generated tabular information. Charts make it
possible to view them in rows and columns (usually in HTML
\emph{tables}), scroll up and down, and associate them with their
corresponding visible GUI components.

In fact, the logic to handle charts makes up a substantial part of the
whole framework, with large impact on all internal mechanisms. Each GUI
component must know whether it is part of a chart or not, to be able to
handle its contents properly during updates and user interactions.

Let's assume we want to collect textual and numerical data. We might
create a table


\begin{wideverbatim}
########################################################################
(app)

(action
   (html 0 "Table" "@lib.css" NIL
      (form NIL
         (<table> NIL NIL '((NIL "Text") (NIL "Number"))
            (do 4
               (<row> NIL
                  (gui '(+TextField) 20)
                  (gui '(+NumField) 10) ) ) )
         (<submit> "Save") ) ) )
########################################################################
\end{wideverbatim}

with two columns ``Text'' and ``Number'', and four rows, each containing a
\texttt{+TextField} and a \texttt{+NumField}.

You can enter text into the first column, and numbers into the second.
Pressing the ``Save'' button stores these values in the components on the
server (or produces an error message if a string in the second column is
not a legal number).

There are two problems with this solution:

\begin{enumerate}
\item Though you can get at the user input for the individual fields, e.g.


\begin{wideverbatim}
: (val> (get *Top 'gui 2))  # Value in the first row, second column
-> 123
\end{wideverbatim}

   there is no direct way to get the whole data structure as a single
   list. Instead, you have to traverse all GUI components and collect
   the data.
\item The user cannot input more than four rows of data, because there is
   no easy way to scroll down and make space for more.
\end{enumerate}

A chart can handle these things:


\begin{wideverbatim}
########################################################################
(app)

(action
   (html 0 "Chart" "@lib.css" NIL
      (form NIL
         (gui '(+Chart) 2)                         # Inserted a +Chart
         (<table> NIL NIL '((NIL "Text") (NIL "Number"))
            (do 4
               (<row> NIL
                  (gui 1 '(+TextField) 20)         # Inserted '1'
                  (gui 2 '(+NumField) 10) ) ) )    # Inserted '2'
         (<submit> "Save") ) ) )
########################################################################
\end{wideverbatim}

Note that we inserted a \texttt{+Chart} component before the GUI components
which should be managed by the chart. The argument `2' tells the chart
that it has to expect two columns.

Each component got an index number (here `1' and `2') as the first
argument to \texttt{gui}, indicating the column into which this component
should go within the chart.

Now - if you entered ``a'', ``b'' and ``c'' into the first, and 1, 2, and 3
into the second column - we can retrieve the chart's complete contents
by sending it the \texttt{val>} message


\begin{wideverbatim}
: (val> (get *Top 'chart 1))  # Retrieve the value of the first chart
-> (("a" 1) ("b" 2) ("c" 3))
\end{wideverbatim}

BTW, a more convenient function is \texttt{chart}


\begin{wideverbatim}
: (val> (chart))  # Retrieve the value of the current chart
-> (("a" 1) ("b" 2) ("c" 3))
\end{wideverbatim}

\texttt{chart} can be used instead of the above construct when we want to
access the ``current'' chart, i.e. the chart most recently processed in
the current form.


\subsubsection{ Scrolling}
\label{sec:appl-devel-scrolling}%

To enable scrolling, let's also insert two buttons. We use the
pre-defined classes \texttt{+UpButton} and \texttt{+DnButton}


\begin{wideverbatim}
########################################################################
(app)

(action
   (html 0 "Scrollable Chart" "@lib.css" NIL
      (form NIL
         (gui '(+Chart) 2)
         (<table> NIL NIL '((NIL "Text") (NIL "Number"))
            (do 4
               (<row> NIL
                  (gui 1 '(+TextField) 20)
                  (gui 2 '(+NumField) 10) ) ) )
         (gui '(+UpButton) 1)                   # Inserted two buttons
         (gui '(+DnButton) 1)
         (----)
         (<submit> "Save") ) ) )
########################################################################
\end{wideverbatim}

to scroll down and up a single (argument `1') line at a time.

Now it is possible to enter a few rows of data, scroll down, and
continue. It is not necessary (except in the beginning, when the scroll
buttons are still disabled) to press the ``Save'' button, because \textbf{any}
button in the form will send changes to the server's internal structures
before any action is performed.

\subsubsection{ Put and Get Functions}
\label{sec:appl-devel-put-and-get-functions}%

As we said, a chart is a virtual component to edit two-dimensional data.
Therefore, a chart's native data format is a list of lists: Each sublist
represents a single row of data, and each element of a row corresponds
to a single GUI component.

In the example above, we saw a row like


\begin{wideverbatim}
("a" 1)
\end{wideverbatim}

being mapped to


\begin{wideverbatim}
(gui 1 '(+TextField) 20)
(gui 2 '(+NumField) 10)
\end{wideverbatim}

Quite often, however, such a one-to-one relationship is not desired. The
internal data structures may have to be presented in a different form to
the user, and user input may need conversion to an internal
representation.

For that, a chart accepts - in addition to the ``number of columns''
argument - two optional function arguments. The first function is
invoked to `put' the internal representation into the GUI components,
and the second to `get' data from the GUI into the internal
representation.

A typical example is a chart displaying customers in a database. While
the internal representation is a (one-dimensional) list of customer
objects, `put' expands each object to a list with, say, the customer's
first and second name, telephone number, address and so on. When the
user enters a customer's name, `get' locates the matching object in the
database and stores it in the internal representation. In the following,
`put' will in turn expand it to the GUI.

For now, let's stick with a simpler example: A chart that holds just a
list of numbers, but expands in the GUI to show also a textual form of
each number (in German).


\begin{wideverbatim}
########################################################################
(app)

(load "@lib/zahlwort.l")

(action
   (html 0 "Numerals" "@lib.css" NIL
      (form NIL
         (gui '(+Init +Chart) (1 5 7) 2
            '((N) (list N (zahlwort N)))
            car )
         (<table> NIL NIL '((NIL "Numeral") (NIL "German"))
            (do 4
               (<row> NIL
                  (gui 1 '(+NumField) 9)
                  (gui 2 '(+Lock +TextField) 90) ) ) )
         (gui '(+UpButton) 1)
         (gui '(+DnButton) 1)
         (----)
         (<submit> "Save") ) ) )
########################################################################
\end{wideverbatim}

``@lib/zahlwort.l'' defines the utility function \texttt{zahlwort}, which is
required later by the `put' function. \texttt{zahlwort} accepts a number and
returns its wording in German.

Now look at the code


\begin{wideverbatim}
(gui '(+Init +Chart) (1 5 7) 2
   '((N) (list N (zahlwort N)))
   car )
\end{wideverbatim}

We prefix the \texttt{+Chart} class with \texttt{+Init}, and pass it a list of numbers
\texttt{(1 5 7)} for the initial value of the chart. Then, following the `2'
(the chart has two columns), we pass a `put' function


\begin{wideverbatim}
'((N) (list N (zahlwort N)))
\end{wideverbatim}

which takes a number and returns a list of that number and its wording,
and a `get' function


\begin{wideverbatim}
car )
\end{wideverbatim}

which in turn accepts such a list and returns a number, which happens to
be the list's first element.

You can see from this example that `get' is the inverse function of
`put'. `get' can be omitted, however, if the chart is read-only
(contains no (or only locked) input fields).

The field in the second column


\begin{wideverbatim}
(gui 2 '(+Lock +TextField) 90) ) ) )
\end{wideverbatim}

is locked, because it displays the text generated by `put', and is not
supposed to accept any user input.

When you start up this form in your browser, you'll see three pre-filled
lines with ``1/eins'', ``5/f\"unf'' and ``7/sieben'', according to the \texttt{+Init}
argument \texttt{(1 5 7)}. Typing a number somewhere into the first column, and
pressing ENTER or one of the buttons, will show a suitable text in the
second column.


\section{GUI Classes}
\label{sec:appl-devel-gui-classes}

In previous chapters we saw examples of GUI classes like \texttt{+TextField},
\texttt{+NumField} or \texttt{+Button}, often in combination with prefix classes like
\texttt{+Lock}, \texttt{+Init} or \texttt{+Able}. Now we take a broader look at the whole
hierarchy, and try more examples.

The abstract class \texttt{+gui} is the base of all GUI classes. A live view of
the class hierarchy can be obtained with the \texttt{dep} (``dependencies'')
function:


\begin{wideverbatim}
: (dep '+gui)
+gui
   +JsField
   +Button
      +UpButton
      +PickButton
         +DstButton
      +ClrButton
      +ChoButton
         +Choice
      +GoButton
      +BubbleButton
      +DelRowButton
      +ShowButton
      +DnButton
   +Img
   +field
      +Checkbox
      +TextField
         +FileField
         +ClassField
         +numField
            +NumField
            +FixField
         +BlobField
         +DateField
         +SymField
         +UpField
         +MailField
         +SexField
         +AtomField
         +PwField
         +ListTextField
         +LinesField
         +TelField
         +TimeField
         +HttpField
      +Radio
-> +gui
\end{wideverbatim}

We see, for example, that \texttt{+DnButton} is a subclass of \texttt{+Button}, which
in turn is a subclass of \texttt{+gui}. Inspecting \texttt{+DnButton} directly


\begin{wideverbatim}
: (dep '+DnButton)
   +Tiny
   +Rid
   +JS
   +Able
      +gui
   +Button
+DnButton
-> +DnButton
\end{wideverbatim}

shows that \texttt{+DnButton} inherits from \texttt{+Tiny}, \texttt{+Rid}, \texttt{+Able} and
\texttt{+Button}. The actual definition of \texttt{+DnButton} can be found in
``@lib/form.l''


\begin{wideverbatim}
(class +DnButton +Tiny +Rid +JS +Able +Button)
...
\end{wideverbatim}

In general, ``@lib/form.l'' is the ultimate reference to the framework,
and should be freely consulted.

 
\subsection{Input Fields}
\label{sec:appl-devel-input-fields}


Input fields implement the visual display of application data, and allow
\begin{itemize}
\item when enabled - input and modification of these data.
\end{itemize}

On the HTML level, they can take the form of

\begin{itemize}
\item Normal text input fields
\item Textareas
\item Checkboxes
\item Drop-down selections
\item Password fields
\item HTML links
\item Plain HTML text
\end{itemize}

Except for checkboxes, which are implemented by the
\emph{Checkbox} class, all these HTML representations are
generated by \texttt{+TextField} and its content-specific subclasses like
\texttt{+NumField}, \texttt{+DateField} etc. Their actual appearance (as one of the
above forms) depends on their arguments:

We saw already ``normal'' text fields. They are created with a single
numeric argument. This example creates an editable field with a width of
10 characters:


\begin{wideverbatim}
(gui '(+TextField) 10)
\end{wideverbatim}

If you supply a second numeric for the line count (`4' in this case),
you'll get a text area:


\begin{wideverbatim}
(gui '(+TextField) 10 4)
\end{wideverbatim}

Supplying a list of values instead of a count yields a drop-down
selection (combo box):


\begin{wideverbatim}
(gui '(+TextField) '("Value 1" "Value 2" "Value 3"))
\end{wideverbatim}

In addition to these arguments, you can pass a string. Then the field is
created with a label:


\begin{wideverbatim}
(gui '(+TextField) 10 "Plain")
(gui '(+TextField) 10 4 "Text Area")
(gui '(+TextField) '("Value 1" "Value 2" "Value 3") "Selection")
\end{wideverbatim}

Finally, without any arguments, the field will appear as a plain HTML
text:


\begin{wideverbatim}
(gui '(+TextField))
\end{wideverbatim}

This makes mainly sense in combination with prefix classes like \texttt{+Var}
and \texttt{+Obj}, to manage the contents of these fields, and achieve special
behavior as HTML links or scrollable chart values.

\subsubsection{ Numeric Input Fields}
\label{sec:appl-devel-numeric-input-fields}%

A \texttt{+NumField} returns a number from its \texttt{val>} method, and accepts a
number for its \texttt{set>} method. It issues an error message when user input
cannot be converted to a number.

Large numbers are shown with a thousands-separator, as determined by the
current locale.


\begin{wideverbatim}
########################################################################
(app)

(action
   (html 0 "+NumField" "@lib.css" NIL
      (form NIL
         (gui '(+NumField) 10)
         (gui '(+JS +Button) "Print value"
            '(msg (val> (: home gui 1))) )
         (gui '(+JS +Button) "Set to 123"
            '(set> (: home gui 1) 123) ) ) ) )
########################################################################
\end{wideverbatim}

A \texttt{+FixField} needs an additional scale factor argument, and
accepts/returns scaled fixpoint numbers.

The decimal separator is determined by the current locale.


\begin{wideverbatim}
########################################################################
(app)

(action
   (html 0 "+FixField" "@lib.css" NIL
      (form NIL
         (gui '(+FixField) 3 10)
         (gui '(+JS +Button) "Print value"
            '(msg (format (val> (: home gui 1)) 3)) )
         (gui '(+JS +Button) "Set to 123.456"
            '(set> (: home gui 1) 123456) ) ) ) )
########################################################################
\end{wideverbatim}

\subsubsection{ Time \& Date}
\label{sec:appl-devel-time-date}%

A \texttt{+DateField} accepts and returns a \texttt{date} value.

\begin{wideverbatim}
########################################################################
(app)

(action
   (html 0 "+DateField" "@lib.css" NIL
      (form NIL
         (gui '(+DateField) 10)
         (gui '(+JS +Button) "Print value"
            '(msg (datStr (val> (: home gui 1)))) )
         (gui '(+JS +Button) "Set to \"today\""
            '(set> (: home gui 1) (date)) ) ) ) )
########################################################################
\end{wideverbatim}

The format displayed to - and entered by - the user depends on the
current locale (see \texttt{datStr} and \texttt{expDat}). You can change it, for
example to


\begin{wideverbatim}
: (locale "DE" "de")
-> NIL
\end{wideverbatim}

If no locale is set, the format is YYYY-MM-DD. Some pre-defined locales
use patterns like DD.MM.YYYY (DE), YYYY/MM/DD (JP), DD/MM/YYYY (UK), or
MM/DD/YYYY (US).

An error is issued when user input does not match the current locale's
date format.

Independent from the locale setting, a \texttt{+DateField} tries to expand
abbreviated input from the user. A small number is taken as that day of
the current month, larger numbers expand to day and month, or to day,
month and year:

\begin{itemize}
\item ``7'' gives the 7th of the current month
\item ``031'' or ``0301'' give the 3rd of January of the current year
\item ``311'' or ``3101'' give the 31st of January of the current year
\item ``0311'' gives the 3rd of November of the current year
\item ``01023'' or ``010203'' give the first of February in the year 2003
\item and so on
\end{itemize}

Similar is the \texttt{+TimeField}. It accepts and returns a \texttt{time} value.


\begin{wideverbatim}
########################################################################
(app)

(action
   (html 0 "+TimeField" "@lib.css" NIL
      (form NIL
         (gui '(+TimeField) 8)
         (gui '(+JS +Button) "Print value"
            '(msg (tim$ (val> (: home gui 1)))) )
         (gui '(+JS +Button) "Set to \"now\""
            '(set> (: home gui 1) (time)) ) ) ) )
########################################################################
\end{wideverbatim}

When the field width is `8', like in this example, time is displayed in
the format \texttt{HH:MM:SS}. Another possible value would be `5', causing
\texttt{+TimeField} to display its value as \texttt{HH:MM}.

An error is issued when user input cannot be converted to a time value.

The user may omit the colons. If he inputs just a small number, it
should be between `0' and `23', and will be taken as a full hour. `125'
expands to ``12:05'', `124517' to ``12:45:17'', and so on.

\subsubsection{ Telephone Numbers}
\label{sec:appl-devel-telephone-numbers}%

Telephone numbers are represented internally by the country code
(without a leading plus sign or zero) followed by the local phone number
(ideally separated by spaces) and the phone extension (ideally separated
by a hyphen). The exact format of the phone number string is not
enforced by the GUI, but further processing (e.g. database searches)
normally uses \texttt{fold} for better reproducibility.

To display a phone number, \texttt{+TelField} replaces the country code with a
single zero if it is the country code of the current locale, or prepends
it with a plus sign if it is a foreign country (see \texttt{telStr}).

For user input, a plus sign or a double zero is simply dropped, while a
single leading zero is replaced with the current locale's country code
(see \texttt{expTel}).


\begin{wideverbatim}
########################################################################
(app)
(locale "DE" "de")

(action
   (html 0 "+TelField" "@lib.css" NIL
      (form NIL
         (gui '(+TelField) 20)
         (gui '(+JS +Button) "Print value"
            '(msg (val> (: home gui 1))) )
         (gui '(+JS +Button) "Set to \"49 1234 5678-0\""
            '(set> (: home gui 1) "49 1234 5678-0") ) ) ) )
########################################################################
\end{wideverbatim}

\subsubsection{ Checkboxes}
\label{sec:appl-devel-checkboxes}%

A \texttt{+Checkbox} is straightforward. User interaction is restricted to
clicking it on and off. It accepts boolean (\texttt{NIL} or non \texttt{NIL}  values,
and returns \texttt{T} or \texttt{NIL}.


\begin{wideverbatim}
########################################################################
(app)

(action
   (html 0 "+Checkbox" "@lib.css" NIL
      (form NIL
         (gui '(+Checkbox))
         (gui '(+JS +Button) "Print value"
            '(msg (val> (: home gui 1))) )
         (gui '(+JS +Button) "On"
            '(set> (: home gui 1) T) )
         (gui '(+JS +Button) "Off"
            '(set> (: home gui 1) NIL) ) ) ) )
########################################################################
\end{wideverbatim}


\subsection{Field Prefix Classes}
\label{sec:appl-devel-field-prefix-classes}

A big part of this framework's power is owed to the combinatorial
flexibility of prefix classes for GUI- and DB-objects. They allow to
surgically override individual methods in the inheritance tree, and can
be combined in various ways to achieve any desired behavior.

Technically, there is nothing special about prefix classes. They are
just normal classes. They are called ``prefix'' because they are intended
to be written \emph{before} other classes in a class's or object's list of
superclasses.

Usually they take their own arguments for their \texttt{T} method from the list
of arguments to the \texttt{gui} function.


\subsubsection{ Initialization}
\label{sec:appl-devel-initialization}%

\texttt{+Init} overrides the \texttt{init>} method for that component. The \texttt{init>}
message is sent to a \texttt{+gui} component when the page is loaded for the
first time (during a GET request). \texttt{+Init} takes an expression for the
initial value of that field.


\begin{wideverbatim}
(gui '(+Init +TextField) "This is the initial text" 30)
\end{wideverbatim}

Other classes which automatically give a value to a field are \texttt{+Var}
(linking the field to a variable) and \texttt{+E/R} (linking the field to a
database entity/relation).

\texttt{+Cue} can be used, for example in ``mandatory'' fields, to give a hint to
the user about what he is supposed to enter. It will display the
argument value, in angular brackets, if and only if the field's value is
\texttt{NIL}, and the \texttt{val>} method will return \texttt{NIL} despite the fact that
this value is displayed.

Cause an empty field to display ``\textless Please enter some text
here\textgreater '':


\begin{wideverbatim}
(gui '(+Cue +TextField) "Please enter some text here" 30)
\end{wideverbatim}


\subsubsection{ Disabling and Enabling}
\label{sec:appl-devel-disabling-and-enabling}%
An important feature of an interactive GUI is the context-sensitive
disabling and enabling of individual components, or of a whole form.

The \texttt{+Able} prefix class takes an argument expression, and disables the
component if this expression returns \texttt{NIL}. We saw an example for its
usage already in the \emph{square root button} of the
calculator example. Or, for illustration purposes, imagine a button
which is supposed to be enabled only after Christmas


\begin{wideverbatim}
(gui '(+Able +Button)
   '(>= (cdr (date (date))) (12 24))
   "Close this year"
   '(endOfYearProcessing) )
\end{wideverbatim}

or a password field that is disabled as long as somebody is logged in


\begin{wideverbatim}
(gui '(+Able +PwField) '(not *Login) 10 "Password")
\end{wideverbatim}

A special case is the \texttt{+Lock} prefix, which permanently and
unconditionally disables a component. It takes no arguments


\begin{wideverbatim}
(gui '(+Lock +NumField) 10 "Count")
\end{wideverbatim}

(`10' and ``Count'' are for the \texttt{+NumField}), and creates a read-only
field.

The whole form can be disabled by calling \texttt{disable} with a non
\texttt{NIL} argument. This affects all components in this form.
Staying with the above example, we can make the form read-only until
Christmas


\begin{wideverbatim}
(form NIL
   (disable (> (12 24) (cdr (date (date)))))  # Disable whole form
   (gui ..)
   .. )
\end{wideverbatim}

Even in a completely disabled form, however, it is often necessary to
re-enable certain components, as they are needed for navigation,
scrolling, or other activities which don't affect the contents of the
form. This is done by prefixing these fields with \texttt{+Rid} (i.e. getting
``rid'' of the lock).


\begin{wideverbatim}
(form NIL
   (disable (> (12 24) (cdr (date (date)))))
   (gui ..)
   ..
   (gui '(+Rid +Button) ..)  # Button is enabled despite the disabled form
   .. )
\end{wideverbatim}


\subsubsection{ Formatting}
\label{sec:appl-devel-formatting}%

GUI prefix classes allow a fine-grained control of how values are stored
in - and retrieved from - components. As in predefined classes like
\texttt{+NumField} or \texttt{+DateField}, they override the \texttt{set>} and/or \texttt{val>}
methods.

\texttt{+Set} takes an argument function which is called whenever that field is
set to some value. To convert all user input to upper case


\begin{wideverbatim}
(gui '(+Set +TextField) uppc 30)
\end{wideverbatim}

\texttt{+Val} is the complement to \texttt{+Set}. It takes a function which is called
whenever the field's value is retrieved. To return the square of a
field's value


\begin{wideverbatim}
(gui '(+Val +NumField) '((N) (* N N)) 10)
\end{wideverbatim}

\texttt{+Fmt} is just a combination of \texttt{+Set} and \texttt{+Val}, and takes two
functional arguments. This example will display upper case characters,
while returning lower case characters internally


\begin{wideverbatim}
(gui '(+Fmt +TextField) uppc lowc 30)
\end{wideverbatim}

\texttt{+Map} does (like \texttt{+Fmt}) a two-way translation. It uses a list of cons
pairs for a linear lookup, where the CARs represent the displayed values
which are internally mapped to the values in the CDRs. If a value is not
found in this list during \texttt{set>} or \texttt{val>}, it is passed through
unchanged.

Normally, \texttt{+Map} is used in combination with the combo box incarnation
of text fields (see \emph{Input Fields}). This example
displays ``One'', ``Two'' and ``Three'' to the user, but returns a number 1, 2
or 3 internally


\begin{wideverbatim}
########################################################################
(app)

(action
   (html 0 "+Map" "@lib.css" NIL
      (form NIL
         (gui '(+Map +TextField)
            '(("One" . 1) ("Two" . 2) ("Three" . 3))
            '("One" "Two" "Three") )
         (gui '(+Button) "Print"
            '(msg (val> (field -1))) ) ) ) )
########################################################################
\end{wideverbatim}


\subsubsection{ Side Effects}
\label{sec:appl-devel-side-effects}%

Whenever a button is pressed in the GUI, any changes caused by \texttt{action}
in the current environment (e.g. the database or application state) need
to be reflected in the corresponding GUI fields. For that, the \texttt{upd>}
message is sent to all components. Each component then takes appropriate
measures (e.g. refresh from database objects, load values from
variables, or calculate a new value) to update its value.

While the \texttt{upd>} method is mainly used internally, it can be overridden
in existing classes via the \texttt{+Upd} prefix class. Let's print updated
values to standard error


\begin{wideverbatim}
########################################################################
(app)
(default *Number 0)

(action
   (html 0 "+Upd" "@lib.css" NIL
      (form NIL
         (gui '(+Upd +Var +NumField)
            '(prog (extra) (msg *Number))
            '*Number 8 )
         (gui '(+JS +Button) "Increment"
            '(inc '*Number) ) ) ) )
########################################################################
\end{wideverbatim}

\subsubsection{ Validation}
\label{sec:appl-devel-validation}%

To allow automatic validation of user input, the \texttt{chk>} message is sent
to all components at appropriate times. The corresponding method should
return \texttt{NIL} if the value is all right, or a string describing the error
otherwise.

Many of the built-in classes have a \texttt{chk>} method. The \texttt{+NumField} class
checks for legal numeric input, or the \texttt{+DateField} for a valid calendar
date.

An on-the-fly check can be implemented with the \texttt{+Chk} prefix class. The
following code only accepts numbers not bigger than 9: The \texttt{or}
expression first delegates the check to the main \texttt{+NumField} class, and
\begin{itemize}
\item if it does not give an error - returns an error string when the
\end{itemize}
current value is greater than 9.


\begin{wideverbatim}
########################################################################
(app)

(action
   (html 0 "+Chk" "@lib.css" NIL
      (form NIL
         (gui '(+Chk +NumField)
            '(or
               (extra)
               (and (> (val> This) 9) "Number too big") )
            12 )
         (gui '(+JS +Button) "Print"
            '(msg (val> (field -1))) ) ) ) )
########################################################################
\end{wideverbatim}

A more direct kind of validation is built-in via the \texttt{+Limit} class. It
controls the \texttt{maxlength} attribute of the generated HTML input field
component. Thus, it is impossible to type to more characters than
allowed into the field.


\begin{wideverbatim}
########################################################################
(app)

(action
   (html 0 "+Limit" "@lib.css" NIL
      (form NIL
         (gui '(+Limit +TextField) 4 8)
         (gui '(+JS +Button) "Print"
            '(msg (val> (field -1))) ) ) ) )
########################################################################
\end{wideverbatim}

 


\subsubsection{ Data Linkage}
\label{sec:appl-devel-data-linkage}%

Although \texttt{set>} and \texttt{val>} are the official methods to get a value in
and out of a GUI component, they are not very often used explicitly.
Instead, components are directly linked to internal Lisp data
structures, which are usually either variables or database objects.

The \texttt{+Var} prefix class takes a variable (described as the \texttt{var} data
type - either a symbol or a cell - in the \emph{Function Reference}). In the following example, we initialize a global variable
with the value ``abc'', and let a \texttt{+TextField} operate on it. The ``Print''
button can be used to display its current value.


\begin{wideverbatim}
########################################################################
(app)

(setq *TextVariable "abc")

(action
   (html 0 "+Var" "@lib.css" NIL
      (form NIL
         (gui '(+Var +TextField) '*TextVariable 8)
         (gui '(+JS +Button) "Print"
            '(msg *TextVariable) ) ) ) )
########################################################################
\end{wideverbatim}

\texttt{+E/R} takes an entity/relation specification. This is a cell, with a
relation in its CAR (e.g. \texttt{nm}, for an object's name), and an expression
in its CDR (typically \texttt{(: home obj)}, the object stored in the \texttt{obj}
property of the current form).

For an isolated, simple example, we create a temporary database, and
access the \texttt{nr} and \texttt{nm} properties of an object stored in a global
variable \texttt{*Obj}.


\begin{wideverbatim}
########################################################################
(when (app)                # On start of session
   (class +Tst +Entity)    # Define data model
   (rel nr (+Number))      # with a number
   (rel nm (+String))      # and a string
   (pool (tmp "db"))       # Create temporary DB
   (setq *Obj              # and a single object
      (new! '(+Tst) 'nr 1 'nm "New Object") ) )

(action
   (html 0 "+E/R" "@lib.css" NIL
      (form NIL
         (gui '(+E/R +NumField) '(nr . *Obj) 8)    # Linkage to 'nr'
         (gui '(+E/R +TextField) '(nm . *Obj) 20)  # Linkage to 'nm'
         (gui '(+JS +Button) "Show"                # Show the object
            '(out 2 (show *Obj)) ) ) ) )           # on standard error
########################################################################
\end{wideverbatim}


\subsection{Buttons}
\label{sec:appl-devel-buttons}


Buttons are, as explained in \emph{Control Flow}, the only way
(via POST requests) for an application to communicate with the server.

Basically, a \texttt{+Button} takes

\begin{itemize}
\item a label, which may be either a string or the name of an image file
\item an optional alternative label, shown when the button is disabled
\item and an executable expression.
\end{itemize}

Here is a minimal button, with just a label and an expression:


\begin{wideverbatim}
(gui '(+Button) "Label" '(doSomething))
\end{wideverbatim}

And this is a button displaying different labels, depending on the
state:


\begin{wideverbatim}
(gui '(+Button) "Enabled" "Disabled" '(doSomething))
\end{wideverbatim}

To show an image instead of plain text, the label(s) must be preceeded
by the \texttt{T} symbol:


\begin{wideverbatim}
(gui '(+Button) T "img/enabled.png" "img/disabled.png" '(doSomething))
\end{wideverbatim}

The expression will be executed during \texttt{action} handling (see
\emph{Action Forms}), when this button was pressed.

Like other components, buttons can be extended and combined with prefix
classes, and a variety of predefined classes and class combinations are
available.


\subsubsection{ Dialog Buttons}
\label{sec:appl-devel-dialog-buttons}%

Buttons are essential for the handling of \emph{alerts and dialogs}. Besides buttons for normal functions, like
\emph{scrolling} in charts or other \emph{side effects}, special buttons exist which can \emph{close} an alert or dialog in
addition to doing their principal job.

Such buttons are usually subclasses of \texttt{+Close}, and most of them can be
called easily with ready-made functions like \texttt{closeButton},
\texttt{cancelButton}, \texttt{yesButton} or \texttt{noButton}. We saw a few examples in
\emph{Alerts and Dialogs}.

 


\subsubsection{ Active JavaScript}
\label{sec:appl-devel-active-javascript}%

When a button inherits from the \texttt{+JS} class (and JavaScript is enabled
in the browser), that button will possibly show a much faster response
in its action.

The reason is that the activation of a \texttt{+JS} button will - instead of
doing a normal POST - first try to send only the contents of all GUI
components via an XMLHttpRequest to the server, and receive the updated
values in response. This avoids the flicker caused by reloading and
rendering of the whole page, is much faster, and also does not jump to
the beginning of the page if it is larger than the browser window. The
effect is especially noticeable while scrolling in charts.

Only if this fails, for example because an error message was issued, or
a dialog popped up, it will fall back, and the form will be POSTed in
the normal way.

Thus it makes no sense to use the \texttt{+JS} prefix for buttons that cause a
change of the HTML code, open a dialog, or jump to another page. In such
cases, overall performance will even be worse, because the
XMLHttpRequest is tried first (but in vain).

When JavaScript is disabled int the browser, the XMLHttpRequest will not
be tried at all. The form will be fully usable, though, with identical
functionality and behavior, just a bit slower and not so smooth.

\section{A Minimal Complete Application}
\label{sec:appl-devel-a-minimal-complete-application}

The PicoLisp release includes in the ``app/'' directory a minimal, yet
complete reference application. This application is typical, in the
sense that it implements many of the techniques described in this
document, and it can be easily modified and extended. In fact, we use it
as templates for our own production application development.

It is a kind of simplified ERP system, containing customers/suppliers,
products (items), orders, and other data. The order input form performs
live updates of customer and product selections, price, inventory and
totals calculations, and generates on-the-fly PDF documents.
Fine-grained access permissions are controlled via users, roles and
permissions. It comes localized in six languages (English, Spanish,
German, Norwegian, Russian and Japanese), with some initial data and two
sample reports.

 
\subsection{Getting Started}
\label{sec:appl-devel-getting-started}

For a global installation (see \emph{Installation}), please
create a symbolic link to the place where the program files are
installed. This is necessary because the application needs read/write
access to the current working directory (for the database and other
runtime data).

\begin{wideverbatim}
$ ln -s /usr/share/picolisp/app
\end{wideverbatim}

As ever, you may start up the application in debugging mode


\begin{wideverbatim}
$ pil app/main.l -main -go +
\end{wideverbatim}

or in (non-debug) production mode


\begin{wideverbatim}
$ pil app/main.l -main -go -wait
\end{wideverbatim}

and go to `\texttt{http://localhost:8080}' with your browser. You can login as
user ``admin'', with password ``admin''. The demo data contain several other
users, but those are more restricted in their role permissions.

Another possibility is to try the online version of this application at
\href{http://app.7fach.de}{app.7fach.de}.

 
\subsubsection{ Localization}
\label{sec:appl-devel-localization}%


Before or after you logged in, you can select another language, and
click on the ``Change'' button. This will effect all GUI components
(though not text from the database), and also the numeric, date and
telephone number formats.

\subsubsection{ Navigation}
\label{sec:appl-devel-navigation}%
The navigation menu on the left side shows two items ``Home'' and
``logout'', and three submenus ``Data'', ``Report'' and ``System''.

Both ``Home'' and ``logout'' bring you back to the initial login form. Use
``logout'' if you want to switch to another user (say, for another set of
permissions), and - more important - before you close your browser, to
release possible locks and process resources on the server.

The ``Data'' submenu gives access to application specific data entry and
maintenance: Orders, product items, customers and suppliers. The
``Report'' submenu contains two simple inventory and sales reports. And
the ``System'' submenu leads to role and user administration.

You can open and close each submenu individually. Keeping more than one
submenu open at a time lets you switch rapidly between different parts
of the application.

The currently active menu item is indicated by a highlighted list style
(no matter whether you arrived at this page directly via the menu or by
clicking on a link somewhere else).

 
\subsubsection{ Choosing Objects}
\label{sec:appl-devel-choosing-objects}%

Each item in the ``Data'' or ``System'' submenu opens a search dialog for
that class of entities. You can specify a search pattern, press the top
right ``Search'' button (or just ENTER), and scroll through the list of
results.

While the ``Role'' and ``User'' entities present simple dialogs (searching
just by name), other entities can be searched by a variety of criteria.
In those cases, a ``Reset'' button clears the contents of the whole
dialog. A new object can be created with bottom right ``New'' button.

In any case, the first column will contain either a ``@''-link (to jump to
that object) or a ``@''-button (to insert a reference to that object into
the current form).

By default, the search will list all database objects with an attribute
value greater than or equal to the search criterion. The comparison is
done arithmetically for numbers, and alphabetically (case sensitive!)
for text. This means, if you type ``Free'' in the ``City'' field of the
``Customer/Supplier'' dialog, the value of ``Freetown'' will be matched. On
the other hand, an entry of ``free'' or ``town'' will yield no hits.

Some search fields, however, show a different behavior depending on the
application:

\begin{itemize}
\item The names of persons, companies or products allow a tolerant search,
   matching either a slightly misspelled name (``Mühler'' instead of
   ``Miller'') or a substring (``Oaks'' will match ``Seven Oaks Ltd.'').
\item The search field may specify an upper instead of a lower limit,
   resulting in a search for database objects with an attribute value
   less than or equal to the search criterion. This is useful, for
   example in the ``Order'' dialog, to list orders according to their
   number or date, by starting with the newest then and going backwards.
\end{itemize}

Using the bottom left scroll buttons, you can scroll through the result
list without limit. Clicking on a link will bring up the corresponding
object. Be careful here to select the right column: Some dialogs (those
for ``Item'' and ``Order'') also provide links for related entities (e.g.
``Supplier'').

\subsubsection{ Editing}
\label{sec:appl-devel-editing}%

A database object is usually displayed in its own individual form, which
is determined by its entity class.

The basic layout should be consistent for all classes: Below the heading
(which is usually the same as the invoking menu item) is the object's
identifier (name, number, etc.), and then a row with an ``Edit'' button on
the left, and ``Delete'' button, a ``Select'' button and two navigation
links on the right side.

The form is brought up initially in read-only mode. This is necessary to
prevent more than one user from modifying an object at the same time
(and contrary to the previous PicoLisp Java frameworks, where this was
not a problem because all changes were immediately reflected in the GUIs
of other users).

So if you want to modify an object, you have to gain exclusive access by
clicking on the ``Edit'' button. The form will be enabled, and the ``Edit''
button changes to ``Done''. Should any other user already have reserved
this object, you will see a message telling his name and process ID.

An exception to this are objects that were just created with ``New''. They
will automatically be reserved for you, and the ``Edit'' button will show
up as ``Done''.

The ``Delete'' button pops up an alert, asking for confirmation. If the
object is indeed deleted, this button changes to ``Restore'' and allows to
undelete the object. Note that objects are never completely deleted from
the database as long as there are any references from other objects.
When a ``deleted'' object is shown, its identifier appears in square
brackets.

The ``Select'' button (re-)displays the search dialog for this class of
entities. The search criteria are preserved between invocations of each
dialog, so that you can conveniently browse objects in this context.

The navigation links, pointing left and right, serve a similar purpose.
They let you step sequentially through all objects of this class, in the
order of the identifier's index.

Other buttons, depending on the entity, are usually arranged at the
bottom of the form. The bottom rightmost one should always be another
``Edit'' / ``Done'' button.

As we said in the chapter on \emph{Scrolling}, any button in
the form will save changes to the underlying data model. As a special
case, however, the ``Done'' button releases the object and reverts to
``Edit''. Besides this, the edit mode will also cease as soon as another
object is displayed, be it by clicking on an object link (the pencil
icon), the top right navigation links, or a link in a search dialog.


\subsubsection{ Buttons vs. Links}
\label{sec:appl-devel-buttons-vs.-links}%

The only way to interact with a HTTP-based application server is to
click either on a HTML link, or on a submit button (see also
\emph{Control Flow}). It is essential to understand the different
effects of such a click on data entered or modified in the current form.

\begin{itemize}
\item A click on a link will leave or reload the page. Changes are
   discarded.
\item A click on a button will commit changes, and perform the associated
   action.
\end{itemize}

For that reason the layout design should clearly differentiate between
links and buttons. Image buttons are not a good idea when in other
places images are used for links. The standard button components should
be preferred; they are usually rendered by the browser in a
non-ambiguous three-dimensional look and feel.

Note that if JavaScript is enabled in the browser, changes will be
automatically committed to the server.

The enabled or disabled state of a button is an integral part of the
application logic. It must be indicated to the user with appropriate
styles.

\subsection{The Data Model}
\label{sec:appl-devel-the-data-model}

\paragraph{Source Code}
\label{sec:appl-devel-data-model-source-code}

\begin{wideverbatim}

# 21jul11abu
# (c) Software Lab. Alexander Burger

### Entity/Relations ###
#
#           nr    nm                   nr    nm               nm
#            |    |                     |    |                |
#          +-*----*-+                 +-*----*-+           +--*-----+
#          |        |             sup |        |           |        |
#    str --*  CuSu  O-----------------*  Item  *-- inv     |  Role  @-- perm
#          |        |                 |        |           |        |
#          +-*-*--O-+                 +----O---+           +----@---+
#            | |  |                        |                    | usr
#   nm  tel -+ |  |                        |                    |
#    |         |  |                        | itm                | role
#  +-*-----+   |  |   +-------+        +---*---+           +----*---+
#  |       |   |  |   |       |    ord |       |           |        |
#  |  Sal  +---+  +---*  Ord  @--------*  Pos  |      nm --*  User  *-- pw
#  |       |      cus |       | pos    |       |           |        |
#  +-*---*-+          +-*---*-+        +-*---*-+           +--------+
#    |   |              |   |            |   |
#   hi   sex           nr  dat          pr   cnt


(extend +Role)

(dm url> (Tab)
   (and (may RoleAdmin) (list "app/role.l" '*ID This)) )

\end{wideverbatim}

\begin{wideverbatim}

(extend +User)
(rel nam (+String))                    # Full Name
(rel tel (+String))                    # Phone
(rel em (+String))                     # EMail

(dm url> (Tab)
   (and (may UserAdmin) (list "app/user.l" '*ID This)) )


# Salutation
(class +Sal +Entity)
(rel nm (+Key +String))                # Salutation
(rel hi (+String))                     # Greeting
(rel sex (+Any))                       # T:male, 0:female

(dm url> (Tab)
   (and (may Customer) (list "app/sal.l" '*ID This)) )

(dm hi> (Nm)
   (or (text (: hi) Nm) ,"Dear Sir or Madam,") )


# Customer/Supplier
(class +CuSu +Entity)
(rel nr (+Need +Key +Number))          # Customer/Supplier Number
(rel sal (+Link) (+Sal))               # Salutation
(rel nm (+Sn +Idx +String))            # Name
(rel nm2 (+String))                    # Name 2
(rel str (+String))                    # Street
(rel plz (+Ref +String))               # Zip
(rel ort (+Fold +Idx +String))         # City
(rel tel (+Fold +Ref +String))         # Phone
(rel fax (+String))                    # Fax
(rel mob (+Fold +Ref +String))         # Mobile
(rel em (+String))                     # EMail
(rel txt (+Blob))                      # Memo

(dm url> (Tab)
   (and (may Customer) (list "app/cusu.l"  '*Tab Tab  '*ID This)) )

(dm check> ()
   (make
      (or (: nr) (link ,"No customer number"))
      (or (: nm) (link ,"No customer name"))
      (unless (and (: str) (: plz) (: ort))
         (link ,"Incomplete customer address") ) ) )


\end{wideverbatim}

\begin{wideverbatim}

# Item
(class +Item +Entity)
(rel nr (+Need +Key +Number))          # Item Number
(rel nm (+Fold +Idx +String))          # Item Description
(rel sup (+Ref +Link) NIL (+CuSu))     # Supplier
(rel inv (+Number))                    # Inventory
(rel pr (+Ref +Number) NIL 2)          # Price
(rel txt (+Blob))                      # Memo
(rel jpg (+Blob))                      # Picture

(dm url> (Tab)
   (and (may Item) (list "app/item.l" '*ID This)) )

(dm cnt> ()
   (-
      (or (: inv) 0)
      (sum '((This) (: cnt))
         (collect 'itm '+Pos This) ) ) )

(dm check> ()
   (make
      (or (: nr) (link ,"No item number"))
      (or (: nm) (link ,"No item description")) ) )


# Order
(class +Ord +Entity)
(rel nr (+Need +Key +Number))          # Order Number
(rel dat (+Need +Ref +Date))           # Order date
(rel cus (+Ref +Link) NIL (+CuSu))     # Customer
(rel pos (+List +Joint) ord (+Pos))    # Positions

(dm lose> ()
   (mapc 'lose> (: pos))
   (super) )

(dm url> (Tab)
   (and (may Order) (list "app/ord.l" '*ID This)) )

(dm sum> ()
   (sum 'sum> (: pos)) )

(dm check> ()
   (make
      (or (: nr) (link ,"No order number"))
      (or (: dat) (link ,"No order date"))
      (if (: cus)
         (chain (check> @))
         (link ,"No customer") )
      (if (: pos)
         (chain (mapcan 'check> @))
         (link ,"No positions") ) ) )

\end{wideverbatim}

\begin{wideverbatim}

(class +Pos +Entity)
(rel ord (+Dep +Joint)                 # Order
   (itm)
   pos (+Ord) )
(rel itm (+Ref +Link) NIL (+Item))     # Item
(rel pr (+Number) 2)                   # Price
(rel cnt (+Number))                    # Quantity

(dm sum> ()
   (* (: pr) (: cnt)) )

(dm check> ()
   (make
      (if (: itm)
         (chain (check> @))
         (link ,"Position without item") )
      (or (: pr) (link ,"Position without price"))
      (or (: cnt) (link ,"Position without quantity")) ) )


# Database sizes
(dbs
   (3 +Role +User +Sal)                         # 512 Prevalent objects
   (0 +Pos)                                     # A:64 Tiny objects
   (1 +Item +Ord)                               # B:128 Small objects
   (2 +CuSu)                                    # C:256 Normal objects
   (2 (+Role nm) (+User nm) (+Sal nm))          # D:256 Small indexes
   (4 (+CuSu nr plz tel mob))                   # E:1024 Normal indexes
   (4 (+CuSu nm))                               # F:1024
   (4 (+CuSu ort))                              # G:1024
   (4 (+Item nr sup pr))                        # H:1024
   (4 (+Item nm))                               # I:1024
   (4 (+Ord nr dat cus))                        # J:1024
   (4 (+Pos itm)) )                             # K:1024

# vi:et:ts=3:sw=3


\end{wideverbatim}


\paragraph{Discussion}
\label{sec:appl-devel-data-model-discussion}


The data model for this mini application consists of only six entity
classes (see the E/R diagram at the beginning of ``app/er.l''):

\begin{itemize}
\item The three main entities are \texttt{+CuSu} (Customer/Supplier), \texttt{+Item}
   (Product Item) and \texttt{+Ord} (Order).
\item A \texttt{+Pos} object is a single position in an order.
\item \texttt{+Role} and \texttt{+User} objects are needed for authentication and
   authorization.
\end{itemize}

The classes \texttt{+Role} and \texttt{+User} are defined in ``@lib/adm.l''. A \texttt{+Role}
has a name, a list of permissions, and a list of users assigned to this
role. A \texttt{+User} has a name, a password and a role.

In ``app/er.l'', the \texttt{+Role} class is extended to define an \texttt{url>} method
for it. Any object whose class has such a method is able to display
itself in the GUI. In this case, the file ``app/role.l'' will be loaded -
with the global variable \texttt{*ID} pointing to it - whenever an HTML link to
this role object is activated.

The \texttt{+User} class is also extended. In addition to the login name, a
full name, telephone number and email address is declared. And, of
course, the ubiquitous \texttt{url>} method.

The application logic is centered around orders. An order has a number,
a date, a customer (an instance of \texttt{+CuSu}) and a list of positions
(\texttt{+Pos} objects). The \texttt{sum>} method calculates the total amount of this
order.

Each position has an \texttt{+Item} object, a price and a quantity. The price
in the position overrides the default price from the item.

Each item has a number, a description, a supplier (also an instance of
\texttt{+CuSu}), an inventory count (the number of these items that were
counted at the last inventory taking), and a price. The \texttt{cnt>} method
calculates the current stock of this item as the difference of the
inventory and the sold item counts.

The call to \texttt{dbs} at the end of ``app/er.l'' configures the physical
database storage. Each of the supplied lists has a number in its CAR
which determines the block size as (64 << N) of the corresponding
database file. The CDR says that the instances of this class (if the
element is a class symbol) or the tree nodes (if the element is a list
of a class symbol and a property name) are to be placed into that file.
This allows for some optimizations in the database layout.

 
\subsection{Usage}
\label{sec:appl-devel-usage}

When you are connected to the application (see \emph{Getting Started})
you might try to do some ``real'' work with it. Via the ``Data'' menu
(see \emph{Navigation}) you can create or modify customers, suppliers,
items and orders, and produce simple overviews via the ``Report''
menu.
 

\subsubsection{ Customer/Supplier}
\label{sec:appl-devel-customer/supplier}%

\paragraph{Source Code}
\label{sec:appl-devel-customer/supplier-source-code}

\begin{wideverbatim}

# 05nov09abu
# (c) Software Lab. Alexander Burger

(must "Customer/Supplier" Customer)

(menu ,"Customer/Supplier"
   (ifn *ID
      (prog
         (<h3> NIL ,"Select" " " ,"Customer/Supplier")
         (form 'dialog (choCuSu)) )
      (<h3> NIL ,"Customer/Supplier")
      (form NIL
         (<h2> NIL (<id> (: nr) " -- " (: nm)))
         (panel T (pack ,"Customer/Supplier" " @1") '(may Delete) '(choCuSu) 'nr '+CuSu)
         (<hr>)
         (<tab>
            (,"Name"
               (<grid> 3
                  ,"Number" NIL (gui '(+E/R +NumField) '(nr : home obj) 10)
                  ,"Salutation"
                  (gui '(+Hint) ,"Salutation"
                     '(mapcar '((This) (cons (: nm) This)) (collect 'nm '+Sal)) )
                  (gui '(+Hint2 +E/R +Obj +TextField) '(sal : home obj) '(nm +Sal) 20)
                  ,"Name" NIL (gui '(+E/R +Cue +TextField) '(nm : home obj) ,"Name" 40)
                  ,"Name 2" NIL (gui '(+E/R +TextField) '(nm2 : home obj) 40) ) )
            (,"Address"
               (<grid> 2
                  ,"Street" (gui '(+E/R +TextField) '(str : home obj) 40)
                  NIL NIL
                  ,"Zip" (gui '(+E/R +TextField) '(plz : home obj) 10)
                  ,"City" (gui '(+E/R +TextField) '(ort : home obj) 40) ) )
            (,"Contact"
               (<grid> 2
                  ,"Phone" (gui '(+E/R +TelField) '(tel : home obj) 40)
                  ,"Fax" (gui '(+E/R +TelField) '(fax : home obj) 40)
                  ,"Mobile" (gui '(+E/R +TelField) '(mob : home obj) 40)
                  ,"EMail" (gui '(+E/R +MailField) '(em : home obj) 40) ) )
            ((pack (and (: obj txt) "@ ") ,"Memo")
               (gui '(+BlobField) '(txt : home obj) 60 8) ) )
         (<hr>)
         (<spread> NIL (editButton T)) ) ) )

# vi:et:ts=3:sw=3

\end{wideverbatim}


\paragraph{Discussion}
\label{sec:appl-devel-customer/supplier-discussion}

The Customer/Supplier search dialog (\texttt{choCuSu} in ``app/gui.l'') supports
a lot of search criteria. These become necessary when the database
contains a large number of customers, and can filter by zip, by phone
number prefixes, and so on.

In addition to the basic layout (see \emph{Editing}), the form is
divided into four separate tabs. Splitting a form into several tabs
helps to reduce traffic, with possibly better GUI response. In this
case, four tabs are perhaps overkill, but ok for demonstration purposes,
and they leave room for extensions.

Be aware that when data were modified in one of the tabs, the ``Done''
button has to be pressed before another tab is clicked, because tabs are
implemented as HTML links (see \emph{Buttons vs. Links}).

New customers or suppliers will automatically be assigned the next free
number. You can enter another number, but an error will result if you
try to use an existing number. The ``Name'' field is mandatory, you need
to overwrite the ``<Name>'' clue.

Phone and fax numbers in the ``Contact'' tab must be entered in the
correct format, depending on the locale (see \emph{Telephone Numbers}).

The ``Memo'' tab contains a single text area. It is no problem to use it
for large pieces of text, as it gets stored in a database blob
internally.


\subsubsection{ Item}
\label{sec:appl-devel-item}%


\paragraph{Source Code}
\label{sec:appl-devel-item-source-code}

\begin{wideverbatim}

# 09aug10abu
# (c) Software Lab. Alexander Burger

(must "Item" Item)

(menu ,"Item"
   (ifn *ID
      (prog
         (<h3> NIL ,"Select" " " ,"Item")
         (form 'dialog (choItem)) )
      (<h3> NIL ,"Item")
      (form NIL
         (<h2> NIL (<id> (: nr) " -- " (: nm)))
         (panel T (pack ,"Item" " @1") '(may Delete) '(choItem) 'nr '+Item)
         (<grid> 4
            ,"Number" NIL (gui '(+E/R +NumField) '(nr : home obj) 10) NIL
            ,"Description" NIL (gui '(+E/R +Cue +TextField) '(nm : home obj) ,"Item" 30) NIL
            ,"Supplier" (gui '(+ChoButton) '(choCuSu (field 1)))
            (gui '(+E/R +Obj +TextField) '(sup : home obj) '(nm +CuSu) 30)
            (gui '(+View +TextField) '(field -1 'obj 'ort) 30)
            ,"Inventory" NIL (gui '(+E/R +NumField) '(inv : home obj) 12)
            (gui '(+View +NumField) '(cnt> (: home obj)) 12)
            ,"Price" NIL (gui '(+E/R +FixField) '(pr : home obj) 2 12) )
         (--)
         (<grid> 2
            ,"Memo" (gui '(+BlobField) '(txt : home obj) 60 8)
            ,"Picture"
            (prog
               (gui '(+Able +UpField) '(not (: home obj jpg)) 30)
               (gui '(+Drop +Button) '(field -1)
                  '(if (: home obj jpg) ,"Uninstall" ,"Install")
                  '(cond
                     ((: home obj jpg)
                        (ask ,"Uninstall Picture?"
                           (put!> (: home top 1 obj) 'jpg NIL) ) )
                     ((: drop) (blob! (: home obj) 'jpg @)) ) ) ) )
         (<spread> NIL (editButton T))
         (gui '(+Img)
            '(and (: home obj jpg) (allow (blob (: home obj) 'jpg)))
            ,"Picture") ) ) )

# vi:et:ts=3:sw=3


\end{wideverbatim}

\paragraph{Discussion}
\label{sec:appl-devel-data-model-discussion}

Items also have a unique number, and a mandatory ``Description'' field.

To assign a supplier, click on the ``+'' button. The Customer/Supplier
search dialog will appear, and you can pick the desired supplier with
the ``@'' button in the first column. Alternatively, if you are sure to
know the exact spelling of the supplier's name, you can also enter it
directly into the text field.

In the search dialog you may also click on a link, for example to
inspect a possible supplier, and then return to the search dialog with
the browser's back button. The ``Edit'' mode will then be lost, however,
as another object has been visited (this is described in the last part
of \emph{Editing}).

You can enter an inventory count, the number of items currently in
stock. The following field will automatically reflect the remaining
pieces after some of these items were sold (i.e. referenced in order
positions). It cannot be changed manually.

The price should be entered with the decimal separator according to the
current locale. It will be formatted with two places after the decimal
separator.

The ``Memo'' is for an arbitrary info text, like in
\emph{Customer/Supplier} above, stored in a database blob.

Finally, a JPEG picture can be stored in a blob for this item. Choose a
file with the browser's file select control, and click on the ``Install''
button. The picture will appear at the bottom of the page, and the
``Install'' button changes to ``Uninstall'', allowing the picture's removal.

\subsubsection{ Order}
\label{sec:appl-devel-order}%

\paragraph{Source Code}
\label{sec:appl-devel-order-source-code}


\begin{wideverbatim}

# 03sep09abu
# (c) Software Lab. Alexander Burger

(must "Order" Order)

(menu ,"Order"
   (ifn *ID
      (prog
         (<h3> NIL ,"Select" " " ,"Order")
         (form 'dialog (choOrd)) )
      (<h3> NIL ,"Order")
      (form NIL
         (<h2> NIL (<id> (: nr)))
         (panel T (pack ,"Order" " @1") '(may Delete) '(choOrd) 'nr '+Ord)
         (<grid> 4
            ,"Date" NIL
            (gui '(+E/R +DateField) '(dat : home obj) 10)
            (gui '(+View +TextField)
               '(text ,"(@1 Positions)" (length (: home obj pos))) )
            ,"Customer" (gui '(+ChoButton) '(choCuSu (field 1)))
            (gui '(+E/R +Obj +TextField) '(cus : home obj) '(nm +CuSu) 30)
            (gui '(+View +TextField) '(field -1 'obj 'ort) 30) )
         (--)
         (gui '(+Set +E/R +Chart) '((L) (filter bool L)) '(pos : home obj) 8
            '((Pos I)
               (with Pos
                  (list I NIL (: itm) (or (: pr) (: itm pr)) (: cnt) (sum> Pos)) ) )
            '((L D)
               (cond
                  (D
                     (put!> D 'itm (caddr L))
                     (put!> D 'pr (cadddr L))
                     (put!> D 'cnt (; L 5))
                     (and (; D itm) D) )
                  ((caddr L)
                     (new! '(+Pos) 'itm (caddr L)) ) ) ) )


\end{wideverbatim}

\begin{wideverbatim}


         (<table> NIL NIL
            '((align) (btn) (NIL ,"Item") (NIL ,"Price") (NIL ,"Quantity") (NIL ,"Total"))
            (do 8
               (<row> NIL
                  (gui 1 '(+NumField))
                  (gui 2 '(+ChoButton) '(choItem (field 1)))
                  (gui 3 '(+Obj +TextField) '(nm +Item) 30)
                  (gui 4 '(+FixField) 2 12)
                  (gui 5 '(+NumField) 8)
                  (gui 6 '(+Sgn +Lock +FixField) 2 12)
                  (gui 7 '(+DelRowButton))
                  (gui 8 '(+BubbleButton)) ) )
            (<row> NIL NIL NIL (scroll 8 T) NIL NIL
               (gui '(+Sgn +View +FixField) '(sum> (: home obj)) 2 12) ) )
         (<spread>
            (gui '(+Rid +Button) ,"PDF-Print"
               '(if (check> (: home obj))
                  (note ,"Can't print order" (uniq @))
                  (psOut 0 ,"Order" (ps> (: home obj))) ) )
            (editButton T) ) ) ) )

# vi:et:ts=3:sw=3


\end{wideverbatim}


\paragraph{Discussion}
\label{sec:appl-devel-order-discussion}

Oders are identified by number and date.

The number must be unique. It is assigned when the order is created, and
cannot be changed for compliance reasons.

The date is initialized to ``today'' for a newly created order, but may be
changed manually. The date format depends on the locale. It is
YYYY-MM-DD (ISO) by default, DD.MM.YYYY in the German and YYYY/MM/DD in
the Japanese locale. As described in \emph{Time \& Date},
this field allows input shortcuts, e.g. just enter the day to get the
full date in the current month.

To assign a customer to this order, click on the ``+'' button. The
Customer/Supplier search dialog will appear, and you can pick the
desired customer with the ``@'' button in the first column (or enter the
name directly into the text field), just as described above for
\emph{Item}s.

Now enter order the positions: Choose an item with the ``+'' button. The
``Price'' field will be preset with the item's default price, you may
change it manually. Then enter a quantity, and click a button (typically
the ``+'' button to select the next item, or a scroll button go down in
the chart). The form will be automatically recalculated to show the
total prices for this position and the whole order.

Instead of the ``+'' or scroll buttons, as recommended above, you could of
course also press the ``Done'' button to commit changes. This is all
right, but has the disadvantage that the button must be pressed a second
time (now ``Edit'') if you want to continue with the entry of more
positions.

The ``x'' button at the right of each position deletes that position
without further confirmation. It has to be used with care!

The ``\^'' button is a ``bubble'' button. It exchanges a row with the row
above it. Therefore, it can be used to rearrange all items in a chart,
by ``bubbling'' them to their desired positions.

The ``PDF-Print'' button generates and displays a PDF document for this
order. The browser should be configured to display downloaded PDF
documents in an appropriate viewer. The source for the postscript
generating method is in ``app/lib.l''. It produces one or several A4 sized
pages, depending on the number of positions.


\subsubsection{ Reports}
\label{sec:appl-devel-reports}%

\paragraph{Source Code}
\label{sec:appl-devel-reports-source-code}


\begin{wideverbatim}

# 08mar10abu
# (c) Software Lab. Alexander Burger

(must "Inventory" Report)

(menu ,"Inventory"
   (<h3> NIL ,"Inventory")
   (form NIL
      (<grid> "-.-"
         ,"Number" NIL
         (prog
            (gui '(+Var +NumField) '*InvFrom 10)
            (prin " - ")
            (gui '(+Var +NumField) '*InvTill 10) )
         ,"Description" NIL (gui '(+Var +TextField) '*InvNm 30)
         ,"Supplier" (gui '(+ChoButton) '(choCuSu (field 1)))
         (gui '(+Var +Obj +TextField) '*InvSup '(nm +CuSu) 30) )
      (--)
      (gui '(+ShowButton) NIL
         '(csv ,"Inventory"
            (<table> 'chart NIL
               (<!>
                  (quote
                     (align)
                     (NIL ,"Description")
                     (align ,"Inventory")
                     (NIL ,"Supplier")
                     NIL
                     (NIL ,"Zip")
                     (NIL ,"City")
                     (align ,"Price") ) )
               (catch NIL
                  (pilog
                     (quote
                        @Rng (cons *InvFrom (or *InvTill T))
                        @Nm *InvNm
                        @Sup *InvSup
                        (select (@Item)
                           ((nr +Item @Rng) (nm +Item @Nm) (sup +Item @Sup))
                           (range @Rng @Item nr)
                           (tolr @Nm @Item nm)
                           (same @Sup @Item sup) ) )

\end{wideverbatim}

\begin{wideverbatim}
                     (with @Item
                        (<row> (alternating)
                           (<+> (: nr) This)
                           (<+> (: nm) This)
                           (<+> (cnt> This))
                           (<+> (: sup nm) (: sup))
                           (<+> (: sup nm2))
                           (<+> (: sup plz))
                           (<+> (: sup ort))
                           (<-> (money (: pr))) ) )
                     (at (0 . 10000) (or (flush) (throw))) ) ) ) ) ) ) )
# vi:et:ts=3:sw=3
\end{wideverbatim}


\begin{wideverbatim}

# 08mar10abu
# (c) Software Lab. Alexander Burger

(must "Sales" Report)

(menu ,"Sales"
   (<h3> NIL ,"Sales")
   (form NIL
      (<grid> "-.-"
         ,"Date" NIL
         (prog
            (gui '(+Var +DateField) '*SalFrom 10)
            (prin " - ")
            (gui '(+Var +DateField) '*SalTill 10) )
         ,"Customer" (gui '(+ChoButton) '(choCuSu (field 1)))
         (gui '(+Var +Obj +TextField) '*SalCus '(nm +CuSu) 30) )
      (--)
      (gui '(+ShowButton) NIL
         '(csv ,"Sales"
            (<table> 'chart NIL
               (<!>
                  (quote
                     (align)
                     (NIL ,"Date")
                     (NIL ,"Customer")
                     NIL
                     (NIL ,"Zip")
                     (NIL ,"City")
                     (align ,"Total") ) )

\end{wideverbatim}

\begin{wideverbatim}

               (catch NIL
                  (let Sum 0
                     (pilog
                        (quote
                           @Rng (cons *SalFrom (or *SalTill T))
                           @Cus *SalCus
                           (select (@Ord)
                              ((dat +Ord @Rng) (cus +Ord @Cus))
                              (range @Rng @Ord dat)
                              (same @Cus @Ord cus) ) )
                        (with @Ord
                           (let N (sum> This)
                              (<row> (alternating)
                                 (<+> (: nr) This)
                                 (<+> (datStr (: dat)) This)
                                 (<+> (: cus nm) (: cus))
                                 (<+> (: cus nm2))
                                 (<+> (: cus plz))
                                 (<+> (: cus ort))
                                 (<-> (money N)) )
                              (inc 'Sum N) ) )
                        (at (0 . 10000) (or (flush) (throw))) )
                     (<row> 'nil
                        (<strong> ,"Total") - - - - -
                        (<strong> (prin (money Sum))) ) ) ) ) ) ) ) )

# vi:et:ts=3:sw=3


\end{wideverbatim}

\paragraph{Discussion}
\label{sec:appl-devel-reports-discussion}

The two reports (``Inventory'' and ``Sales'') come up with a few search
fields and a ``Show'' button.

If no search criteria are entered, the ``Show'' button will produce a
listing of the relevant part of the whole database. This may take a long
time and cause a heavy load on the browser if the database is large.

So in the normal case, you will limit the domain by stating a range of
item numbers, a description pattern, and/or a supplier for the inventory
report, or a range of order dates and/or a customer for the sales
report. If a value in a range specification is omitted, the range is
considered open in that direction.

At the end of each report appears a ``CSV'' link. It downloads a file with
the TAB-separated values generated by this report.


