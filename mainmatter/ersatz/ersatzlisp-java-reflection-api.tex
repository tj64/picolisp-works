\title{Ersatz PicoLisp Java Reflection API}
\author{Alexander Burger}
% Use \authorrunning{Short Title} for an abbreviated version of
% your contribution title if the original one is too long
\institute{\texttt{abu@software-lab.de}}
%
% Use the package "url.sty" to avoid
% problems with special characters
% used in your e-mail or web address
%


\maketitle

\begin{abstract}
This article introduces the \emph{Ersatz PicoLisp Java Reflection API}
and its most important functions. 
\end{abstract}

\section{Introduction}
\label{sec:ersatz-java-reflection-introduction}

As Ersatz Lisp (available in the ``ersatz/'' directory in the Picolisp
\href{http://software-lab.de/picoLisp.tgz}{release}, or as a separate
\href{http://software-lab.de/ersatz.tgz}{tarball}) is completely
written in Java, it comes with a set of dedicated functions to
interface to the Java Reflection API.


\section{Important functions}
\label{sec:ersatz-java-reflection-important-functions}


\subsection{The \texttt{java} function}
\label{sec:ersatz-java-reflection-the-java-function}

The central function is \texttt{java}. It comes in four forms:

\begin{wideverbatim}
 (java `cls `T ['any ..]) -> obj
\end{wideverbatim}

is a constructor call, returning a Java object. \texttt{cls} is the
name of a Java class. The optional \texttt{any} arguments are passed
to the Java constructor.

\begin{wideverbatim}
 (java `cls `msg ['any ..]) -> obj
\end{wideverbatim}

calls a static method \texttt{msg} for a class \texttt{cls}.

\begin{wideverbatim}
 (java `obj `msg ['any ..]) -> obj
\end{wideverbatim}

calls a dynamic method \texttt{msg} for an object \texttt{obj}. The
optional any arguments to the above three forms can be

\begin{itemize}
\item T or NIL, then the type passed to Java is boolean
\item A number, then it must fit in 32 bits and will be passed as int
\item A normal symbol, then it will be passed as String
\item A Java object, as returned by java or one of the xxx: conversion
   functions (see below).
\item A list, then all elements must be of the same type and will be passed
   as Array.
\end{itemize}


\begin{wideverbatim}
(java `obj ['cnt]) -> any
\end{wideverbatim}

converts a Java object to the corresponding Lisp type. Supported
types, and their conversion results are:

\begin{itemize}
\item Boolean objects are converted to T or NIL
\item Byte, Character, Integer, Long or BigInteger objects are converted to
   numbers
\item Double objects are converted to fixnums, using the scale cnt
\item String objects are converted to transient symbols
\item Arrays of byte, char, int or long are converted to lists of numbers
\item Arrays of double are converted lists of fixnums, using the scale cnt
\end{itemize}

Example:

\begin{wideverbatim}
: (setq Sb (java "java.lang.StringBuilder" T "abc"))
-> $StringBuilder
: (java Sb "append" (char: 44))
-> $StringBuilder
: (java Sb "append" 123)
-> $StringBuilder
: (java Sb "toString")
-> $String
: (java @)
-> "abc,123"
\end{wideverbatim}


\begin{wideverbatim}
: (setq S (java "java.lang.String" T "abcde"))
-> $String
: (java (java S "getBytes"))
-> (97 98 99 100 101)
\end{wideverbatim}


\begin{wideverbatim}
: (java "java.lang.String" T (mapcar byte: (100 101 102)))
-> $String
: (java @)
-> "def"
\end{wideverbatim}

\subsection{The \texttt{public} function}
\label{sec:ersatz-java-reflection-the-public-function}

To access public fields in Java objects or classes, public can be
used:

\begin{wideverbatim}
 (public `obj `any ['any ..]) -> obj
\end{wideverbatim}

Returns the value of a public field any in object obj.

\begin{wideverbatim}
 (public `cls `any ['any ..]) -> obj
\end{wideverbatim}

Returns the value of a public field \texttt{any} in class
\texttt{cls}. In both forms, the optional \texttt{any} arguments will in turn
access corresponding fields in the object retrieved so far.

Example:

\begin{wideverbatim}
: (public "java.lang.System" "err")
-> $PrintStream
: (java @ "println" "Hello world")
Hello world
-> NIL
\end{wideverbatim}

\subsection{The \texttt{interface} function}
\label{sec:ersatz-java-reflection-the-interface-function}

To create an interface object:

\begin{wideverbatim}
 (interface `cls|lst `sym `fun ..) -> obj
\end{wideverbatim}

Creates an interface, i.e. a set of methods for a class \texttt{cls} (or a list
of classes \texttt{lst}).

Example:

\begin{wideverbatim}
#!ersatz/pil
(let
   (Frame (java "javax.swing.JFrame" T "Bye-Frame")
      Button (java "javax.swing.JButton" T "OK") )
   (java Frame "add" "South" Button)
   (java Button "addActionListener"
      (interface "java.awt.event.ActionListener"  # When button is clicked,
         'actionPerformed '((Ev) (bye)) ) )       # Exit PicoLisp
   (java Frame "setSize" 100 60)
   (java Frame "setVisible" T) )
\end{wideverbatim}

\subsection{Type conversion functions}
\label{sec:ersatz-java-reflection-type-conversion-functions}

Finally, we have a set of type conversion functions, to produce Java
objects from Lisp data:

\begin{wideverbatim}
(byte: `num|sym) -> obj
(char: `num|sym) -> obj
(int: `num) -> obj
(long: `num) -> obj
(double: `str `cnt) -> obj
(double: `num `cnt) -> obj
(big: `num) -> obj
\end{wideverbatim}

For a more elaborated example, take a look at the article
\hyperref[wiki-SwingRepl]{Swing REPL written in Ersatz PicoLisp}.

