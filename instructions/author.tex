%%%%%%%%%%%%%%%%%%%%%%%%%% author.tex %%%%%%%%%%%%%%%%%%%%%%%%%
%
% sample root file for your contribution to a "contributed book"
%
% "contributed book"
%
% Use this file as a template for your own input.
%
%%%%%%%%%%%%%%%%%%%%%%%% Springer-Verlag %%%%%%%%%%%%%%%%%%%%%%%%%%


% RECOMMENDED %%%%%%%%%%%%%%%%%%%%%%%%%%%%%%%%%%%%%%%%%%%%%%%%%%%
\documentclass{svmult}

% choose options for [] as required from the list
% in the Reference Guide, Sect. 2.2

\usepackage{makeidx}         % allows index generation
\usepackage{graphicx}        % standard LaTeX graphics tool
                             % when including figure files
\usepackage{multicol}        % used for the two-column index
\usepackage[bottom]{footmisc}% places footnotes at page bottom
% etc.
% see the list of further useful packages
% in the Reference Guide, Sects. 2.3, 3.1-3.3

\makeindex             % used for the subject index
                       % please use the style sprmidx.sty with
                       % your makeindex program


%%%%%%%%%%%%%%%%%%%%%%%%%%%%%%%%%%%%%%%%%%%%%%%%%%%%%%%%%%%%%%%%%%%%%

\begin{document}

\title*{Contribution Title}
% Use \titlerunning{Short Title} for an abbreviated version of
% your contribution title if the original one is too long
\author{Name of Author\inst{1}\and
Name of Author\inst{2}}
% Use \authorrunning{Short Title} for an abbreviated version of
% your contribution title if the original one is too long
\institute{Name and Address of your Institute
\texttt{name@email.address}
\and Name and Address of your Institute \texttt{name@email.address}}
%
% Use the package "url.sty" to avoid
% problems with special characters
% used in your e-mail or web address
%
\maketitle

Your text goes here. Separate text sections with the standard \LaTeX\
sectioning commands.

\section{Section Heading}
\label{sec:1}
% Always give a unique label
% and use \ref{<label>} for cross-references
% and \cite{<label>} for bibliographic references
% use \sectionmark{}
% to alter or adjust the section heading in the running head
Your text goes here. Use the \LaTeX\ automatism for your citations
\cite{monograph}.

\subsection{Subsection Heading}
\label{sec:2}
Your text goes here.

\begin{equation}
\vec{a}\times\vec{b}=\vec{c}
\end{equation}

\subsubsection{Subsubsection Heading}
Your text goes here. Use the \LaTeX\ automatism for cross-references as
well as for your citations, see Sect.~\ref{sec:1}.

\paragraph{Paragraph Heading} %
Your text goes here.

\subparagraph{Subparagraph Heading.} Your text goes here.%
%
\index{paragraph}
% Use the \index{} command to code your index words
%
% For tables use
%
\begin{table}
\centering
\caption{Please write your table caption here}
\label{tab:1}       % Give a unique label
%
% For LaTeX tables use
%
\begin{tabular}{lll}
\hline\noalign{\smallskip}
first & second & third  \\
\noalign{\smallskip}\hline\noalign{\smallskip}
number & number & number \\
number & number & number \\
\noalign{\smallskip}\hline
\end{tabular}
\end{table}
%
%
% For figures use
%
\begin{figure}
\centering
% Use the relevant command for your figure-insertion program
% to insert the figure file.
% For example, with the option graphics use
\includegraphics[height=4cm]{figure.eps}
%
% If not, use
%\picplace{5cm}{2cm} % Give the correct figure height and width in cm
%
\caption{Please write your figure caption here}
\label{fig:1}       % Give a unique label
\end{figure}
%
% For built-in environments use
%
\begin{theorem}
Theorem text\footnote{Footnote} goes here.
\end{theorem}
%
% or
%
\begin{lemma}
Lemma text goes here.
\end{lemma}
%
%
% BibTeX users please use
% \bibliographystyle{}
% \bibliography{}
%
% Non-BibTeX users please follow the syntax
% the syntax of "referenc.tex" for your own citations
\input{referenc}
%%%%%%%%%%%%%%%%%%%%%%%%%%%%%%%%%%%%%%%%%%%%%%%%%%%%%%%%%%%%%%%%%%%%%%

%%%%%%%%%%%%%%%%%%%%%%%%%%%%%%%%%%%%%%%%%%%%%%%%%%%%%%%%%%%%%%%%%%%%%%

\printindex
\end{document}





